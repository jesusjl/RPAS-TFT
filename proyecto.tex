
\documentclass[11pt,]{article}
\usepackage[left=1in,top=1in,right=1in,bottom=1in]{geometry}
\newcommand*{\authorfont}{\fontfamily{phv}\selectfont}
\usepackage[]{mathpazo}


  \usepackage[T1]{fontenc}
  \usepackage[utf8]{inputenc}
  \usepackage[nottoc,notlof,notlot]{tocbibind} 
  \renewcommand\refname{Referencias} % change original References title
  \renewcommand{\contentsname}{Tabla de contenidos  }



\usepackage{abstract}
\renewcommand{\abstractname}{}    % clear the title
\renewcommand{\absnamepos}{empty} % originally center

\renewenvironment{abstract}
 {{%
    \setlength{\leftmargin}{0mm}
    \setlength{\rightmargin}{\leftmargin}%
  }%
  \relax}
 {\endlist}

\makeatletter
\def\@maketitle{%
  \newpage
%  \null
%  \vskip 2em%
%  \begin{center}%
  \let \footnote \thanks
    {\fontsize{18}{20}\selectfont\raggedright  \setlength{\parindent}{0pt} \@title \par}%
}
%\fi
\makeatother




\setcounter{secnumdepth}{0}



\title{Contribución de los RPAS en investigación y conservación en espacios
protegidos: presente y futuro  }



\author{\Large Jesús Jiménez López\vspace{0.05in} \newline\normalsize\emph{Experto Universitario en Vehículos Aéreos no Tripulados y sus
Aplicaciones Civiles. Universidad de Cádiz}  }


\date{}

\usepackage{titlesec}

\titleformat*{\section}{\LARGE\bfseries}
\titleformat*{\subsection}{\normalsize\bfseries}
\titleformat*{\subsubsection}{\normalsize\itshape}
\titleformat*{\paragraph}{\normalsize\itshape}
\titleformat*{\subparagraph}{\normalsize\itshape}


\usepackage{natbib}
\bibliographystyle{apsr}


\newtheorem{hypothesis}{Hypothesis}
\usepackage{setspace}

\makeatletter
\@ifpackageloaded{hyperref}{}{%
\ifxetex
  \usepackage[setpagesize=false, % page size defined by xetex
              unicode=false, % unicode breaks when used with xetex
              xetex]{hyperref}
\else
  \usepackage[unicode=true]{hyperref}
\fi
}
\@ifpackageloaded{color}{
    \PassOptionsToPackage{usenames,dvipsnames}{color}
}{%
    \usepackage[usenames,dvipsnames]{color}
}
\makeatother
\hypersetup{breaklinks=true,
            bookmarks=true,
            pdfauthor={Jesús Jiménez López (Experto Universitario en Vehículos Aéreos no Tripulados y sus
Aplicaciones Civiles. Universidad de Cádiz)},
             pdfkeywords = {RPAs, espacios naturales, conservación},  
            pdftitle={Contribución de los RPAS en investigación y conservación en espacios
protegidos: presente y futuro},
            colorlinks=true,
            citecolor=blue,
            urlcolor=blue,
            linkcolor=magenta,
            pdfborder={0 0 0}}

\urlstyle{same}  % don't use monospace font for urls



\begin{document}
	
% \pagenumbering{arabic}% resets `page` counter to 1 
%
% \maketitle

{% \usefont{T1}{pnc}{m}{n}
\setlength{\parindent}{0pt}
\thispagestyle{plain}
{\fontsize{18}{20}\selectfont\raggedright 
\maketitle  % title \par  

}

{
   \vskip 13.5pt\relax \normalsize\fontsize{11}{12} 
\textbf{\authorfont Jesús Jiménez López} \hskip 15pt \vskip 8.5pt  \emph{\small Experto Universitario en Vehículos Aéreos no Tripulados y sus
Aplicaciones Civiles. Universidad de Cádiz}   

}

}






\begin{abstract}

    \hbox{\vrule height .2pt width 39.14pc}

    \vskip 8.5pt % \small 

\noindent En este estudio se procedió a identificar y resumir el estado actual y
las tendencias en el uso de los drones en proyectos científicos con
fines de conservación en espacios naturales protegidos, mediante la
recopilación y revisión de material bibliográfico en forma de artículos
científicos, revistas, proyectos de conservación publicados en Internet
y otras fuentes de información relevantes.


\vskip 8.5pt \noindent \emph{Palabras claves}: RPAs, espacios naturales, conservación \par

    \hbox{\vrule height .2pt width 39.14pc}



\end{abstract}


{
\hypersetup{linkcolor=black}
\setcounter{tocdepth}{2}
\tableofcontents
}


\vskip 6.5pt

\noindent \doublespacing \section{Introducción}\label{introduccion}

Introducción al uso de los RPAs en temas de conservación en espacios
protegidos, agrupados en los principales campos de aplicación y líneas
de investigación que conducen a nuevas metodologías de aplicación.

Las aplicaciones de los vehículos aéreos no tripulados (RPAs o UAVs) en
el campo de la conservación han sido directa o indirectamente planteadas
en un número cada vez mayor de artículos científicos. Durante los
últimos años ha habido un incremento significativo de las líneas de
investigación sobre vida silvestre que hacen uso de RPAS
\citep{Linchant2015}. En el campo de la biología de la conservación, el
auge de estos equipos ha conducido a un mayor desarrollo de las
metodologías que tratan de complementar o sustituir los medios más
tradicionales con los que se ha venido llevando a cabo las diferentes
actividades enfocadas hacia el manejo de los espacios naturales
protegidos, en sus diversas facetas. Aunque el número de áreas
protegidas ha experimentado un aumento sensible a nivel mundial, con un
12 \% de la superficie terrestre bajo alguna figura de protección,
algunos autores resaltan la necesidad de mejorar las herramientas para
asegurar la efectividad de la conservación de la biodiversidad en áreas
protegidas \citep{Chape2005}.

Existen actualmente algunas iniciativas que tratan de recoger el estado
actual de los RPAS en las áreas de ecología y conservación. Con fecha
reciente de finalización , la revista \emph{Remote Sensing in Ecology
and Conservation} hizo una llamada a la comunidad científica para el
envío de propuestas dentro de la temática mencionada, con objeto de
sensibilizar a estudiantes y profesionales y demostrar el uso
responsable de RPAS. Por otro lado, es remarcable la mayor presencia de
portales en internet que centran su actividad en torno a las
aplicaciones con RPAS. En el campo de la investigación aplicada en
conservación, el sitio referencia es
\url{http://conservationdrones.org/}, cuya misión se resumen en
facilitar el uso y desarrollo de RPAs con tal propósito y en el cual se
pueden consultar casos de usos de RPAS, los cuales no siempre aparecen
reflejados en artículos científicos. Dentro de las aplicaciones de los
RPAS con caracter general destaca la comunidad online
\url{http://diydrones.com/}, en la que tiene gran acogida el uso de
plataformas abiertas, de gran popularidad frente a los tradicionales
sistemas cerrados promovidos por compañias comerciales del sector. Esto
ha dado como resultado la reducción de los costes de estos equipos,
junto con el software asociado, permitiendo acercar la tecnología
disponible a un mayor número de usuarios y organizaciones. Estas
plataformas abiertas tienen la ventaja adicional de tener un mayor grado
de personalización de los equipos. El incremento en la flexibilidad en
el montaje de diferentes sensores y sistemas de control permite cubrir
las necesidades específicas de cada proyecto \citep{Koh2012}.

Por otra parte, este estudio se analiza el grado en el que los RPAS
están llamados a sustituir herramientas tradicionales de apoyo a la
conservación, tanto en su vertiente científica como conservacionista. En
este sentido, las limitaciones desde el punto de vista financiero y
tecnológico de la teledetección, por la cual se obtienen imágenes de la
superficie terrestre a partir de sensores instalados en plataformas
aéreas o espaciales, son descritas por diversos autores \citep{Koh2012}.
Si bien es posible adquirir imágenes satelitales a coste cero (LandSat,
MODIS, Sentinel, etc.) en ocasiones la resolución espacial y temporal
necesaria para este tipo de estudios, junto con los problemas de
presencia de nubes especialmente acusados en zonas tropicales, reduce la
efectividad de la teledetección como herramienta de apoyo a la
conservación. En paises en vías de desarrollo, especialmente sensibles
en cuanto a dotaciones presupuestarias, se han desarrollado con gran
éxito programas de monitoreo y vigilancia a partir del uso del RPAS,
eliminando los inconvenientes descritos con anterioridad. Además, el
gran tamaño de estas áreas protegidas reducen en muchos casos la
efectividad de los trabajos de campo, por lo que los RPAS se han
posicionado como un complemento adecuado para las actividades de
conservación \citep{Zahawi2015}.

Junto con la mejora en los costes y la reducción de la logística
necesaria mediante el empleo de los RPAS en el diseño y planificación de
los proyectos de investigación frente a los levantamientos aéreos
convencionales, existe una sensibilización cada vez mayor en relación a
la mejora de la seguridad de los biólogos en las operaciones de campo.
Algunos estudios señalan que los accidentes aéreos se situan como
primera causa de mortandad en especialistas en vida silvestre en los
Estados Unidos \citep{Sasse2003}.

Finalmente, algunos autores señalan la necesidad de mejorar el marco
regulatorio respecto al uso civil de los RPAS \citep{Nugraha2016}. En
los Estados Unidos y en la mayoría de los paises de Europa consultados,
se han adoptado leyes provisionales que en cierta medida equiparan el
manejo de los RPAS con el de aeronaves tradicionales. Este tipo de
restricciones podría limitar las posibilidades de uso de los RPAS en el
ámbito de la conservación, por lo que se hace patente la necesidad
urgente de armonizar la legislación en relación a este tipo de
actividades. En términos generales, la situación en America Latina es
desigual, con algunos paises que siguen sin desarrollar leyes
específicas para hacer frente al auge de los RPAS tanto en el sector
civil como militar.

La incertidumbre de los usuarios ha promovido el desarrollo de
asociaciones con objeto de asesorar sobre los aspectos legales a tener
en cuenta durante la operación. En España, la Asociación Española de
Drones y Afines \url{https://www.aedron.com} promueve un uso consciente
y responsable de los RPAS y organiza seminarios para informar a los
socios sobre temas de interés. En su web se puede consultar el borrador
de la nueva normativa que regula la utilización civil de las aeronaves
pilotadas por control remoto en España.

\section{Métodos}\label{metodos}

Descripción de la metodología para la identificación de las referencias
bibliográficas más emblemáticas, clasificación según temática (líneas de
investigación), criterios de selección de artículos relevantes.

Para alcanzar los objetivos propuestos se procedió a la revisión
bibliográfica de artículos, tesis de postgrado, sitios web y revistas
especializadas, siguiendo una línea similar a otros estudios realizados
con anterioridad \citet{Linchant2015}. Mediante artículos seleccionados
para el curso de Experto Universitario en Vehículos Aéreos no Tripulados
y sus Aplicaciones Civiles organizado por la Universidad de Cádiz en su
edición de 2016-2017, junto con herramientas como Google Schoolar,
ResearchGate y Mendeley Desktop se obtuvo la mayor parte de la
bibliografía seleccionada, mientras que el uso de los motores de
búsqueda por internet incluyeron el resto de materiales mencionados.
Dicha actividad tuvo lugar hasta el mes de Abril, 2017.

La información recolectada se clasificó en tres categorías principales
que son objeto de la conservación en espacios protegidos: 1. Estudios de
vida silvestre, 2. Monitoreo y mapeo de ecosistemas terrestres y
acuáticos y 3. Apoyo para el cumplimiento de las leyes en áreas
protegidas, existiendo cierto lógico solape entre categorías dado el
caracter multidisciplinar de estos estudios. Otros proyectos que no
entraron dentro de alguna de las tres categorías propuestas se
clasificaron en un solo grupo. Finalmente se identificó el propósito
principal de cada estudio, junto con las técnicas y materiales
empleados.

\section{Discusión}\label{discusion}

\subsection{Estudios de vida
silvestre}\label{estudios-de-vida-silvestre}

\subsubsection{Estudios de poblaciones}\label{estudios-de-poblaciones}

Actualmente se experimenta un incremento de los trabajos de
investigación que incorporan el uso de RPAS en la disciplina de ecología
de poblaciones. Algunos estudios comparan el uso de drones para el
desarrollo de modelos de distribución de especie y caracterización del
habitat de las especies objetivos frente a los sistemas de seguimiento
por satélite o radiocontrol, que permiten registrar el movimiento del
animal para su análisis posterior \citep{PazmanyMulero2015},
\citep{Mulero-Pazmany2015}. En determinados casos, frente a las
dificultades para detectar directamente a la especie de interés, los
estudios se enfocan en la localización y caracterización de sus áreas de
cría y nidificación \citep{VanAndel2015}. En grandes extensiones de
terreno se ha ensayado con éxito el conteo de grandes mamíferos,
mientras que no se han registrado reacciones adversas en vuelos
realizados a cierta altura \citep{Schiffman2014}. Otros ensayos se
dirigen al desarrollo de algoritmos que permitan contar con exactitud el
número de individuos capturados por los dispositivos fotográficos,
teniendo en cuenta la masiva cantidad de información que el uso de RPAS
puede generar, con la consiguiente dificultad para la interpretación
visual manual de las escenas adquiridas
\citep{Lhoest2015}\citep{Abd-Elrahman2005a}.

Fuera de la literatura científica, existen proyectos para el monitoreo
de la fauna tanto en ecosistemas marinos como terrestres. A partir de la
información recopilada en la comunidad online
\url{https://conservationdrones.org} se han identificado varios estudios
relacionados con el registro de individuos en poblaciones situadas en
áreas protegidas o frecuentemente visitadas por estas especies. La
mayoría de estos trabajos están respaldados por organizaciones no
gubernamentales y centros de investigación. Por ejemplo, un estudio
realizado en la cuenca del Amazonas en Brasil está experimentando el uso
de drones para mejorar la estimación de la densidad y abundancia de
diferentes especies de delfines, en comparación con la observación
directa realizada por especialistas. Dentro de los objetivos de la
investigación se contempla la validación y armonización de ambas
metodologías e indirectamente evaluar su viabilidad para la aplicación
regular en proyectos de monitoreo con similar propósito, teniendo en
cuenta el coste-beneficio de la ejecución.

\subsubsection{Evaluación de
infraestructuras}\label{evaluacion-de-infraestructuras}

Otros trabajos resaltan la utilidad de los RPAS en la caracterización de
infraestructuras humanas y su relación con el impacto ambiental sobre
especies vulnerables.

\subsubsection{Ecología espacial}\label{ecologia-espacial}

Mapa de zonas

\subsection{Monitoreo y mapeo de ecosistemas terrestres y
acuáticos}\label{monitoreo-y-mapeo-de-ecosistemas-terrestres-y-acuaticos}

\subsubsection{Ecosistemas acuáticos}\label{ecosistemas-acuaticos}

\subsubsection{Ecosistemas terrestres}\label{ecosistemas-terrestres}

\subsection{Apoyo para el cumplimiento de las leyes en áreas
protegidas}\label{apoyo-para-el-cumplimiento-de-las-leyes-en-areas-protegidas}

\subsubsection{Caza ilegal}\label{caza-ilegal}

\subsubsection{Otras actividades
ilegales}\label{otras-actividades-ilegales}

\section{Resultados}\label{resultados}

\section{Conclusión}\label{conclusion}

Elaborar conclusiones basadas en los resultados obtenidos, destacando
los campos con mayor interés.

A raíz de los resultados obtenidos parece claro que el ámbito de la
conservación se va

\newpage
\singlespacing 
\bibliography{/home/jesus/Documents/CURSOS/drones/RTF/project/master.bib}

\end{document}
