
\documentclass[onecolumn]{extarticle}
%\documentclass[a4paper]{article}
\usepackage{layout}
\usepackage[left=1in,top=1in,right=1in,bottom=1in]{geometry}
\setlength{\headsep}{15pt}
\usepackage{booktabs}
\usepackage{rotating}
\usepackage{graphicx}
\usepackage{setspace}
\usepackage{geometry}
\usepackage{pdflscape}
\onehalfspacing
\usepackage{helvet}
\renewcommand{\familydefault}{\sfdefault}
\usepackage{caption}
\newcommand*{\authorfont}{\fontfamily{phv}\selectfont}
\usepackage[]{mathpazo}


\usepackage[T1]{fontenc}
\usepackage[utf8]{inputenc}
\usepackage[nottoc,notlof,notlot]{tocbibind} 
\renewcommand\refname{Literatura citada} % change original References title

\renewcommand{\contentsname}{Tabla de contenidos  }

\usepackage[tablename=Tabla]{caption}


\usepackage{abstract}
\renewcommand{\abstractname}{}    % clear the title
\renewcommand{\absnamepos}{empty} % originally center

\renewenvironment{abstract}
 {{%
    \setlength{\leftmargin}{0mm}
    \setlength{\rightmargin}{\leftmargin}%
  }%
  \relax}
 {\endlist}

\makeatletter
\def\@maketitle{%
  \newpage
%  \null
%  \vskip 2em%
%  \begin{center}%
  \let \footnote \thanks
    {\fontsize{18}{20}\selectfont\raggedright  \setlength{\parindent}{0pt} \@title \par}%
}
%\fi
\makeatother




\setcounter{secnumdepth}{0}


% notice vspace after title
\title{Contribución de los RPAS en investigación y conservación en espacios
protegidos: presente y futuro\vspace{0.25in}  }



\author{\Large Autor: Jesús Jiménez López\vspace{0.05in} \newline\normalsize\emph{Trabajo Fin de Curso de Experto Universitario en Vehículos Aéreos no
Tripulados y sus Aplicaciones Civiles. Edición 2016 - 2017. Universidad
de Cádiz. 12/05/2017}   \and \Large Supervisora: Margarita Mulero-Pázmány\vspace{0.05in} \newline\normalsize\emph{}  }


\date{}

\usepackage{titlesec}


\titleformat*{\section}{\large\bfseries}
\titlespacing{\section}{0em}{\parskip}{-\parskip}
\titleformat*{\subsection}{\normalsize\bfseries}
\titleformat*{\subsubsection}{\normalsize\itshape}
\titleformat*{\paragraph}{\normalsize\itshape}
\titleformat*{\subparagraph}{\normalsize\itshape}



\usepackage[round]{natbib}
\bibliographystyle{plainnat}


\newtheorem{hypothesis}{Hypothesis}
\usepackage{setspace}

\makeatletter
\@ifpackageloaded{hyperref}{}{%
\ifxetex
  \usepackage[setpagesize=false, % page size defined by xetex
              unicode=false, % unicode breaks when used with xetex
              xetex]{hyperref}
\else
  \usepackage[unicode=true]{hyperref}
\fi
}
\@ifpackageloaded{color}{
    \PassOptionsToPackage{usenames,dvipsnames}{color}
}{%
    \usepackage[usenames,dvipsnames]{color}
}
\makeatother
\hypersetup{breaklinks=true,
            bookmarks=true,
            pdfauthor={Autor: Jesús Jiménez López (Trabajo Fin de Curso de Experto Universitario en Vehículos Aéreos no
Tripulados y sus Aplicaciones Civiles. Edición 2016 - 2017. Universidad
de Cádiz. 12/05/2017) and Supervisora: Margarita Mulero-Pázmány ()},
             pdfkeywords = {RPAs, UAVs, drones, espacios naturales, conservación, biodiversidad,
investigación, innovación},  
            pdftitle={Contribución de los RPAS en investigación y conservación en espacios
protegidos: presente y futuro},
            colorlinks=true,
            citecolor=blue,
            urlcolor=blue,
            linkcolor=magenta,
            pdfborder={0 0 0}}

\urlstyle{same}  % don't use monospace font for urls



\begin{document}
	
% \pagenumbering{arabic}% resets `page` counter to 1 
%
% \maketitle

{% \usefont{T1}{pnc}{m}{n}

\setlength{\parindent}{0pt}
\thispagestyle{plain}

{\fontsize{16}{20}\selectfont\raggedright 
\maketitle  % title \par  

}

 
{
   \vskip 13.5pt\relax \normalsize\fontsize{8}{10} 
 \textbf{\authorfont Autor: Jesús Jiménez López} \hskip 10pt \vskip 5pt  \bf\emph{\small Trabajo Fin de Curso de Experto Universitario en Vehículos Aéreos no
Tripulados y sus Aplicaciones Civiles. Edición 2016 - 2017. Universidad
de Cádiz. 12/05/2017} \vskip 5pt  \par  \textbf{\authorfont Supervisora: Margarita Mulero-Pázmány} \hskip 10pt \vskip 5pt  \bf\emph{\small } \vskip 5pt  

}

}







\begin{abstract}

    \hbox{\vrule height .2pt width 38.9pc}

    \vskip 6pt % \small 

\noindent En este estudio se describe el estado actual y las tendencias en el uso
de los RPAS en proyectos científicos con fines de conservación en
espacios naturales protegidos, mediante la recopilación y revisión de
material bibliográfico en forma de artículos científicos, revistas,
proyectos de conservación y otras fuentes de información relevantes.


\vskip 8pt \noindent \textbf{Palabras claves}: \emph{RPAs, UAVs, drones, espacios naturales, conservación, biodiversidad,
investigación, innovación} \par

    \hbox{\vrule height .2pt width 38.9pc}



\end{abstract}


{
\fontsize{8}{10} 
\hypersetup{linkcolor=black}
\setcounter{tocdepth}{2}
\tableofcontents
}


\vskip 6.5pt

\noindent \doublespacing \section{Introducción}\label{introduccion}

Las aplicaciones civiles de las aeronaves pilotadas remotamente (siglas
en inglés RPAS, también conocidos como sistemas aéreos no tripulados,
UAS, drones) han sido planteadas en un número cada vez mayor de
artículos científicos. Durante los últimos años se ha producido un
incremento significativo de las líneas de investigación sobre vida
silvestre en espacios protegidos que hacen uso de RPAS
\citep{Linchant2015, Christie2016}. En la mayoría de los casos son
estudios pilotos que evalúan la capacidad de estos sistemas frente a
instrumentos tradicionales de apoyo a la conservación, delimitando sus
ventajas e inconvenientes, estableciendo pautas y recomendaciones de uso
y abriendo nuevas perspetivas de aplicación.

A inicios de 1980 comenzaron a realizarse los primeros experimentos con
RPAS en temas ambientales con el objetivo de adquirir fotografías aéreas
y demostrar su utilidad en aplicaciones forestales, la gestión de
recursos pesqueros o el acoplamiento de sensores para estudios
atmosféricos \citep{Tomlins1983}. Para finales del siglo XX aparecen los
primeros estudios de vegetación en especies amenazadas
\citep{quilter1997vegetation}, mientras que con la llegada del nuevo
milenio el número de publicaciones comienza a aumentar
significativamente \citep{Hardin2013}. En la actualidad existen algunas
iniciativas que tratan de recoger el estado actual de los RPAS en las
áreas de la ecología y conservación. Recientemente, la revista
\emph{Remote Sensing in Ecology and Conservation} hizo una llamada a la
comunidad científica para el envío de propuestas, con objeto de
sensibilizar a estudiantes y profesionales y demostrar el uso
responsable de RPAS. Es de esperar que del resultado de este llamamiento
se produzca un aumento significativo de la literatura científica en este
ámbito. Por otro lado, es remarcable la mayor presencia de portales en
internet que centran su actividad en torno a las aplicaciones con RPAS.
En el campo de la investigación aplicada a la conservación, el portal
web \url{http://conservationdrones.org/} es un referente mundial, cuyos
contenidos ilustran proyectos recientes de carácter pionero, por lo que
no siempre aparecen reflejados en la literatura científica. La
popularidad de los RPAS ha traspasado el ámbito científico-técnico,
dando lugar a la aparición de comunidades de usuarios con amplia
presencia en internet. Uno de los portales más activo es
\url{http://diydrones.com/}, que reune a aficionados afines a la
filosofía ``hazlo tú mismo'' (\emph{do it yourself}) que fomenta el uso
de plataformas abiertas frente a los tradicionales sistemas cerrados
ofertados por la industria tradicional del sector. Esto ha dado como
resultado la reducción de los costes de estos equipos y junto con el
desarrollo de software libre especializado, han permitido democratizar
la tecnología, acercándola a un mayor número de usuarios y
organizaciones. La comunidad científica se ha visto probablemente
beneficiado de esta tendencia general, puesto que la flexibilidad en el
ensamblaje de RPAS ofrece en principio un mayor grado de
personalización, permitiendo incluir diferentes sensores y sistemas de
control según las necesidades concretas de cada proyecto y dentro del
propio grupo de investigación \citep{Koh2012}. En el terreno comercial,
cada vez son más las empresas que ofrecen RPAS de grandes prestaciones y
cualificados para desarrollar aplicaciones profesionales, estimulando la
competencia en el sector.

Las limitaciones desde el punto de vista financiero y tecnológico de la
teledetección, por la cual se obtienen imágenes de la superficie
terrestre a partir de sensores instalados en plataformas aéreas o
espaciales, son descritas por diversos autores
\citep{Koh2012, Rodriguez2012}. Si bien es posible adquirir imágenes
satelitales a bajo coste o prácticamente nulo (LandSat, MODIS, Sentinel,
etc.), la mayor parte de estas plataformas operan a escala global o
regional. La limitada resolución espacial y temporal, junto con los
problemas de presencia de nubes especialmente acusados en zonas
tropicales, reduce la viabilidad de la teledetección en la recolección
de datos a escala suficientemente detallada para hacer frente a los
requerimientos de estudios ecológicos a nivel de especies, hábitats o
poblaciones \citep{Wulder2004}. Además, el gran tamaño de estas áreas
protegidas reducen en muchos casos la efectividad y aumenta
significativamente los costes de los trabajos de campo, mientras que
aumenta los riesgos en zonas especialmente inaccesibles. Como
consecuencia, los RPAS se han posicionado como un complemento adecuado
para las actividades de conservación \citep{Zahawi2015}, a pesar de la
reducida autonomía de vuelo que reduce la capacidad de estos sistemas
para sobrevolar regiones extensas. En países en vías de desarrollo,
especialmente sensibles en cuanto a dotaciones presupuestarias y
capacidades técnicas, se están desarrollando con éxito programas de
monitoreo y vigilancia a partir del uso de RPAS, eliminando alguno de
los inconvenientes descritos con anterioridad. Por ejemplo, mediante la
captura de imágenes aéreas en el delta del Volta, Ghana, un equipo de
científicos mide los efectos del cambio climático en zonas costera y
evalua la efectividad de las medidas de prevención y restauración frente
a los procesos erosivos \citep{Georg2016}. Los vehículos aéreos
tripulados ofrecen en principio una mejor alternativa en la captura de
imágenes de la superficie terrestre, sin embargo su uso no está
justificado en estudios a escala local, debido a costes operacionales
excesivamente altos. Por otro lado, el riesgo de sufrir accidentes
aéreos es mayor, situándose como primera causa de mortandad en
especialistas en vida silvestre en los Estados Unidos \citep{Sasse2003}.

Con objeto de reducir el impacto de los drones en estudios de fauna,
algunos experimentos analizan la respuesta de aves frente a RPAS
\citep{Vas2015}. Otros ensayos se centran en mamíferos y miden el estreś
fisiológico y posibles cambios en el comportamiento frente a vuelos
realizados con RPAS \citep{Ditmer2015}. Fruto de los resultados
obtenidos, se están comenzando a documentar manuales de buenas prácticas
y recomendaciones con objeto de reducir el impacto negativo en el
bienestar y evitar perturbaciones en los patrones de comportamiento de
las especies.

Finalmente, algunos autores señalan la necesidad de mejorar el marco
regulatorio respecto al uso civil de los RPAS \citep{Nugraha2016}. En
los Estados Unidos y en la mayoría de los países de Europa consultados,
se han adoptado leyes provisionales que en cierta medida equiparan el
manejo de los RPAS con el de aeronaves tradicionales. Este tipo de
restricciones podría limitar las posibilidades de uso de los RPAS en el
ámbito de la conservación, por lo que se hace patente la necesidad
urgente de armonizar la legislación en relación a este tipo de
actividades. En términos generales, la situación en América Latina es
desigual, con algunos países que siguen sin desarrollar leyes
específicas para hacer frente al auge de los RPAS tanto en el sector
civil como militar \citep{Nacion2013}. África es uno de los continentes
donde el impacto de los drones en conservación ha tenido mayor
repercusión. Sin embargo, según la opinión de algunos conservacionistas,
su uso no ha estado exento de problemas, dando como resultado gobiernos
que han prohibido total o parcialmente la operación con drones,
argumentando problemas de seguridad nacional en detrimento de la
protección de los espacios naturales protegidos \citep{Andrews2014}. La
incertidumbre de los usuarios ha promovido el desarrollo de asociaciones
con objeto de asesorar sobre los aspectos legales a tener en cuenta
durante la operación. En España, la Asociación Española de Drones y
Afines \url{https://www.aedron.com} promueve un uso consciente y
responsable de los RPAS y organiza seminarios para informar a los socios
sobre temas de interés. En su web se puede consultar el borrador de la
nueva normativa que regula la utilización civil de las aeronaves
pilotadas por control remoto en España \citep{Aedron2017}. A nivel
global han surgido otras iniciativas, siendo la Asociación Internacional
para Sistemas de Vehículos No Tripulados (AUVSI)
\url{http://www.auvsi.org} la organización sin fines de lucro más grande
del mundo dedicada al avance de la comunidad de usuarios de sistemas
aéreos no tripulados.

\subsection{Los espacios naturales
protegidos}\label{los-espacios-naturales-protegidos}

Los espacios naturales protegidos son aquellos en los que la
intervención del hombre no ha llegado a alterar de forma significativa
la presencia y funcionamiento de los elementos bióticos y abióticos que
lo integran \citep{Bravo2008}. Cumplen con finalidades de protección y
conservación del medio biofísico y cultural, donde se promueven
iniciativas en el ámbito científico, educativo y de restauración,
actividades recreativas y turísticas compatibles con el medio natural y
acciones de índole socioeconómica enmarcadas en el desarrollo sostenible
del territorio. Están amparados bajo alguna figura nacional o
internacional de protección y regulados a través de los planes de manejo
específicos. A pesar de que el número de áreas protegidas ha
experimentado un aumento considerable a nivel mundial, con un 15.4 \% de
la superficie terrestre y un 8.4\% de las áreas marinas bajo alguna
figura de protección \citep{juffe2014protected} , hay autores que
resaltan la necesidad de mejorar las herramientas de gestión de áreas
protegidas que aseguren la efectividad de la conservación de la
biodiversidad \citep{Chape2005}. Por otra parte algunas áreas protegidas
sufren procesos de degradación, continúan disminuyendo en tamaño o han
dejado de existir \citep{Mascia2011}, o bien han sido declaradas bajo
criterios oportunistas que no reflejan necesariamente el valor ecológico
de los ecosistemas a preservar \citep{Knight2007}. En un reciente
informe realizado por la Sociedad Zoológica de Londres
\citep{Living2016} , se calcula que el tamaño de las poblaciones de vida
silvestre ha disminuido en un 52 \% en el periodo de 1970 hasta 2012.
Todos los indicios apuntan al hombre como principal desencadenante de la
ya denominada sexta extinción masiva en nuestro planeta
\citep{Barnosky2011}, hasta el punto de que algunos investigadores
comienzan a hablar del antropoceno, como inicio de una nueva época del
Cuaternario. La fragmentación del hábitat, el aumento de la
contaminación, especialmente grave en ecosistemas de agua dulce, la
sobreexplotación de los recursos, las consecuencias a nivel global del
cambio climático y el impacto de las especies invasoras sobre
poblaciones autóctonas han sido identificados como las principales
amenazas para la diversidad biológica \citep{Conabio2017}.

El Grupo sobre Observaciones de la Tierra (GEOBON) ha identificado un
conjunto de variables esenciales para la biodiversidad
\citep{Pereira2013} con objeto de desarrollar un abanico de indicadores
que permitan conocer el estado global de nuestros ecosistemas y ayuden a
la mejor toma de decisiones en materia de biodiversidad mediante la
integración de técnicas de observación remota y observaciones in-situ
como piezas clave para el levantamiento de información ambiental
\citep{Forum2008}. Por otra parte, el Convenio sobre la Diversidad
Biológica desarrollado como parte del Programa de las Naciones Unidas
para el Medio Ambiente (PNUMA) estableció en Nagoya (Japón) un plan
estratégico para el periodo 2011-2020 que incluye las denominadas
\emph{metas de Aichi} para la diversidad biológica. Dentro de los
objetivos planteados destaca el aumento de los sistemas de áreas
protegidas de especial importancia para la biodiversidad y los servicios
ecosistémicos (Meta 11) y se establece una serie de criterios de
gobernanza, equidad, gestión, representatividad y conectividad ecológica
para la inclusión de áreas prioritarias para la conservación. Para hacer
frente a la crisis ambiental actual, es necesario desarrollar soluciones
que mejoren el estado de la biodiversidad.

En este contexto, el presente documento realiza una revisión de los RPAS
en estudios de conservación y gestión de áreas protegidas, mencionando
las barreras técnicas y legales que limitan su efectividad.

\section{Métodos}\label{metodos}

Para alcanzar los objetivos propuestos se procedió a la revisión
bibliográfica de artículos, literatura gris, tesis de postgrado, sitios
web y revistas especializadas, siguiendo una línea similar a otros
estudios realizados con anterioridad \citep{Linchant2015, Christie2016}
Mediante artículos seleccionados para el curso de Experto Universitario
en Vehículos Aéreos no Tripulados y sus Aplicaciones Civiles organizado
por la Universidad de Cádiz en su edición de 2016-2017, junto con
herramientas como Google Schoolar, ResearchGate y Mendeley Desktop, se
obtuvo la mayor parte de la bibliografía seleccionada, mientras que el
uso de los motores de búsqueda por internet incluyeron el resto de
materiales mencionados. Los principales criterios de búsqueda por
palabras claves incluyeron los vehículos aéreos no tripulados en sus
diversas acepciones y acrónimos (RPAS, UAV, drones, etc.), junto con una
variedad de términos que hacen referencia a áreas naturales protegidas,
fundamentalmente en inglés. Dicha actividad tuvo lugar desde el 4 de
Abril de 2017 hasta el 12 de Mayo del mismo año.

La información obtenida se categorizó según el propósito de aplicación
de los RPAS en relación directa o indirecta con la conservación en
espacios protegidos. La mayoría de las fuentes analizadas se centran en
proyectos de conservación a escala local y estudios de viabilidad de los
RPAS en la caracterización de poblaciones y comunidades de vida
silvestre, especialmente en estudios de distribución y abundancia. La
literatura comienza a ser igualmente prolífica en actividades de
monitoreo y mapeo en ecosistemas terrestres y acuáticos, nicho
actualmente ocupado por las plataformas aéreas y espaciales de
teledetección ambiental. A pesar de la dificultad de encontrar artículos
dedicados al uso de RPAS en el control y vigilancia de áreas protegidas,
es uno de los temas que mayor debate social genera y no es extraño
encontrar iniciativas gubernamentales o promovidas por organizaciones
ambientales en la lucha contra la caza y pesca furtiva. Adicionalmente
se revisan algunos aspectos de calado social recogidos en los materiales
seleccionados y que son motivo de controversia, con especial referencia
a la privacidad de las personas y el bienestar de las especies
estudiadas, las implicaciones éticas y legales y la repercusión en la
efectividad de los RPAS en la conservación a largo plazo. En cualquier
caso, dado el carácter multidisciplinar y multipropósito de estos
estudios existe cierto solape entre los objetivos marcados dentro de
cada proyecto, por lo que se ha tenido en cuenta aquellos objetivos que
mayor peso tienen en el contexto de la investigación.

La información seleccionada se presenta en formato tabular,
identificando los países implicados, el propósito principal de cada
estudio, las técnicas y sistemas empleados, haciendo referencia
explicita al tipo de aplicación y plataformas de vuelo, tanto de ala
fija como de pala rotatoria. Finalmente se discuten los posibles
escenarios que presentan los RPAS como herramientas fundamentales para
contribuir a la consecución de los planes de conservación en espacios
protegidos, destacando algunas tendencias y oportunidades que
aparentemente aún no han sido convenientemente explotadas.

\begin{landscape}
\begin{table}

\caption{APLICACIONES RPAS EN ESPACIOS NATURALES PROTEGIDOS}
\tiny
\begin{tabular}{p{3cm}p{1.2cm}p{3cm}p{1cm}p{2cm}p{2cm}p{1cm}p{2cm}p{2cm}p{1cm}p{2cm}}

\cmidrule(r){1-11}

Estudio & Área protegida & Objetivo/s & País & Lugar & Especie & Tipo RPAS & Modelo RPAS & Sensor & Georref. & Costo \\ \cmidrule(r){1-11}

\multicolumn{11}{c}{} \\
\multicolumn{11}{c}{ {\bf ESTUDIOS DE FAUNA Y VIDA SILVESTRE}  } \\
\multicolumn{11}{c}{} \\

\cite{PazmanyMulero2015}  & Si & Estudio comparativo modelos distribución de especies & España & Parque Nacional de Doñana & Bos taurus  & Ala fija & Easy Fly plane, Ikarus autopilot, Eagletree GPS logger & Panasonic Lumix LX-3 11MP & Si & 
5700 euros \\ 

\citealt{Hodgson2013} & Si & Detección e identificación de dugongs.  Comprobar actitud  RPAS en diferentes condiciones ambientales. Determinar altura y resolución ideal  & Australia & Shark Bay Marine Park & Dugong & Ala fija &  ScanEagle & Nikon D90 12 megapixel digital SLR camera  & Si & ?  \\ 

\cite{Longmore2017} & No & Desarrollo de software detección especies infrarrojo térmico & Inglaterra & ? & Fauna & Multicóptero & 750mm carbon-folding Y6 multi-rotor APM 2 autopilot 3Drobotics & FLIR, Tau 2 LWIR Thermal Imaging Camera Core  & ? & ?  \\ 

\cite{Wilson2017}  & No & Monitoreo bioacústico con RPAS & USA & State Game Lands & Aves  & Multicóptero & DJI Phantom 2 & ZOOM H1 Handy Recorder  & Si & ? \\ 

\cite{Bayram2016}  &  No & Detección de collares VHF & ? & ? & Oso (Ursus)  & Multicóptero & DJI F550 hexarotor, Pixhawk autopilot & Telonics MOD-500 VHF, Uniden handheld scanner  & Si & ? \\ 

\cite{Christie2016}  &  Si  & Estimación abundancia & USA &  Aleutian Islands & León marino de Steller (Eumetopias jubatus) & Multicóptero & APH- 22 hexacopter & ?  & Si & \$ 25.000 , \$ 3000 alquiler barco, or \$ 1700 por sitio \\ 

\cite{Christie2016}  &  Si & Estimación abundancia & USA &  Monte Vista National Wildlife Refuge & Grus canadensis (sandhill cranes)  & Ala fija & Raven RQ- 11A & ?  & Si & \$ 400 \\ 

\multicolumn{11}{c}{} \\
\multicolumn{11}{c}{{\bf  MONITOREO DE ECOSISTEMAS TERRESTRES Y ACUÁTICOS  }} \\
\multicolumn{11}{c}{} \\

\cite{Perroy2017}  & No & Monitoreo de plantas invasoras & USA & Pahoa, Hawai & Miconia calvescens & Multicóptero & DJ Inspire-1 & DJI FC350 camera  & Si & ?  \\ 

\cite{Szantoi2017}  & Si & Mapeo de hábitat & Indonesia & Gunung Leuser National Park & Orangután (Pongo abelii)  & Ala fija & Skywalker & Canon S100  & Si & \$ 4000 \\ 
  
\cite{Ivosevic2015}  & Si & Monitoreo hábitats zonas restringidas; Modelos; Comprobar actitud  RPAS en diferentes condiciones ambientales & South Korea & Chiaksan National Park;Taeanhaean National Park &  ? & Multicóptero & DJI Phantom 2 Vision+  & full HD videos 1080p/30fps and 720p/60fps, cámara 14 megapixels 4384x3288 & Si & ? \\ 
  
\cite{Lisein2015}  & No & Discriminación de especies de  hoja caduca, inventario forestal & Bélgica & Grand-Leez & English oak, birches (Betula pendula Roth. and Betula pubescens Ehrh.), sycamore maple (Acer pseudoplatanus L.), common ash (Fraxinus excelsior L.) and poplars (Populus spp.) & Ala fija & Gatewing X100  & Ricoh GR2 GR3 GR4 10 megapixels CCD  & Si & ?  \\ 
  
\cite{Puttock2015}  & Si & Caracterización ecosistemas afectados por la actividad del castor & UK & Devon Beaver Project site & Eurasian beaver (Castor fiber) & Multicóptero & 3D Robotics Y6 & Canon ELPH 520 HS  & Si & ?  \\ 
  
\cite{Zahawi2015}  & No & Caracterización estructura bosques tropicales para acciones de restauración & Costa Rica & Devon Beaver Project site & Varias especies & Multicóptero & 3D Robotics Y6 & Canon S100  & Si & \$ 1500 \\ 
  
\cite{Bustamante2015}  & Si &  Monitoreo de bosques & Brasil & Riverine Forests (Permanent Protected Areas), Rio de Janeiro, Barrãcao do Mendes, Santa Cruz and São Lorenço & Bosques de rivera & Multicóptero & DJI Phantom Vision 2S   & RGB digital camera with 14 mega pixels & Si & \$ 9700  \\ 
  
\cite{Gini2012}  & Si & Modelamiento 3D, clasificación de especies arbóreas & Italy & Parco Adda Nord &  Varias especies &  Multicóptero  &  Microdrones TM MD4-200 & RGB CCD 12 megapixels Pentax Optio A40, modified NIR Sigma DP1 with a Foveon X3 sensor  & Si & ?  \\

\cite{Miyamoto2004}& Si & Clasificación de especies en humedales & Japón & Humedales de Kushiro &  Varias especies & Globo helio  & ? & NIKON F-801, NIKKOR 28 mm f/2.8  & Si & Helio \$ 600, globo \$ 1000  \\ 

\multicolumn{11}{c}{} \\
\multicolumn{11}{c}{{\bf EVALUACIÓN DE INFRAESTRUCTURAS Y RIESGO, VIGILANCIA, ECOTURISMO, IMPACTO EN LA FAUNA }} \\
\multicolumn{11}{c}{} \\

\cite{Lobermeier2015} & No  & Mitigar el riesgo de colisión mediante la instalación de marcadores en líneas electríca & USA & ?  & Aves  & Multicóptero  & Mikrokopter Hexa XL  & KX 171 Microcam & ? & ? \\ 

\cite{Mulero-Pazmany2014a} & Si  & Evaluación riesgo riesgo eléctrico de nidos en postes de alta tensión & España & Parque Nacional de Doñana &  Aves  & Ala fija  & Easy fly St-330 & GoPro Hero 2 11 MP, Panasonic LX3 11MP & Si & 7800 euros  \\ 

\cite{Mulero-Pazmany2014}  & Si  & Vigilancia en áreas protegidas & Africa & KwaZulu-Nata & black rhinoceros (Diceros bicornis), white rhinoceros (Ceratotherium simum)  & Ala fija  & Easy Fly St-330 & Panasonic Lumix LX-3 11 MP, GoPro Hero2, Thermoteknix Micro CAM microbolometer & Si & 13750 euros  \\ 

\cite{Hansen2016} & Si  & Monitoreo actividad visitantes  & Suecia & Kosterhavet National Park &  Humanos  & ?  & ? & ?  & ? & ?  \\ 

\cite{King2014} & Si  & Aplicaciones RPAS en actividades ecoturismo & Suecia & Kosterhavet National Park &  Humanos  & ?  & ? & ?  & ? & ?  \\ 
  
\cite{Vas2015} & Si  & Impacto RPAS  especies aves lacustres  & Francia & e Zoo du Lunaret, Cros Martin Natural Area &  Anas platyrhyncho, Phoenicopterus roseus, Tringa nebularia  & Multicóptero &  Phantom & Hero3 GoPro & Si  & ?  \\ 

\cite{Ditmer2015} & Si  & Impacto  RPAS oso negro americano   & USA & Kosterhavet National Park &  Oso negro americano (Ursus americanus) & Multicóptero & 3DR IRIS Pixhawk & GoPro HERO3+ & ?  & ?  \\ 

\end{tabular}
\end{table}
\end{landscape}

\section{Resultados}\label{resultados}

\subsection{Estudios de fauna y vida
silvestre}\label{estudios-de-fauna-y-vida-silvestre}

Uno de los temas centrales de la ecología es el desarrollo de modelos
estadísticos de distribución de especies que permiten inferir el hábitat
potencial o idóneo de los organismos a partir de la recolección de
información ambiental y datos de presencia procedentes de diversas
fuentes \citep{Mateo2011}. La radiotelemetría es uno de los métodos más
comunes para la recolección de datos de movimiento en individuos
marcados con geolocalizadores. Algunos estudios comparan el uso de RPAS
frente a estos sistemas \citep{Mulero-Pazmany2015} en animales de gran
tamaño y fácilmente identificables mediante imágenes aéreas de alta
resolución, obteniendo resultados similares en cuanto al rendimiento de
los modelos pero con diferencias notables en cuanto a costes derivados
de la compra de los equipos y gastos de logística, favoreciendo en este
caso a los RPAS. Las limitaciones financieras también afectan al tamaño
del muestreo con el uso de geolocalizadores, con el riesgo añadido de
marcar individuos bajo criterios no aleatorios, si bien se remarca la
ventaja de estos sistemas en cuanto a su capacidad para generar grandes
volumenes de datos en un periodo de tiempo mayor. En cuanto a la
exactitud posicional, los RPAS tienen un error máximo entre 1 y 3
metros, mientras que los errores del GPS pueden ser mayores a 20 metros.
En cualquier caso los autores remarcan que ambas metodologías tienen
potencial para complementarse a lo largo de todas las fases del estudio.
Otras técnicas innovadoras han sido recientemente ilustradas en
artículos cientificos que evaluan la viabilidad del uso combinado de
radiolocalizadores en RPAS en la búsqueda de individuos marcados con
radio collares VHF
\citep{Korner2010, Bayram2016, Cliff2015, Leonardo2013}.

En determinados casos, frente a las dificultades para detectar
directamente a la especie de interés, los estudios se enfocan en la
localización y caracterización de sus áreas de cría y nidificación
\citep{VanAndel2015}. En áreas protegidas de gran extensión se han
ensayado con éxito el conteo de grandes mamíferos terrestres , no
habiéndose registrado reacciones adversas en vuelos realizados a una
altura mínima de 100 metros \citep{Vermeulen2013}. La estimación de
poblaciones de mamíferos en ecosistemas marinos también ha sido
documentada con resultados positivos \citep{Hodgson2013}, mientras que
en aves se han usado para estudiar las dinámicas poblacionales en
colonias \citep{Sarda-Palomera2012}. La utilidad de estos sistemas
también queda manifiesta en la inspección y caracterización de nidos de
aves en zonas de difícil acceso \citep{Weissensteiner2015}, permitiendo
evaluar el estado en el que se encuentran de forma menos intrusiva.

Dada la masiva cantidad de información generada, no es de extrañar que
se hayan aplicado métodos desarrollados en el campo de la visión
computerizada, dirigidos al conteo automático de individuos capturados
en las escenas adquiridas por los sensores fotográficos
\citep{Lhoest2015, Abd-Elrahman2005a, VanGemert2015}. Esto conlleva una
reducción de los costes respecto al conteo manual de las escenas
adquiridas, con la ventaja adicional de no estar sujetos en mayor o
menor medida a la interpretación del especialista. En este sentido, los
métodos de observación directa desde vehículos aéreos tripulados también
representan desventajas con respecto a la toma de imágenes aéreas,
puesto que necesitan un mayor número de observadores que garantizen un
conteo exahustivo de las poblaciones para evitar errores en la
estimación. Fuera de la literatura científica, existen proyectos para el
monitoreo de la fauna tanto en ecosistemas marinos como terrestres, en
su mayoría respaldados por organizaciones no gubernamentales y centros
de investigación.. A partir de la información recopilada en la comunidad
online \url{https://conservationdrones.org} se han identificado varios
estudios relacionados con el registro de individuos en poblaciones de
mamíferos marinos, primates y macrofauna en general, situados en áreas
protegidas o frecuentemente visitadas por especies sujetas a alguna
figura de amenaza. Por ejemplo, un estudio realizado en la cuenca del
Amazonas en Brasil está experimentando el uso de drones para mejorar la
estimación de la densidad y abundancia de diferentes especies de
delfines, comparándolo con la observación directa realizada por
especialistas \citep{WichS2017}. Dentro de los objetivos de la
investigación se contempla la validación y armonización de ambas
metodologías y de forma indirecta, evaluar la viabilidad para su
aplicación regular en proyectos de monitoreo con similar propósito,
teniendo en cuenta el coste-beneficio de la ejecución.

\subsection{Evaluación de infraestructuras y
riesgo}\label{evaluacion-de-infraestructuras-y-riesgo}

Otros trabajos resaltan la utilidad de los RPAS en la evaluación del
riesgo de infraestructuras humanas y la puesta en marcha de medidas
preventivas frente a especies de aves que nidifican en postes de líneas
eléctricas de alta tensión, haciéndolas especialmente vulnerables a
colisiones y electrocutamiento. Para la ejecución de trabajos de
precisión donde la estabilidad, maniobrabilidad y el detalle en la
identificación es esencial \citep{Lobermeier2015} se recomienda el uso
de multicópteros. En evaluaciones de estructuras lineales de gran
extensión en las que el costo y tiempo de vuelo es determinante en
contraposición a la resolución espacial, los vehículos de ala fija
ofrecen mejores ventajas \citep{Mulero-Pazmany2014a, Zhang2016}.

Si bien estos estudios no están dirigidos exclusivamente a áreas
protegidas, podrían resultan de especial interés en zonas aledañas de
amortiguamiento, donde el desarrollo antrópico puede generar situaciones
de conflicto con la fauna circundante. Por ejemplo, se sabe que hay
ciertas especies de aves que nidifican en el suelo, especialmente en
zonas de cultivo de cereal. Como actividad previa a la cosecha,
realizada generalmente bajo procedimientos mecánicos, se podría realizar
un sobrevuelo para identificar posibles nidos, y en su caso, tomar las
medidas adecuadas para evitar su destrucción \citep{Mulero-Negro}.

\subsection{Monitoreo y mapeo de ecosistemas terrestres y
acuáticos}\label{monitoreo-y-mapeo-de-ecosistemas-terrestres-y-acuaticos}

Durante las últimas décadas el auge de los sensores remotos a bordo de
plataformas aéreas o espaciales ha desencadenado un aumento de las
aplicaciones para el estudio de los ecosistemas \citep{Wulder2004}. Los
datos obtenidos han permitido desarrollar mapas de cobertura vegetal y
suelos, caracterizar hábitats, mejorar la compresión de la estructura y
función de las masas forestales, desarrollar modelos digitales de
elevaciones o levantar cartas geomorfológicas de aplicación en el
modelamiento de distribución de especies. La aparición de los RPAS ha
propiciado el análisis cuantitativo de hábitats a un nivel de detalle
que no había sido posible anteriormente, bien por motivos económicos o
por limitaciones propias de la ingeniería. Este impulso ha sido
especialmente notable con el desarrollo paralelo de sensores
multiespectrales e hiperespectrales adaptados a aeronaves de pequeño
tamaño, cuyo precio se espera disminuya según las tendencias del sector
tecnológico. Dentro de las actividades del Servicio Geológico de los
Estados Unidos (USGS) se han realizado vuelos con objeto de clasificar
la cobertura vegetal en humedales \citep{USGS2014}. Otros estudios
monitorean la distribución de especies invasoras bajo diferentes
condiciones de vuelo y cobertura vegetal \citep{Perroy2017}, mientras
que la caracterización de masas forestales constituye un importante
apartado dado el número de artículos que abordan el problema desde
diferentes perspectivas. \citep{Gini2012} emplea un modelo de
cuadrocóptero y cámaras en RGB y NIR a baja altura en áreas de pequeña
extensión. Debido a la reducida fiabilidad y autonomía de la plataforma
y las dificultades para aumentar la capacidad de carga, la planificación
del vuelo se ve reducida a tres pasadas con un porcentaje del 80\% y
30\% de solape longitudinal y transversal respectivamente.
\citep{Lisein2015} realiza un análisis multitemporal de la respuesta
espectral frente a variaciones en la fenología en diferentes especies de
árboles de hoja caduca y concluye que la variación espectral
intraespecífica es de máximo interés para la optimización de los
algoritmos de clasificación y discriminación entre especies. En su
investigación, opera un modelo de RPAS de ala fija, utiliza diferentes
sensores en el rango visible e infrarrojo cercano y optimiza los
parámetros de vuelo con objeto de cubrir la máxima superficie con el
menor número de vuelos posible. \citep{Zahawi2015} aplica la metodología
\emph{Ecosynth}, un conjunto de herramientas para cartografiar y medir
la vegetación en 3D utilizando cámaras digitales y software de visión
artificial de código abierto, con objeto de evaluar la eficacia de las
acciones de restauración en bosques tropicales mediante RPAS, como
alternativa viable para las medidas de campo tradicionales y aplica
diferentes modelos predictivos de presencia de pájaros frugívoros a
partir de los datos de altura y estructura del dosel vegetal.

\subsection{Vigilancia y apoyo para el cumplimiento de las leyes en
áreas
protegidas}\label{vigilancia-y-apoyo-para-el-cumplimiento-de-las-leyes-en-areas-protegidas}

Los RPAS también tienen especial proyección en el control y vigilancia
de áreas protegidas. Así lo demuestran diferentes experiencias enfocadas
principalmente en el control de la caza y pesca furtiva. Este tipo de
estudios se caracteriza por dar una mayor importancia a la mejora de los
sistemas de visión en primera persona (FPV) con objeto de obtener una
panorámica en tiempo real de la zona monitoreada, el uso de RPAS de ala
fija cuya mayor autonomía frente a los multirrotores permite cubrir una
mayor extensión, la necesidad de utilizar cámaras térmicas en
condiciones de baja visibilidad, usualmente relacionadas con horas de
mayor actividad furtiva, junto con avances en los sistemas de visión
computerizada programados para detectar la presencia de humanos y
especies sometidas a la presión de comercio ilegal en áreas protegidas.
\citep{Mulero-Pazmany2014} se enfocan en el rinoceronte africano y
constatan las ventajas del video en tiempo real frente a la toma de
fotografías, que necesitan un mayor tiempo de postprocesamiento.
Adicionalmente recalcan la necesidad de mejorar la resolución de los
sensores térmicos para aumentar las posibilidades de detectar
actividades sospechosas en horas nocturnas. \citep{Duffy2014} analiza
las consecuencias de la militarización de las prácticas de conservación,
como tendencia cada vez mayor en áreas naturales protegidas de todo el
mundo e ilustra el uso de RPAS a través de varios ejemplos. Respecto a
zonas costeras, una búsqueda rápida por internet permite recoger
diversas iniciativas que tratan de optimizar las labores de control de
la pesca ilegal mediante RPAS. Sin embargo no hemos podido constatar
estudios científicos que avalen tales iniciativas, por lo que se abre
una vía interesante para su investigación. Por ilustrar algunos de los
numerosos ejemplos, en Belice se realizó un estudio pionero para el
monitoreo de pesquerías mediante un modelo de ala fija Skywalker. El
Gobierno de Canarias está considerando el uso de RPAS en labores de
control e inspección en zonas de difícil acceso para hacer frente al
furtivismo \citep{Canarias2017}. Finalmente \url{http://soarocean.org/}
es una iniciativa de \emph{National Geographic} y \emph{Lindblad
Expedition} para el uso de drones de bajo coste en la protección de los
océanos y es un buen punto de partida para buscar aplicaciones pioneras
en este ámbito.

\subsection{Ecoturismo}\label{ecoturismo}

El alto grado de diversificación que ofrece la aplicación de los RPAS en
la industria ecoturística queda resumido en un artículo reciente, en el
que se exponen posibles actividades recreativas, oportunidades de
negocio, operaciones de búsqueda y rescate, mapeo y fórmulas para la
concesión de operaciones con RPAS en áreas designadas para tal fin
\citep{King2014}. Dentro de la aún escasa literatura, \citep{Hansen2016}
valora la eficacia de los RPAS en el monitoreo de visitantes en áreas
marinas y costeras, en combinación con otras soluciones innovadoras.
Según el autor los RPAS permitirían teóricamente operar bajo diferentes
condiciones ambientales, mejorando el nivel de detalle y ofreciendo una
cobertura continua en el flujo y comportamiento de los visitantes , en
contraposición a otras técnicas de uso habitual como la observación
manual o la instalación de redes de cámaras de vigilancia.

\subsection{Impacto de los RPAS en la fauna
silvestre}\label{impacto-de-los-rpas-en-la-fauna-silvestre}

La priorización del bienestar del animal debe ser incluida en las
aplicaciones de RPAS, estableciendo unos principios éticos que
complementen los estándares vigentes en materia de investigación y
conservación. \citep{Vas2015} obtienen resultados prometedores de
relevancia en estudios ornitológicos, valorando el impacto del color, la
velocidad y el ángulo de vuelo en las respuestas de comportamiento de
aves lacustres frente a la aproximación de multirrotores, siendo este
último factor el principal desencadenante de cambios en los patrones de
comportamiento, especialmente en aproximaciones desde la vertical, a un
ángulo de 90º. Finalmente concluyen con una serie de recomendaciones
básicas y consideran recomendable extender los ensayos a una amplia gama
de RPAS y especies. \citep{Ditmer2015} mide el estrés fisiológico en el
oso negro americano mediante el registro electrónico de la actividad
cardíaca en presencia de RPAS. Si bien no registran cambios en los
patrones de comportamiento, el aumento de los latidos por minuto (bmp)
es significativo en la mayoría de los casos observados. Ante la falta de
experiencias que aborden de manera explícita el fenómeno,
\citep{Hodgson2016a} concluyen con una serie de recomendaciones
generales como base para un código de buenas prácticas, destacando la
adopción del principio de precaución y respeto a las normas de aviación
civil, el entrenamiento específico de los operadores, la selección
apropiada de los equipos, el cese de las operaciones en caso de generar
perturbaciones evidentes en las poblaciones estudiadas y el reporte de
las observaciones en publicaciones científicas, que permita compartir el
conocimiento con vistas a una mejora progresiva en los protocolos de
operaciones con RPAS.

\section{Discusión}\label{discusion}

\subsubsection{Estudios de fauna y vida
silvestre}\label{estudios-de-fauna-y-vida-silvestre-1}

A razón de la bibliografía seleccionada, el uso de RPAS de ala fija y
multirrotores tienen su ámbito especifico de actuación. En el primer
caso, la mayor parte de los estudios se centran en el conteo de
poblaciones, con resultados prometedores en la macrofauna. Todavía es
pronto para generalizar su uso en especies de menor tamaño y en zonas de
alta cobertura vegetal, si bien el desarrollo de la tecnología LIDAR y
sensores de amplio espectro podría ayudar a superar las barreras
técnicas. Adicionalmente, es necesario mejorar el conocimiento en cuanto
a la planificación de los muestreos realizados con RPAS, para evitar
errores en la estimación. Los multirrotores podrían cubrir algunas de
las limitaciones anteriormente citadas, pero aún parece no existir
estudios que combinen ambos sistemas. Por otro lado, el uso de RPAS
puede aumentar la complejidad de la investigación, con equipos de
trabajo cada vez más técnicos y necesidades computacionales que no estén
al alcance de cualquier institución. En cualquier caso, los RPAS podrían
convertirse en una herramienta esencial para especialistas en ecología y
su uso podría estar justificado mientras no se registren avances en
otras técnicas tradicionales de apoyo a la investigación.

\subsubsection{Evaluación de infraestructuras y
riesgo}\label{evaluacion-de-infraestructuras-y-riesgo-1}

Los RPAS han demostrado su capacidad para la inspección técnica de
recintos industriales. El alto coste que supone la evaluación de riesgos
para la fauna en zonas de alta incidencia de muertes podría persuadir a
las autoridades ambientales para fomentar su uso con fines preventivos.
Los RPAs también podrían evitar accidentes mediante la aplicación de
medidas disuasorias que eviten la colisión de aves en parques eólicos.
Otros usos incluyen la revisión del equipamiento en espacios naturales
protegidos, mediante la programación de vuelos periódicos en
instalaciones de uso público. Fuera del ámbito de este estudio aunque
estrechamente relacionados con la gestión de espacios naturales, los
RPAS se posicionan como herramientas fundamentales en la prevención y
evaluación de incendios en zonas forestales.

\subsubsection{Monitoreo y mapeo de ecosistemas terrestres y
acuáticos}\label{monitoreo-y-mapeo-de-ecosistemas-terrestres-y-acuaticos-1}

Los estudios analizados demuestran un gran potencial de los RPAS para
integrarse como parte esencial en estudios relacionados con la
teledetección ambiental. El mayor poder de cálculo de los procesadores
modernos, así como los recientes avances en fotogrametría han permitido
ir superando las limitaciones tecnológicas para el manejo de grandes
volumenes de datos Probablemente sea necesario continuar investigando en
el desarrollo de algoritmos cada vez más exactos que mejoren la
identificación de las características de interés.

\subsubsection{Vigilancia y apoyo para el cumplimiento de las leyes en
áreas
protegidas}\label{vigilancia-y-apoyo-para-el-cumplimiento-de-las-leyes-en-areas-protegidas-1}

El futuro de los RPAS en la lucha contra la caza furtiva y la pesca
ilegal en áreas protegidas se enfrentan a limitaciones técnicas que
probablemente puedan ser solventadas en los próximos años. Los estudios
revisados mencionan la necesidad de desarrollar de sistemas de visión
nocturna más eficientes y capaces de discernir actividades de tipo
criminal. La autonomía de los RPAS es especialmente crítica en parques
naturales de gran extensión, junto con la presencia de condiciones
atmosféricas adversas que afectan a la capacidad de vuelo. En algunos
paises se prohibe volar más allá del alcance visual del operador,
dificultando la posibilidad de realizar inspecciones en tiempo real. Es
de esperar que este modo de operar pueda estar sujeto a una regulación
especial que favorezca a este tipo de actividades. La mayor parte de los
conflictos son de tipo legal y su uso es cuestionable en cuanto a la
manera en que podrían infringir la privacidad y las libertades civiles.
Todavía es pronto para asegurar si el uso de RPAS en la lucha contra el
furtivismo puede persuadir a los infractores, que en muchos casos se
enfrentan a situaciones de máxima necesidad. Probablemente el éxito de
este tipo de iniciativas requiera de un consenso mayor entre las partes
implicadas.

\subsubsection{Ecoturismo}\label{ecoturismo-1}

Una permisiva regularización del uso de RPAS en actividades
ecoturísticas en parques naturales podría conducir a situaciones
impredecibles. Por un lado la presencia constante del ruido de hélices y
motores, la sensación de invasión o falta de privacidad y el impacto
visual de los RPAS sobre el paisaje podrían afectar negativamente la
experiencia del turista o en su caso, pertubar significativamente los
ecosistemas. La sensibilización frente a el abuso de RPAS para la
grabación de la vida salvaje ha resultado en la prohibición de volar con
fines recreativos en parques naturales de Estados Unidos y otras partes
del mundo. Es necesario mencionar el riesgo real de accidentes que
podrían dan lugar, por ejemplo, a la contaminación de reservas de agua,
debido a la toxicidad de los materiales. El abandono o pérdida de RPAS
siniestrados también podría aumentar el riesgo incendios en zonas
sensibles, debido a la presencia de componentes inflamables. No parece
que se hayan publicado estudios pilotos ni encuestas de opinión que
respondan a las cuestiones planteadas y a las implicaciones éticas y
legales derivadas de su uso. Aún cuando las posibilidades de ocio son
amplias y reconocidas, sería conveniente ser cautelosos frente a la
demanda de la industria ecoturística para incorporar los RPAS en sus
actividades.

\subsubsection{Impacto de los RPAS en la fauna
silvestre}\label{impacto-de-los-rpas-en-la-fauna-silvestre-1}

La revisión de la literatura sugiere que aún quedan ciertos nichos que
necesitan de mayor atención por parte de la comunidad. Aún no han sido
sopesadas convenientemente las implicaciones éticas de los RPAS en
estudios de fauna. Por ejemplo, la mayor parte de los estudios solo
abordan marginálmente la presencia o ausencia de reacciones en especies
frente a la cercanía de RPAS. Consideramos que los experimentos
dirigidos a cuantificar cambios fisiológicos y de comportamiento son
insuficientes y que a pesar del surgimiento de algunas iniciativas y un
mayor grado de sensibilización, sería necesario mejorar nuestro
conocimiento actual con objeto de perfeccionar un conjunto de buenas
prácticas y recomendaciones dirigidas a especies concretas. Otros
factores claves incluyen la profesionalización de los operadores y la
inversión en modelos de RPAS optimizados para reducir el impacto sobre
la fauna y facilitar su observación, especialmente en cuanto a la
disminución del ruido de hélices y motores y el diseño de componentes no
contaminantes.

\section{Conclusiones}\label{conclusiones}

La consolidación de los RPAS como herramientas de gestión e
investigación en áreas naturales protegidas están estrechamente ligadas
al desarrollo tecnológico de los elementos asociados a la plataforma y
el establecimiento de medidas que regulen favorablemente su uso,
aumentando las oportunidades en el sector y estimulando la innovación en
áreas de interés para la conservación. Continuamente se producen mejoras
en el control de la navegación y autonomía de vuelo, mientras que
asistimos a la progresiva minituarización y diversificación de sensores
junto con avances en el campo de la inteligencia artificial. Esta
confluencia de factores en rápida expansión fomenta la aparición de
nuevos escenarios con repercusiones éticas y legales. La mayor parte de
los gobiernos han reaccionado estableciendo limitaciones que podrían
repercutir negativamente en la capacidad de los RPAS para integrarse en
el ámbito civil. Ante esta situación, resulta difícil prever las
acciones que cada país va a adoptar a partir de ahora en un intento por
armonizar las ventajas e inconvenientes de estos sistemas, por lo que es
probable que el futuro de los RPAS en áreas protegidas esté condicionado
fundamentalmente por factores políticos y sociales.

\newpage
\singlespacing 
\bibliography{/home/jesus/Documents/CURSOS/drones/RTF/project/master.bib,/home/jesus/Documents/CURSOS/drones/RTF/project/internet.bib}

\end{document}
