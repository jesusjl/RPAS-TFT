
\documentclass[11pt,]{article}
\usepackage[left=1in,top=1in,right=1in,bottom=1in]{geometry}
\usepackage{booktabs}
\usepackage{rotating}
\usepackage{graphicx}
\usepackage{caption}
\newcommand*{\authorfont}{\fontfamily{phv}\selectfont}
\usepackage[]{mathpazo}


  \usepackage[T1]{fontenc}
  \usepackage[utf8]{inputenc}
  \usepackage[nottoc,notlof,notlot]{tocbibind} 
  \renewcommand\refname{Referencias} % change original References title
  \renewcommand{\contentsname}{Tabla de contenidos  }



\usepackage{abstract}
\renewcommand{\abstractname}{}    % clear the title
\renewcommand{\absnamepos}{empty} % originally center

\renewenvironment{abstract}
 {{%
    \setlength{\leftmargin}{0mm}
    \setlength{\rightmargin}{\leftmargin}%
  }%
  \relax}
 {\endlist}

\makeatletter
\def\@maketitle{%
  \newpage
%  \null
%  \vskip 2em%
%  \begin{center}%
  \let \footnote \thanks
    {\fontsize{18}{20}\selectfont\raggedright  \setlength{\parindent}{0pt} \@title \par}%
}
%\fi
\makeatother




\setcounter{secnumdepth}{0}



\title{Contribución de los RPAS en investigación y conservación en espacios
protegidos: presente y futuro  }



\author{\Large Jesús Jiménez López\vspace{0.05in} \newline\normalsize\emph{Experto Universitario en Vehículos Aéreos no Tripulados y sus
Aplicaciones Civiles. Universidad de Cádiz}  }


\date{}

\usepackage{titlesec}

\titleformat*{\section}{\LARGE\bfseries}
\titleformat*{\subsection}{\normalsize\bfseries}
\titleformat*{\subsubsection}{\normalsize\itshape}
\titleformat*{\paragraph}{\normalsize\itshape}
\titleformat*{\subparagraph}{\normalsize\itshape}


\usepackage{natbib}
\bibliographystyle{apsr}


\newtheorem{hypothesis}{Hypothesis}
\usepackage{setspace}

\makeatletter
\@ifpackageloaded{hyperref}{}{%
\ifxetex
  \usepackage[setpagesize=false, % page size defined by xetex
              unicode=false, % unicode breaks when used with xetex
              xetex]{hyperref}
\else
  \usepackage[unicode=true]{hyperref}
\fi
}
\@ifpackageloaded{color}{
    \PassOptionsToPackage{usenames,dvipsnames}{color}
}{%
    \usepackage[usenames,dvipsnames]{color}
}
\makeatother
\hypersetup{breaklinks=true,
            bookmarks=true,
            pdfauthor={Jesús Jiménez López (Experto Universitario en Vehículos Aéreos no Tripulados y sus
Aplicaciones Civiles. Universidad de Cádiz)},
             pdfkeywords = {RPAs, UAVs, drones, espacios naturales, conservación, biodiversidad,
investigación, innovación},  
            pdftitle={Contribución de los RPAS en investigación y conservación en espacios
protegidos: presente y futuro},
            colorlinks=true,
            citecolor=blue,
            urlcolor=blue,
            linkcolor=magenta,
            pdfborder={0 0 0}}

\urlstyle{same}  % don't use monospace font for urls



\begin{document}
	
% \pagenumbering{arabic}% resets `page` counter to 1 
%
% \maketitle

{% \usefont{T1}{pnc}{m}{n}
\setlength{\parindent}{0pt}
\thispagestyle{plain}
{\fontsize{16}{20}\selectfont\raggedright 
\maketitle  % title \par  

}

{
   \vskip 13.5pt\relax \normalsize\fontsize{10}{12} 
\textbf{\authorfont Jesús Jiménez López} \hskip 15pt \vskip 8.5pt  \emph{\small Experto Universitario en Vehículos Aéreos no Tripulados y sus
Aplicaciones Civiles. Universidad de Cádiz}   

}

}






\begin{abstract}

    \hbox{\vrule height .2pt width 39.14pc}

    \vskip 8.5pt % \small 

\noindent En este estudio se procedió a identificar y resumir el estado actual y
las tendencias en el uso de los drones en proyectos científicos con
fines de conservación en espacios naturales protegidos, mediante la
recopilación y revisión de material bibliográfico en forma de artículos
científicos, revistas, proyectos de conservación y otras fuentes de
información relevantes.


\vskip 8.5pt \noindent \emph{Palabras claves}: RPAs, UAVs, drones, espacios naturales, conservación, biodiversidad,
investigación, innovación \par

    \hbox{\vrule height .2pt width 39.14pc}



\end{abstract}


{
\hypersetup{linkcolor=black}
\setcounter{tocdepth}{2}
\tableofcontents
}


\vskip 6.5pt

\noindent \doublespacing \section{Introducción}\label{introduccion}

Las aplicaciones de los vehículos aéreos no tripulados (RPAs, UAVs,
drones) en el campo de la conservación han sido directa o indirectamente
planteadas en un número cada vez mayor de artículos científicos. Durante
los últimos años ha habido un incremento significativo de las líneas de
investigación sobre vida silvestre que hacen uso de RPAS
\citep{Linchant2015}. Este auge ha conducido a un mayor desarrollo de
las metodologías que tratan de complementar o sustituir las técnicas
tradicionales enfocadas hacia el manejo de los espacios naturales
protegidos, en sus diversas facetas. Aunque el número de áreas
protegidas ha experimentado un aumento sensible a nivel mundial, con un
12 \% de la superficie terrestre bajo alguna figura de protección, hay
autores que resaltan la necesidad de mejorar las herramientas para
asegurar la efectividad de la conservación de la biodiversidad en áreas
protegidas \citep{Chape2005}. Sin embargo, a pesar de esta tendencia
positiva, la creación de estas áreas ha estado motivada en muchos casos
por criterios oportunistas que no reflejan necesariamente el valor
ecológico de los ecosistemas a preservar \citep{Knight2007}. En un
reciente informe realizado por la Sociedad Zoológica de Londres
\citep{Living2016} , se calcula que el tamaño de las poblaciones de vida
silvestre ha disminuido en un 52 \% en el periodo de 1970 hasta 2012.
Todos los indicios apuntan al hombre como principal desencadenante de la
ya denominada sexta extinción masiva en nuestro planeta. Tanto es así,
que algunos investigadores comienzan a hablar del antropoceno, como
inicio de una nueva época en el periodo Cuaternario. La fragmentación
del habitat, el aumento de la contaminación, especialmente grave en
ecosistemas de agua dulce, la sobreexplotación de los recursos, las
consencuencias a nivel global del cambio climático y el impacto de las
especies invasoras sobre poblaciones autóctonas han sido identificados
como los principales amenazas para la biodiversidad. La complejidad del
requieren soluciones noveles que puedan mejorar nuestra compresión de
los ecosistemas y permitan tomar medidas encaminadas a la preservación
de la biodiversidad. Aunque aún en proceso de consolidarse como
herramienta de trabajo esencial en la gestión de áreas protegidas, en
este estudio se argumenta el importante rol de los RPAS en conservación,
en la medida en que se superen las barreras técnicas y legales que
limitan su efectividad.

Existen actualmente algunas iniciativas que tratan de recoger el estado
actual de los RPAS en las áreas de la ecología y conservación. Con fecha
reciente de finalización , la revista \emph{Remote Sensing in Ecology
and Conservation} hizo una llamada a la comunidad científica para el
envío de propuestas dentro de la temática mencionada, con objeto de
sensibilizar a estudiantes y profesionales y demostrar el uso
responsable de RPAS. Es de esperar que del resultado de este llamamiento
se produzca un aumento significativo de la literatura científica en este
ámbito. Por otro lado, es remarcable la mayor presencia de portales en
internet que centran su actividad en torno a las aplicaciones con RPAS.
En el campo de la investigación aplicada en conservación
\url{http://conservationdrones.org/} es uno de los sitios de referencia.
Sus objetivos se enmarcan en la facilitación del uso y desarrollo de
RPAs en actividades conservacionistas. En su web se pueden consultar
casos de usos de RPAS cuyos resultados, dado el caracter pionero de
estos estudios, no siempre aparecen reflejados en artículos científicos.
Dentro de las aplicaciones de los RPAS con caracter general destaca la
comunidad online \url{http://diydrones.com/}, en la que tiene gran
acogida el uso de plataformas abiertas, de gran popularidad frente a los
tradicionales sistemas cerrados promovidos por compañias comerciales del
sector. Esto ha dado como resultado la reducción de los costes de estos
equipos, junto con el software asociado, permitiendo acercar la
tecnología disponible a un mayor número de usuarios y organizaciones.
Estas plataformas abiertas tienen la ventaja adicional de tener un mayor
grado de personalización de los equipos. El incremento en la
flexibilidad en el montaje de diferentes sensores y sistemas de control
permite cubrir las necesidades específicas de cada proyecto
\citep{Koh2012}.

Por otra parte, este estudio se analiza el grado en el que los RPAS
están llamados a complementar o sustituir herramientas tradicionales de
apoyo a la conservación en espacios naturales protegidos, tanto en su
vertiente científica como conservacionista. En este sentido, las
limitaciones desde el punto de vista financiero y tecnológico de la
teledetección, por la cual se obtienen imágenes de la superficie
terrestre a partir de sensores instalados en plataformas aéreas o
espaciales, son descritas por diversos autores \citep{Koh2012}. Si bien
es posible adquirir imágenes satelitales a coste cero (LandSat, MODIS,
Sentinel, etc.) en ocasiones la resolución espacial y temporal necesaria
para este tipo de estudios, junto con los problemas de presencia de
nubes especialmente acusados en zonas tropicales, reduce la efectividad
de la teledetección como herramienta de apoyo a la conservación. En
paises en vías de desarrollo, especialmente sensibles en cuanto a
dotaciones presupuestarias, se han desarrollado con gran éxito programas
de monitoreo y vigilancia a partir del uso del RPAS, eliminando los
inconvenientes descritos con anterioridad. Además, el gran tamaño de
estas áreas protegidas reducen en muchos casos la efectividad de los
trabajos de campo, por lo que los RPAS se han posicionado como un
complemento adecuado para las actividades de conservación
\citep{Zahawi2015}.

Junto con las oportunidades de reducir los costes y la carga logística,
existe una sensibilización cada vez mayor en relación a la mejora de la
seguridad de los biólogos en las operaciones de campo. Algunos estudios
señalan que los accidentes aéreos se situan como primera causa de
mortandad en especialistas en vida silvestre en los Estados Unidos
\citep{Sasse2003}. Desde el punto de vista de las especies de fauna
observadas, algunos estudios analizan la respuesta de las especies
frente a los RPAS y consideran la aplicación de buenas prácticas con
objeto de reducir el impacto negativo en su bienestar y evitar
perturbaciones en los patrones de comportamiento.

Finalmente, algunos autores señalan la necesidad de mejorar el marco
regulatorio respecto al uso civil de los RPAS \citep{Nugraha2016}. En
los Estados Unidos y en la mayoría de los paises de Europa consultados,
se han adoptado leyes provisionales que en cierta medida equiparan el
manejo de los RPAS con el de aeronaves tradicionales. Este tipo de
restricciones podría limitar las posibilidades de uso de los RPAS en el
ámbito de la conservación, por lo que se hace patente la necesidad
urgente de armonizar la legislación en relación a este tipo de
actividades. En términos generales, la situación en America Latina es
desigual, con algunos paises que siguen sin desarrollar leyes
específicas para hacer frente al auge de los RPAS tanto en el sector
civil como militar \citep{Nacion2013}. Africa es uno de los continentes
donde el impacto de los drones en conservación ha tenido mayor
repercusión. Sin embargo, según la opinión de algunos conservacionistas,
su uso no ha estado exento de problemas, dando como resultado gobiernos
que han prohibido total o parcialmente su uso, argumentando problemas de
seguridad nacional en detrimento de la protección de los espacios
naturales protegidos \citep{Andrews2014}.

La incertidumbre de los usuarios ha promovido el desarrollo de
asociaciones con objeto de asesorar sobre los aspectos legales a tener
en cuenta durante la operación. En España, la Asociación Española de
Drones y Afines \url{https://www.aedron.com} promueve un uso consciente
y responsable de los RPAS y organiza seminarios para informar a los
socios sobre temas de interés. En su web se puede consultar el borrador
de la nueva normativa que regula la utilización civil de las aeronaves
pilotadas por control remoto en España \citep{Aedron2017}.

\section{Métodos}\label{metodos}

Para alcanzar los objetivos propuestos se procedió a la revisión
bibliográfica de artículos, literatura gris, tesis de postgrado, sitios
web y revistas especializadas, siguiendo una línea similar a otros
estudios realizados con anterioridad \citet{Linchant2015}. Mediante
artículos seleccionados para el curso de Experto Universitario en
Vehículos Aéreos no Tripulados y sus Aplicaciones Civiles organizado por
la Universidad de Cádiz en su edición de 2016-2017, junto con
herramientas como Google Schoolar, ResearchGate y Mendeley Desktop se
obtuvo la mayor parte de la bibliografía seleccionada, mientras que el
uso de los motores de búsqueda por internet incluyeron el resto de
materiales mencionados. Los principales criterios de búsqueda por
palabras claves incluyeron los vehículos aéreos no tripulados en sus
diversas acepciones, incluyendo acrónimos (RPAS, UAV, drones, etc.),
junto con términos que hacen referencia a áreas naturales,
fundamentalmente en inglés. Dicha actividad tuvo lugar hasta el mes de
Abril, 2017.

La información recolectada se categorizó según los propósitos de
aplicación de los RPAS en la conservación en espacios protegidos. La
mayoría de los estudios analizados se centran en la viabilidad de los
RPAS la conservación a gran escala, junto con la caracterización de
poblaciones y comunidades de vida silvestre, especialmente en estudios
de distribución y abundancia. Las aplicaciones en actividades de
monitoreo y mapeo en ecosistemas terrestres y acuáticos también ocupan
un lugar destacado, en estrecha relación con las mejoras en la
resolución espacio-temporal de las imágenes adquiridas. Adicionalmente
se revisan algunos aspectos de índole social recogidos en los materiales
seleccionados y que son motivo de controversia, con especial referencia
a la privacidad de las personas y el bienestar de las especies
estudiadas, las implicaciones éticas y legales y su repercusión en la
efectividad de estas técnicas en la conservación a largo plazo. En
cualquier caso, dado el carácter multidisciplinar y multipropósito de
estos estudios existe cierto solape entre los objetivos marcados dentro
de cada proyecto, por lo que se ha tenido en cuenta aquellos objetivos
que mayor peso tienen en el contexto de la investigación.

La información seleccionada se presenta en formato tabular,
identificando los paises implicados, el propósito principal de cada
estudio, junto con las técnicas y materiales empleados, haciendo
referencia explicita al tipo de aplicación y plataformas de vuelo, tanto
de ala fija como de pala rotatoria. Finalmente se discuten los posibles
escenarios que presentan los RPAS como herramientas fundamentales para
contribuir a la consecución de los objetivos de conservación en espacios
protegidos, destacando algunas tendencias y oportunidades que aún no han
sido convenientemente explotadas.

\begin{sidewaystable}
\centering
\captionsetup{font=scriptsize,labelfont=scriptsize}
\caption{Estudios con RPAS realizados en áreas protegidas, características técnicas de la plataforma y especies objetivos}
\label{my-label}
\tiny
\begin{tabular}{p{2.5cm}p{1cm}p{3cm}p{1cm}p{2cm}p{2cm}p{1cm}p{2cm}p{2cm}p{1cm}p{0.5cm}}
\cmidrule(r){1-11}

Estudio & Tipo de Estudio & Objetivo/s & País & Lugar & Especie & Tipo RPAS & Modelo RPAS & Sistema de captura & Georref. & Costo \\ \cmidrule(r){1-11}

\cite{PazmanyMulero2015}  & Ecología espacial & Estudio comparativo modelos distribución de especies & España & Parque Nacional de Doñana & Bos taurus  & Ala fija & Easy Fly plane, Ikarus autopilot, Eagletree GPS logger & Panasonic Lumix LX-3 11MP & Si & 5700 euros \\ 

\citealt{Hodgson2013} & Tipo estudio & Determinar la eficacia para detectar e identificar dugongs.  Comprobar la actitud de los RPAS en diferentes condiciones ambientales. Determinar la resolución ideal  & Australia & Shark Bay Marine Park & Dugong & Ala fija &  ScanEagle & Nikon® D90 12 megapixel digital SLR camera  & Si & Costo  \\ 

\cite{Ivosevic2015}  & Monitoreo de ecosistemas terrestres & Monitoreo de habitats en zonas restringidas; Modelos; Comprobar la actitud de los RPAS en diferentes condiciones ambientales. & South Korea & Chiaksan National Park;Taeanhaean National Park & Multicóptero & DJI Phantom 2 Vision+ & built-in full HD videos  1080p/30fps and 720p/60fps, 14 megapixels 4384x3288 resolution camera & Especie & Si & Costo \\ 

\cite{Vermeulen2013}  & Estudio de poblaciones & Estudio &  Burkina Faso & Nazinga Game Ranch & Ala fija  &  Gatewing 6100 UAS & Ricoh GR3 still camera (10 megapixels, 28 mm Charged Coupled Device) & Loxodonta africana & Si & 426 / day \\ 


\cite{Chabot2014}  & Monitoreo de ecosistemas & Calidad del habitat & Canada & South shore of the St. Lawrence River, Lake Saint Pierre Biosphere Reserve  & Ala fija  &  AI-Multi UAS & 10-megapixel Canon S90 & Ixobrychus exilis & Si & 20,000  \\ 
\cite{Mulero-Pazmany2014}  & Vigilancia en áreas protegidas & Estudio & Africa & Parque Nacional de Doñana & Modelo RPAS  & istema de captura & Especie & Si & Costo  \\ 

\cite{PazmanyMulero2015}  & Tipo estudio & Estudio & España & Parque Nacional de Doñana & Modelo RPAS & istema de captura & Especie & Si & Costo  \\ 
\cite{PazmanyMulero2015}  & Tipo estudio & Estudio & España & Parque Nacional de Doñana & Modelo RPAS  & istema de captura & Especie & Si & Costo  \\ 
\cite{PazmanyMulero2015}  & Tipo estudio & Estudio & España & Parque Nacional de Doñana & Modelo RPAS  & istema de captura & Especie & Si & Costo  \\ \\ \cmidrule(r){1-11}
\end{tabular}
\end{sidewaystable}

\section{Discusión}\label{discusion}

En discusión comenta el resultado de esa tabla y los porqués (ej se usan
más multicópteros que fixed por\ldots{}) y las limitaciones que señalan
los usuarios o conflictos que hayan podido encontrar (con el parque,
técnicos etc).

\subsection{Estudios de fauna y vida
silvestre}\label{estudios-de-fauna-y-vida-silvestre}

\subsubsection{Estudios de poblaciones}\label{estudios-de-poblaciones}

Actualmente se experimenta un incremento de los trabajos de
investigación que incorporan el uso de RPAS en diversas disciplinas
relacionadas con la ecología de poblaciones y comunidades. A partir de
la información obtenida, algunos estudios comparan el desarrollo de
modelos de distribución de especie y caracterización del habitat frente
a sistemas de seguimiento por satélite o radiotelemetría, que permiten
registrar el movimiento del animal para su análisis posterior
\citep{PazmanyMulero2015}, \citep{Mulero-Pazmany2015}. Otros estudios
recientes combinan el uso de radiolocalizadores en RPAS y
radiomarcadores VHF en estudios de seguimiento de fauna
\citep{Bayram2016}.

En determinados casos, frente a las dificultades para detectar
directamente a la especie de interés, los estudios se enfocan en la
localización y caracterización de sus áreas de cría y nidificación
\citep{VanAndel2015}. En áreas protegidas de gran extensión se han
ensayado con éxito el conteo de grandes mamíferos terrestres , no
habiéndose registrado reacciones adversas en vuelos realizados a cierta
altura \citep{Vermeulen2013}. La estimación de poblaciones de mamíferos
en ecosistemas marinos también ha sido documentado con resultados
positivos \citep{Hodgson2013}. El uso de RPAS también ha encontrado su
nicho de actuación en el monitoreo de aves, especialmente en estudios de
dinámica poblacional en colonias \citep{Sarda-Palomera2012}. La utilidad
de estos sistemas también queda manifiesta en la inspección y
caracterización de nidos de aves en zonas de dificil acceso
\citep{Weissensteiner2015}, permitiendo evaluar el estado en el que se
encuentran de forma menos intrusiva.

Dada la masiva cantidad de información que generada, no es de extrañar
que se hayan aplicado métodos desarrollados en el campo de la visión
computerizada, dirigidos al conteo automático de individuos capturados
en las escenas adquiridas por los sensores fotográficos
\citep{Lhoest2015},\citep{Abd-Elrahman2005a}, \citep{VanGemert2015}.
Esto conlleva una reducción de los costes respecto al conteo manual de
las escenas adquiridas, con la ventaja adicional de no estar sujetos en
mayor o menor medida a la interpretación del especialista. En este
sentido, los métodos de observación directa desde vehículos aéreos
tripulados también representan desventajas con respecto a la toma de
imágenes aéreas, puesto que necesitan un mayor número de observadores
que garantizen un conteo exahustivo de las poblaciones para evitar
errores en la estimación.

Fuera de la literatura científica, existen proyectos para el monitoreo
de la fauna tanto en ecosistemas marinos como terrestres. A partir de la
información recopilada en la comunidad online
\url{https://conservationdrones.org} se han identificado varios estudios
relacionados con el registro de individuos en poblaciones situadas en
áreas protegidas o frecuentemente visitadas por especies sujetas a
alguna figura de amenaza, siendo la mayoría de estos proyectos
respaldados por organizaciones no gubernamentales y centros de
investigación. Por ejemplo, un estudio realizado en la cuenca del
Amazonas en Brasil está experimentando el uso de drones para mejorar la
estimación de la densidad y abundancia de diferentes especies de
delfines, comparándolo con la observación directa realizada por
especialistas \citep{WichS2017}. Dentro de los objetivos de la
investigación se contempla la validación y armonización de ambas
metodologías y de forma indirecta, evaluar la viabilidad para su
aplicación regular en proyectos de monitoreo con similar propósito,
teniendo en cuenta el coste-beneficio de la ejecución.

\citep{PazmanyMulero2015} blah blah distribution patterns within a
protected area, which is critical for ecosystem management (Bailey
.Complementan a otros sistemas que permiten Mapa de zonas

\subsubsection{Evaluación de
infraestructuras}\label{evaluacion-de-infraestructuras}

Otros trabajos resaltan la utilidad de los RPAS en la caracterización y
evaluación del riesgo de infraestructuras humanas, fundamentalmente
dirigidos a especies de aves que nidifican en postes de líneas
eléctricas de alta tensión, haciéndolas especialmente vulnerables a la
electrocución \citep{Mulero-Pazmany2014a}. Si bien esto estudios no
están dirigidos exclusivamente a áreas protegidas, podrían resultan de
especial interés en zonas de amortiguamiento, donde el desarrollo
antrópico es más acusado.

\subsection{Monitoreo y mapeo de ecosistemas terrestres y
acuáticos}\label{monitoreo-y-mapeo-de-ecosistemas-terrestres-y-acuaticos}

\subsubsection{Ecosistemas acuáticos}\label{ecosistemas-acuaticos}

\subsubsection{Ecosistemas terrestres}\label{ecosistemas-terrestres}

\subsection{Apoyo para el cumplimiento de las leyes en áreas
protegidas}\label{apoyo-para-el-cumplimiento-de-las-leyes-en-areas-protegidas}

\subsubsection{Caza furtiva}\label{caza-furtiva}

Los RPAS también tienen especial proyección en el control y vigilancia
de áreas protegidas. Así lo demuestran diferentes estudios, enfocados
principalmente en el control de la caza furtiva. En áreas marinas
protegidas, los RPAS de ala fija podrían permitir una vigilancia más
eficaz contra la pesca furtiva. L

En estos casos el desarrollo de sistemas autónomos de mayor FPV

\subsubsection{Otras actividades
ilegales}\label{otras-actividades-ilegales}

\subsubsection{Etica animal}\label{etica-animal}

\citep{Ditmer2015}

\section{Resultados y conclusiones}\label{resultados-y-conclusiones}

Una de las mayores limitaciones en el desarrollo y aplicación de los
RPAS en estudios de conservación se debe a las restricciones impuestas
por la legislación actual. Diversos autores mencionan los problemas
burocráticos para obtener permisos de investigación que hagan uso de
esta tecnología emergente \citep{FEE:FEE201513274}. Consideramos que una
regulación favorable permitiría aumentar las oportunidades en el sector,
estimulando la innovación técnológica.

Los conflictos sociales también ocupan un capítulo importante en el
futuro de los drones en conservación. Comunidades implicadas

Hasta ahora no ha habido un desarrollo especifico de drones que

En cuanto a ética animal, pilotos experimentados y conscientes.

En la mayoría de los estudios analizados se remarca el bajo coste
operacional de los drones frente a otras herramientas de conservación.
En estudios de caracter multitemporal o con necesidades de alta
resolución espacial las ventajas son especialmente patentes. Hay drones
que permiten seguir el objetivo\ldots{} limitación batería

Riesgos de los vehiculos aereos tripulados, costes,

Explicar que estudios usan multirotores y ala fija Mejoras en la
autonomia de vuelo, junto con el desarrollo de sensores de mayor
resolución. En al medida en la que aumente el desarrollo de sensores con
mayor resolución espectral, el abanico de posibilidades de

Mejoras en los algoritmos para la detección de fauna junto al desarrollo
de sensores fotográficos de mayor resolución en un formato más reducido,
permitiendo una menor carga y por tanto un mayor tiempo de vuelo.

Elaborar conclusiones basadas en los resultados obtenidos, destacando
los campos con mayor interés.

A raíz de los resultados obtenidos parece claro que el ámbito de la
conservación se va

Paises con escasos recursos, utilidad del control y vigilancia,
especialmente en areas marinas

\newpage
\singlespacing 
\bibliography{/home/jesus/Documents/CURSOS/drones/RTF/project/master.bib,/home/jesus/Documents/CURSOS/drones/RTF/project/internet.bib}

\end{document}
