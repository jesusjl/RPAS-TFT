% interactcadsample.tex
% v1.03 - April 2017

\documentclass[]{interact}
%\documentclass[a4paper]{article}
\usepackage{epstopdf}% To incorporate .eps illustrations using PDFLaTeX, etc.
\usepackage{subfigure}% Support for small, `sub' figures and tables
%\usepackage[nolists,tablesfirst]{endfloat}% To `separate' figures and tables from text if required



\theoremstyle{plain}% Theorem-like structures provided by amsthm.sty
\newtheorem{theorem}{Theorem}[section]
\newtheorem{lemma}[theorem]{Lemma}
\newtheorem{corollary}[theorem]{Corollary}
\newtheorem{proposition}[theorem]{Proposition}

\theoremstyle{definition}
\newtheorem{definition}[theorem]{Definition}
\newtheorem{example}[theorem]{Example}

\theoremstyle{remark}
\newtheorem{remark}{Remark}
\newtheorem{notation}{Notation}


\usepackage{layout}
\setlength{\headsep}{15pt}
\usepackage{booktabs}
\usepackage{rotating}
\usepackage{graphicx}
\usepackage{setspace}
\usepackage{pdflscape}
\onehalfspacing
\usepackage{helvet}


\usepackage[T1]{fontenc}
\usepackage[utf8]{inputenc}
\usepackage[nottoc,notlof,notlot]{tocbibind} 

\usepackage{abstract}
\renewcommand{\abstractname}{}    % clear the title
\renewcommand{\absnamepos}{empty} % originally center

\renewenvironment{abstract}
 {{
    \setlength{\leftmargin}{10mm}
    \setlength{\rightmargin}{\leftmargin}%
  }
  \relax}
 {\endlist}


\newtheorem{hypothesis}{Hypothesis}
\usepackage{setspace}

\makeatletter
\@ifpackageloaded{hyperref}{}{%
\ifxetex
  \usepackage[setpagesize=false, % page size defined by xetex
              unicode=false, % unicode breaks when used with xetex
              xetex]{hyperref}
\else
  \usepackage[unicode=true]{hyperref}
\fi
}
\@ifpackageloaded{color}{
    \PassOptionsToPackage{usenames,dvipsnames}{color}
}{%
    \usepackage[usenames,dvipsnames]{color}
}
\makeatother
\hypersetup{breaklinks=true,
            bookmarks=true,
            pdfauthor={Jesús Jiménez López (University) and Margarita Mulero-Pázmány (University)},
             pdfkeywords = {RPAS, UAV, drones, natural protected areas, conservation, biodiversity,
research, innovation},  
            pdftitle={Contribution of RPAS in research and conservation in natural protected
areas: present and future},
            colorlinks=true,
            citecolor=blue,
            urlcolor=blue,
            linkcolor=magenta,
            pdfborder={0 0 0}}
\urlstyle{same}  % don't use monospace font for urls


% notice vspace after title
\title{Contribution of RPAS in research and conservation in natural protected
areas: present and future\vspace{0.25in}  }



\author{\Large Jesús Jiménez López\vspace{0.05in} \newline\normalsize\emph{University }   \and \Large Margarita Mulero-Pázmány\vspace{0.05in} \newline\normalsize\emph{University }  }


\date{}

\usepackage{titlesec}


\titleformat*{\section}{\large\bfseries}
\titlespacing{\section}{0em}{\parskip}{-\parskip}
\titleformat*{\subsection}{\normalsize\bfseries}
\titleformat*{\subsubsection}{\normalsize\itshape}
\titleformat*{\paragraph}{\normalsize\itshape}
\titleformat*{\subparagraph}{\normalsize\itshape}


\makeatletter
\@ifpackageloaded{hyperref}{}{%
\ifxetex
  \usepackage[setpagesize=false, % page size defined by xetex
              unicode=false, % unicode breaks when used with xetex
              xetex]{hyperref}
\else
  \usepackage[unicode=true]{hyperref}
\fi
}
\@ifpackageloaded{color}{
    \PassOptionsToPackage{usenames,dvipsnames}{color}
}{%
    \usepackage[usenames,dvipsnames]{color}
}
\makeatother
\hypersetup{breaklinks=true,
            bookmarks=true,
            pdfauthor={Jesús Jiménez López (University) and Margarita Mulero-Pázmány (University)},
             pdfkeywords = {RPAS, UAV, drones, natural protected areas, conservation, biodiversity,
research, innovation},  
            pdftitle={Contribution of RPAS in research and conservation in natural protected
areas: present and future},
            colorlinks=true,
            citecolor=blue,
            urlcolor=blue,
            linkcolor=magenta,
            pdfborder={0 0 0}}

\urlstyle{same}  % don't use monospace font for urls



\begin{document}

\articletype{ARTICLE TEMPLATE}

% \pagenumbering{arabic}% resets `page` counter to 1 
%
\author{
\name{A.~N. Author\textsuperscript{a}\thanks{CONTACT A.~N. Author. Email: latex.helpdesk@tandf.co.uk} and John Smith\textsuperscript{b}}
\affil{\textsuperscript{a}Taylor \& Francis, 4 Park Square, Milton Park, Abingdon, UK; \textsuperscript{b}Institut f\"{u}r Informatik, Albert-Ludwigs-Universit\"{a}t, Freiburg, Germany}
}
% \author{\Large Jesús Jiménez López\vspace{0.05in} \newline\normalsize\emph{University}  \and \Large Margarita Mulero-Pázmány\vspace{0.05in} \newline\normalsize\emph{University} }
\title{Contribution of RPAS in research and conservation in natural protected
areas: present and future}
\maketitle  % title \par 
  




  \textbf{ABSTRACT}: \par
  {This paper is aimed at establishing the current state and trends in the
use of RPAS in scientific projects for conservation purposes in natural
protected areas, through the collection and revision of bibliographic
material in the form of scientific articles, journals, conservation
projects and other sources of relevant information.}


    
   \textbf{KEYWORDS}: \par
   {RPAS, UAV, drones, natural protected areas, conservation, biodiversity,
research, innovation}

% 

\vskip 6.5pt

\noindent  \section{Introduction}\label{introduction}

\subsection{Current context}\label{current-context}

Civil applications of remotely piloted aircraft systems (RPAS, also
known as unmanned aerial systems, UAS, drones) have been raised in an
increasing number of scientific articles. During the last few years
there have been a significant amount of wildlife research projects in
natural protected areas using RPAS (J. Linchant et al. 2015; Christie et
al. 2016). In most cases, these are pilot studies that assess the
capacity of these systems in relation to traditional conservation
instruments, delimiting their strenghts and weakness, establishing
guidelines and recommendations, which together result in new
perspectives of application.

In the early 1980s, the first experiments with RPAS on environmental
issues began with the objective of acquiring aerial photographs and
demonstrating their usefulness in forestry applications, the management
of fish resources or the coupling of sensors for atmospheric studies
(Tomlins and Lee 1983). By the end of the 20th century, the first
mapping surveys of vegetation in threatened species appear (Quilter,
1997), while with the arrival of the new millennium the number of
publications began to increase significantly (see Hardin and Jensen
2013). At present there are some initiatives that seek to determine the
current state of the RPAS in the areas of ecology and conservation.
Recently, the journal \emph{Remote Sensing in Ecology and Conservation}
made a call to the scientific community for the sending of proposals, in
order to sensitize students and professionals and demonstrate the
responsible use of RPAS. It is expected that the result of this appeal
will produce a significant increase in the scientific literature in this
area. On the other hand, it is remarkable the greater presence of
portals in Internet that center their activity around civil applications
with RPAS. In the field of applied research to conservation, the web
portal \url{http://conservationdrones.org/} is a worldwide reference,
whose contents illustrate recent pioneering projects, so they are not
always reflected in the scientific literature. The popularity of RPAS
has transcended the scientific-technical field, giving rise to the
emergence of user communities with a large presence on the Internet. One
of the most active portals is \url{http://diydrones.com/}, which brings
together fans of the do-it-yourself philosophy that encourages the use
of open platforms versus the traditional closed systems offered by the
traditional industry. This has resulted in the reduction of costs of
these equipment and, together with the development of specialized free
software, have led to the democratization of technology, bringing it
closer to a greater number of users and organizations. The scientific
community has probably benefited from this general trend. For some
authors, the flexibility in the assembly of RPAS offers in principle a
greater degree of customization, allowing to combine different sensors
and control systems according to the particular needs of each project
and within the research group itself (Koh and Wich 2012). In the
commercial field, more companies offer RPAS of great performance and
qualified to develop professional applications, reason why the sector
benefits from a great dynamism.

The limitations from the financial and technological point of view of
remote sensing, by which images of the earth's surface are obtained from
sensors installed on aerial or space platforms, are described by several
authors (Koh and Wich 2012; A. Rodríguez et al. 2012). While it is
possible to acquire satellite images at low or virtually zero cost
(LandSat, MODIS, Sentinel, etc.), most of these platforms operate on a
global or regional scale. The limited spatial and temporal resolution,
together with the problems of cloud presence especially accentuated in
tropical areas, reduces the effectiveness of remote sensing in the
collection of data in detail scale, according to the requirements of
ecological studies at the level of species, habitats or populations
(Wulder et al. 2004). In addition, the large extent of these protected
areas significantly increases the costs of field work, while increasing
risks in particularly inaccessible areas. As a consequence, RPAS have
been positioned as an appropriate complement for conservation activities
(Zahawi et al. 2015) avoiding to a greater or lesser extent some of the
above-mentioned disadvantages. In developing countries, especially
sensitive in terms of budgetary allocations and technical capacities,
monitoring and surveillance programs are being successfully developed
through the use of RPAS. For example, by capturing aerial images in the
Volta delta, Ghana, a team of scientists measures the effects of climate
change on coastal areas and evaluates the effectiveness of prevention
and restoration measures against erosive processes (Gerster/Panos 2017).
In principle, manned aerial vehicles offer a better alternative in
capturing images of the earth's surface, but their use is not justified
in studies at the local scale due to excessively high operational costs.
On the other hand, the risk of air accidents is higher, ranking as the
leading cause of death in wildlife specialists in the United States
(Sasse 2003).

In order to measure the impact of drones in wildlife studies, some
experiments analyze the response of birds to RPAS (Vas et al. 2015).
OOther trials focus on mammals and measure physiological stress and
possible changes in behavior (M. A. Ditmer et al. 2015). As a result,
manuals of good practices and recommendations of use of RPAS to mitigate
the negative impact on the welfare of the species and to avoid
disturbances in the behavior patterns are being documented.

Finally, some authors point out the need to improve the regulatory
framework regarding the civil use of RPAS (Nugraha, Jeyakodi, and Mahem
2016). In the United States and in most of the European countries
consulted, interim legislation has been adopted which, to a certain
extent, equates the management of RPAS with that of traditional
aircraft. Legal restrictions could limit the possibilities of use of the
RPAS in the field of conservation, which makes clear the urgent need to
harmonize the legislation in relation to this type of activities. In
general terms, the situation in Latin America is uneven, with some
countries still not developing specific laws to deal with the boom of
the RPAS in both the civil and military sectors (Agencia EFE 2013).
Africa is one of the continents where the impact of drones in
conservation has had greater repercussions. However, in the opinion of
some conservationists, their use has not been without problems,
resulting in governments that have totally or partially prohibited drone
operations, arguing national security problems to the detriment of
protection of protected natural areas (Andrews 2014). The uncertainty of
the users has promoted the development of associations in order to
advise on the legal aspects to be taken into account during the
operation. In Spain, the Spanish Association of Drones and Related
Affinities \url{https://www.aedron.com} ppromotes a conscious and
responsible use of RPAS and organizes seminars to inform members about
topics of interest. On its website, it is possible to consult the draft
of the new regulation that regulates the civil use of the piloted
aircraft by remote control in Spain (AEDRON 2017). Globally, other
initiatives have emerged, with the International Association for
Unmanned Vehicle Systems (AUVSI) \url{http://www.auvsi.org} the largest
nonprofit organization in the world dedicated to advancing the community
of unmanned aerial vehicles users.

\subsection{Protected natural areas}\label{protected-natural-areas}

Natural protected areas are those in which human intervention has not
significantly altered the presence and functioning of the biotic and
abiotic elements that comprise it (Bravo 2008). They fulfill the
objectives of conservation of the biophysical and cultural environment,
where initiatives are promoted in the scientific, educational and
recreational field compatible with the natural environment and the
socioeconomic activities framed in the sustainable development of the
territory. They are under some national or international protection
figure and regulated through specific management plans. Despite the fact
that the number of protected areas has increased considerably at a
global level, with 15.4\% of the land area and 8.4\% of the marine areas
under some protection figure (Juffe-Bignoli et al. 2014) there are
authors who emphasize the need to improve the protected areas management
tools that ensure the effectiveness of the conservation of biodiversity
(Chape et al. 2005).Moreover, some protected areas suffer degradation
processes, continue to shrink in size or have ceased to exist (Mascia
and Pailler 2011). In other cases they have been declared under
opportunistic criteria that do not necessarily reflect the ecological
value of the ecosystems to be preserved (Knight and Cowling 2007). In a
recent report by the Zoological Society of London (WWF 2016) , the size
of wildlife populations has been estimated to have decreased by 52\% in
the period 1970 to 2012. Habitat fragmentation, severe pollution
particularly in freshwater ecosystems, overexploitation of resources,
environmental impacts of climate change and the impact of invasive
species on indigenous populations have been identified as the main
threats to biodiversity (Barnosky et al. 2011; Conabio 2017).

The Group on Earth Observations Biodiversity Observation Network
(GEOBON) has identified a set of Essential Biodiversity Variables
(Pereira et al. 2013) as key components for the collection of
environmental information that allow us to know the global state of our
ecosystems and support better decision-making on biodiversity (Forum
2008). In addition, the Convention on Biological Diversity, developed as
part of the United Nations Environment Program (UNEP), established in
Nagoya, Japan, a strategic plan for the period 2011-2020 which includes
the so-called Aichi targets for biological diversity. Among the stated
objectives is the increase of protected area systems of special
importance for biodiversity and ecosystem services (Goal 11) and
establishes a set of governance, equity, management, representativeness
and ecological connectivity criteria for the inclusion of Priority areas
for conservation.

To address the current environmental crisis, it is necessary to develop
solutions that improve our knowledge of the current state of
biodiversity and allow us to manage our natural resources efficiently.
In this context, the present document reviews the current state of the
RPAS in studies of conservation and management of protected areas,
mentioning the technical and legal barriers that limit their
effectiveness.

\section{Methods}\label{methods}

To achieve the proposed objectives, a bibliographical review of
scientific articles, gray literature, postgraduate theses, websites and
specialized journals was carried out, following a similar line to other
studies (J. Linchant et al. 2015; Christie et al. 2016)The main tools
for selection of the cited bibliography include Google Schoolar,
Research Gate and Mendeley Desktop, while the use of Internet search
engines include other references outside the scientific scope. Key
search criteria for keywords included unmanned aerial vehicles in their
various meanings and acronyms (RPAS, UAV, drones, etc.), along with a
variety of terms referring to natural protected areas, primarily in
English. This activity took place from April 4, 2017 until May 12 of the
same year.

The selected information was categorized according to the role played by
RPAS in direct or indirect relation to conservation in natural areas. It
is presented in tabular format, identifying where the study was
conducted, the expected accomplishments and technical specifications of
the aerial platform. Finally, possible scenarios for implementing RPAS
as essential tools to help achieve conservation plans in protected areas
are discussed, highlighting some trends and opportunities that
apparently have not yet been adequately exploited.

\section{Results}\label{results}

\subsection{Wildlife Studies}\label{wildlife-studies}

One of the central themes of ecology is the development of statistical
models of species distribution that allow inferring the potential or
suitable habitat of organisms from the collection of environmental
information and presence data from different sources (Mateo, Felicísimo,
and Muñoz 2011). Radio telemetry is one of the most common methods for
collecting movement data in individuals marked with geolocators. Some
studies compared the use of RPAS against these systems (Mulero-Pázmány
et al. 2015) in large animals, easily identifiable by high resolution
aerial images. The authors obtain similar results regarding the
performance of the models but highlight the cost-benefit factor of RPAS
as the main advantage. Usually the number of available geolocators is
limited, with implications on sample size. Added to the risk of marking
individuals under non-random criteria, the robustness of the analysis
can be seriously affected. However, main advantage of radio telemetry is
based on its ability to generate large volumes of data over a longer
period of time. Regarding positional accuracy, GPS included in RPAS have
a maximum error between 1 and 3 meters, while under unfavorable
conditions can be more than 20 meters for telemetry. In any case, the
authors point out that both methodologies have the potential to
complement each other throughout all phases of the study. (Bayram et al.
2016)

Other innovative techniques have recently been illustrated in scientific
papers evaluating the feasibility of pairing radio locators in RPAS in
the search for individuals marked with VHF radio collars (Körner et al.
2010; Bayram et al. 2016; Cliff et al. 2015; Leonardo et al. 2013). In
some cases, in order to overcome the difficulties of directly detecting
the species of interest, the studies focus on locating and
characterizing their breeding and nesting areas (van Andel et al. 2015).
Large terrestrial mammals have been successfully counted, and no adverse
reactions have been recorded on flights at a minimum height of 100
meters (Jain 2013). The estimation of mammalian populations in marine
ecosystems has also been documented with positive results (A. Hodgson,
Kelly, and Peel 2013), while in birds RPAS have been used to study
population dynamics in colonies (Sardà-Palomera et al. 2012). The
usefulness of these systems is also evident in the inspection and
characterization of bird nests in areas difficult to access
(Weissensteiner, Poelstra, and Wolf 2015), allowing to evaluate the
state in which nests are in a less intrusive way.

Given the large amount of information generated, it is not surprising
that methods have been developed in the field of computer vision that
allow the automatic counting of individuals captured in scenes acquired
by photographic sensors (Lhoest et al. 2015; Abd-Elrahman, Pearlstine,
and Percival 2005; van Gemert et al. 2015). This leads to a reduction in
costs with regard to the manual counting, with the additional advantage
of not being subject to a greater or lesser extent to the interpretation
of the specialist. In this regard, methods of direct observation from
manned aerial vehicles also represent disadvantages with respect to
aerial imaging, since they require a greater number of observers to
guarantee an exhaustive count of populations to avoid errors in
estimation.

Outside the scientific literature, there are projects for monitoring
wildlife in both marine and terrestrial ecosystems, most of which are
supported by non-governmental organizations and research centers. Based
on information gathered at \url{https://conservationdrones.org} sseveral
studies have been identified pursuing methods for registering
individuals in marine mammal populations, primates and macrofauna in
general, located in protected areas or frequently visited by wildlife
under some legal figure of threat. For example, a work conducted in the
Amazon Basin in Brazil is experimenting with the use of drones to
improve the density and abundance estimation of different species of
dolphins, compared with direct observation by specialists (S. Wich
2017). The main research aims include the validation and harmonization
of both methodologies and, indirectly, evaluate the feasibility for its
regular application in monitoring projects with a similar purpose,
taking into account the cost-benefit of the execution.

\begin{landscape}
\begin{table}
\caption{RPAS APPLICATIONS IN  NATURAL PROTECTED AREas}
\tiny
\begin{tabular}{p{3cm}p{1.2cm}p{3cm}p{1cm}p{2cm}p{2cm}p{1cm}p{2cm}p{2cm}p{1cm}p{2cm}}

\cmidrule(r){1-11}

Study & Protected Area & Aims & Country & Place & Species & RPAS type &  RPAS model & Sensor & Georef. & Costs \\ \cmidrule(r){1-11}

\multicolumn{11}{c}{} \\
\multicolumn{11}{c}{ {\bf ESTUDIOS DE FAUNA Y VIDA SILVESTRE}  } \\
\multicolumn{11}{c}{} \\
\cite{pazmany_mulero_unmanned_2015}  & Si & Estudio comparativo modelos distribución de especies & España & Parque Nacional de Doñana & Bos taurus  & Ala fija & Easy Fly plane, Ikarus autopilot, Eagletree GPS logger & Panasonic Lumix LX-3 11MP & Si & 
5700 euros \\ 

\cite{hodgson_unmanned_2013} & Si & Detección e identificación de dugongs.  Comprobar actitud  RPAS en diferentes condiciones ambientales. Determinar altura y resolución ideal  & Australia & Shark Bay Marine Park & Dugong & Ala fija &  ScanEagle & Nikon D90 12 megapixel digital SLR camera  & Si & ?  \\ 

\cite{longmore_adapting_2017} & No & Desarrollo de software detección especies infrarrojo térmico & Inglaterra & ? & Fauna & Multicóptero & 750mm carbon-folding Y6 multi-rotor APM 2 autopilot 3Drobotics & FLIR, Tau 2 LWIR Thermal Imaging Camera Core  & ? & ?  \\ 

\cite{wilson_feasibility_2017}  & No & Monitoreo bioacústico con RPAS & USA & State Game Lands & Aves  & Multicóptero & DJI Phantom 2 & ZOOM H1 Handy Recorder  & Si & ? \\ 

\cite{bayram_active_2016}  &  No & Detección de collares VHF & ? & ? & Oso (Ursus)  & Multicóptero & DJI F550 hexarotor, Pixhawk autopilot & Telonics MOD-500 VHF, Uniden handheld scanner  & Si & ? \\ 

\cite{christie_unmanned_2016}  &  Si  & Estimación abundancia & USA &  Aleutian Islands & León marino de Steller (Eumetopias jubatus) & Multicóptero & APH- 22 hexacopter & ?  & Si & \$ 25.000 , \$ 3000 alquiler barco, or \$ 1700 por sitio \\ 

\cite{christie_unmanned_2016}  &  Si & Estimación abundancia & USA &  Monte Vista National Wildlife Refuge & Grus canadensis (sandhill cranes)  & Ala fija & Raven RQ- 11A & ?  & Si & \$ 400 \\ 

\multicolumn{11}{c}{} \\
\multicolumn{11}{c}{{\bf  MONITOREO DE ECOSISTEMAS TERRESTRES Y ACUÁTICOS  }} \\
\multicolumn{11}{c}{} \\

\cite{perroy_assessing_2017}  & No & Monitoreo de plantas invasoras & USA & Pahoa, Hawai & Miconia calvescens & Multicóptero & DJ Inspire-1 & DJI FC350 camera  & Si & ?  \\ 

\cite{szantoi_mapping_2017}  & Si & Mapeo de hábitat & Indonesia & Gunung Leuser National Park & Orangután (Pongo abelii)  & Ala fija & Skywalker & Canon S100  & Si & \$ 4000 \\ 
  
\cite{ivosevic_use_2015}  & Si & Monitoreo hábitats zonas restringidas; Modelos; Comprobar actitud  RPAS en diferentes condiciones ambientales & South Korea & Chiaksan National Park;Taeanhaean National Park &  ? & Multicóptero & DJI Phantom 2 Vision+  & full HD videos 1080p/30fps and 720p/60fps, cámara 14 megapixels 4384x3288 & Si & ? \\ 
  
\cite{lisein_discrimination_2015}  & No & Discriminación de especies de  hoja caduca, inventario forestal & Bélgica & Grand-Leez & English oak, birches (Betula pendula Roth. and Betula pubescens Ehrh.), sycamore maple (Acer pseudoplatanus L.), common ash (Fraxinus excelsior L.) and poplars (Populus spp.) & Ala fija & Gatewing X100  & Ricoh GR2 GR3 GR4 10 megapixels CCD  & Si & ?  \\ 
  
\cite{puttock_aerial_2015}  & Si & Caracterización ecosistemas afectados por la actividad del castor & UK & Devon Beaver Project site & Eurasian beaver (Castor fiber) & Multicóptero & 3D Robotics Y6 & Canon ELPH 520 HS  & Si & ?  \\ 
  
\cite{zahawi_using_2015}  & No & Caracterización estructura bosques tropicales para acciones de restauración & Costa Rica & Devon Beaver Project site & Varias especies & Multicóptero & 3D Robotics Y6 & Canon S100  & Si & \$ 1500 \\ 
  
\cite{bustamante_forest_2015}  & Si &  Monitoreo de bosques & Brasil & Riverine Forests (Permanent Protected Areas), Rio de Janeiro, Barrãcao do Mendes, Santa Cruz and São Lorenço & Bosques de rivera & Multicóptero & DJI Phantom Vision 2S   & RGB digital camera with 14 mega pixels & Si & \$ 9700  \\ 
  
\cite{gini_aerial_2012}  & Si & Modelamiento 3D, clasificación de especies arbóreas & Italy & Parco Adda Nord &  Varias especies &  Multicóptero  &  Microdrones TM MD4-200 & RGB CCD 12 megapixels Pentax Optio A40, modified NIR Sigma DP1 with a Foveon X3 sensor  & Si & ?  \\

\cite{miyamoto_use_2004}& Si & Clasificación de especies en humedales & Japón & Humedales de Kushiro &  Varias especies & Globo helio  & ? & NIKON F-801, NIKKOR 28 mm f/2.8  & Si & Helio \$ 600, globo \$ 1000  \\ 

\multicolumn{11}{c}{} \\
\multicolumn{11}{c}{{\bf EVALUACIÓN DE INFRAESTRUCTURAS Y RIESGO, VIGILANCIA, ECOTURISMO, IMPACTO EN LA FAUNA }} \\
\multicolumn{11}{c}{} \\

\cite{lobermeier_mitigating_2015} & No  & Mitigar el riesgo de colisión mediante la instalación de marcadores en líneas electríca & USA & ?  & Aves  & Multicóptero  & Mikrokopter Hexa XL  & KX 171 Microcam & ? & ? \\ 

\cite{pazmany_low_2014a} & Si  & Evaluación riesgo riesgo eléctrico de nidos en postes de alta tensión & España & Parque Nacional de Doñana &  Aves  & Ala fija  & Easy fly St-330 & GoPro Hero 2 11 MP, Panasonic LX3 11MP & Si & 7800 euros  \\ 

\cite{mulero-pazmany_remotely_2014}  & Si  & Vigilancia en áreas protegidas & Africa & KwaZulu-Nata & black rhinoceros (Diceros bicornis), white rhinoceros (Ceratotherium simum)  & Ala fija  & Easy Fly St-330 & Panasonic Lumix LX-3 11 MP, GoPro Hero2, Thermoteknix Micro CAM microbolometer & Si & 13750 euros  \\ 

\cite{hansen_applying_2016} & Si  & Monitoreo actividad visitantes  & Suecia & Kosterhavet National Park &  Humanos  & ?  & ? & ?  & ? & ?  \\ 

\cite{king_will_2014} & Si  & Aplicaciones RPAS en actividades ecoturismo & Suecia & Kosterhavet National Park &  Humanos  & ?  & ? & ?  & ? & ?  \\ 
  
\cite{vas_approaching_2015} & Si  & Impacto RPAS  especies aves lacustres  & Francia & e Zoo du Lunaret, Cros Martin Natural Area &  Anas platyrhyncho, Phoenicopterus roseus, Tringa nebularia  & Multicóptero &  Phantom & Hero3 GoPro & Si  & ?  \\ 

\cite{ditmer_bears_2015} & Si  & Impacto  RPAS oso negro americano   & USA & Kosterhavet National Park &  Oso negro americano (Ursus americanus) & Multicóptero & 3DR IRIS Pixhawk & GoPro HERO3+ & ?  & ?  \\ 

\bibliographystyle{plainnat}
\bibliography{master}
\end{tabular}

\end{table}
\end{landscape}

\subsection{Infrastructure and risk
assessment}\label{infrastructure-and-risk-assessment}

Other research projects highlight the utility of RPAS in assessing the
risk of human infrastructure for wildlife, which results in the
implementation of more effective preventive measures. Although not
exclusively addressed to protected areas, they could be of special
interest in buffer zones, where anthropic development may lead to
conflict with the surrounding fauna. For example, some species of birds
nest on high voltage power lines poles, making them especially
vulnerable to death by electrocution. (Margarita Mulero-Pázmány 2014,
Zhang et al. (2016)) use a fixed-wing RPAS for the visual evaluation of
linear electrical structures in which operation costs and flight time is
crucial. On the other hand, one of the most common causes of death in
birds is due to collisions with the wiring. (Lobermeier et al. 2015)
propose to install marks that are easily visible through the use of
robotics arms installed in multicopters. Due to the ease of maneuvering
of the platform, multicopters are more suitable for precision work.
Another possible use case is related to birds that nest in the soil,
especially in cereal crops. As a pre-harvest activity, generally
performed under mechanical procedures, (Mulero-Pázmány M. 2011) suggest
a flyby to identify possible nests, and if necessary, take the
appropriate actions to avoid their destruction.

\subsection{Monitoring and mapping of terrestrial and aquatic
ecosystems}\label{monitoring-and-mapping-of-terrestrial-and-aquatic-ecosystems}

During the last decades, the emergence of remote sensors on board air or
space platforms has led to an increase in applications for the study of
ecosystems (Wulder et al. 2004).The data obtained have enabled the
development of vegetation and soil maps, improvements in the
characterization of habitats, enhance the understanding of the structure
and function of forest ecosystems, develop digital elevation models or
geomorphological maps of application in the modeling of species
distribution. The emergence of RPAS has led to the quantitative analysis
of habitats at a level of detail that had not been possible previously,
either for economic reasons or for technological limitations. This
impulse has been especially notable with the parallel development of
multispectral and hyperspectral sensors adapted to small aircraft, whose
price is expected to decrease according to trends in the sector. The
United States Geological Survey (USGS) has conducted flights to classify
vegetation cover in wetlands (USGS 2014).Other studies monitor the
distribution of invasive species under different flight conditions and
vegetation cover (Perroy, Sullivan, and Stephenson 2017), while the
characterization of forest stands constitutes an important section given
the number of articles that approach the problem from different
perspectives. (Gini et al. 2012) employs a quadcopter model operated at
low-height and RGB and NIR cameras in small areas. Due to the reduced
reliability and autonomy of the platform and the difficulties to
increase the load capacity, the flight planning is reduced to three
passes with a percentage of 80\% and 30\% of longitudinal and transverse
overlap respectively. (Lisein et al. 2015) performs a multitemporal
analysis of the spectral response to phenological variations in
different species of deciduous trees and concludes that the
intraspecific spectral variation is of maximum interest for the
optimization of classification algorithms and discrimination between
species. In his research, he operates a fixed wing RPAS model, uses
different sensors in the visible and near infrared range, and optimizes
flight parameters in order to cover the maximum surface with the fewest
number of flights possible. (Zahawi et al. 2015) applies the Ecosynth
methodology, a toolkit for mapping and measuring 3D vegetation using
digital cameras and open source artificial vision software, in order to
evaluate the effectiveness of restoration actions in forests using RPAS
as a viable alternative for traditional field measurements and applying
different predictive models of the presence of frugivorous birds from
height and canopy structure data.

\subsection{Surveillance and support for compliance with laws in
protected
areas}\label{surveillance-and-support-for-compliance-with-laws-in-protected-areas}

RPAS have also relevance in the control and surveillance of protected
areas, documented through different experiences focused mainly on
controlling poaching. This type of study is characterized in giving
greater emphasis on improving first-person view methods (FPV) in order
to obtain a real-time view of the monitored area. Also, it is worth
mentioning the greater use of fixed-wing RPAS as provide longer flight
times, the convenience of using thermal cameras in low visibility
conditions, generally related to hours of greater furtive activity,
along with advances in computer vision systems programmed to detect the
presence of humans and target species under pressure from illegal trade
in protected areas (Mulero-Pázmány et al. 2014,) focus on the African
rhinoceros and note the advantages of real time video compared to still
photography, which require longer post-processing time. In addition,
authors emphasize the need to improve the resolution of thermal sensors
to increase the chances of detecting suspicious activity at night time.
(Duffy 2014) analyzes the consequences of the militarization of
conservation practices as an increasing trend in natural protected areas
around the world and illustrates the use of RPAS through several
examples. With respect to coastal zones, a quick search on Internet
allows collecting several initiatives that try to optimize the control
of illegal fishing through RPAS. However we have not been able to verify
scientific studies that endorse such initiatives, so it opens an
interesting field of research. To illustrate some examples, a pioneering
survey was conducted in Belize for fisheries monitoring using a fixed
wing model Skywalker. The Government of the Canary Islands is
considering the use of RPAS in hard to reach coastal areas to deal with
poaching (INFORCASA 2017). Finally \url{http://soarocean.org/} is an
initiative of National Geographic and Lindblad Expedition fostering the
use of low cost drones in the protection of the oceans and it looks a
good starting point to search for latest applications in this field.

\subsection{Ecotourism}\label{ecotourism}

The high degree of diversification offered by RPAS in the ecotourism
industry is summarized in a recent article, which shows possible
recreational activities, business opportunities, search and rescue
operations, mapping and formulas for granting RPAS flight permits in
designated areas (King 2014). DWithin the still scarce literature,
(Hansen 2016) values the effectiveness of RPAS in monitoring visitors in
marine and coastal areas, in combination with other innovative
solutions. According to the author the RPAS would theoretically allow to
operate under different environmental conditions, improving the level of
detail and offering a continuous coverage in the flow and behavior of
the visitors, as opposed to other techniques of habitual use like the
manual observation or the installation of networks of Surveillance
cameras.

\subsection{Impact of RPAS on
wildlife}\label{impact-of-rpas-on-wildlife}

Animal welfare should be present on wildlife monitoring using RPAS ,
establishing ethical principles that complement the current standards in
research and conservation. (Vas et al. 2015) obtained promising results
in the field of ornithology, assessing the impact of color, speed and
angle of flight on the behavioral responses of wetland birds to the
approach of multicopters. The latter factor is considered as the primary
trigger for changes in behavioral patterns, especially in vertical
approaches at an angle of 90º. Finally, a core set of recommendations is
included, and authors encourages to extend the trials to a wide range of
RPAS and species. (J. B. Vincent, Werden, and Ditmer 2015) measures
physiological stress in American black bear (Ursus americanus) by
electronic recording of cardiac activity in the presence of RPAS in the
presence of RPAS. Although no changes in behavior patterns are detected,
the increase in beats per minute (bmp) is significant in most cases
observed. In the absence of further experiences explicitly addressing
the phenomenon, (J. C. Hodgson and Koh 2016) conclude with a series of
general recommendations as the basis for a code of good practice,
highlighting the adoption of the precautionary principle and respect for
aviation standards, the specific training of operators, the appropriate
selection of equipment, the cessation of operations in the case of
obvious disturbances in the populations studied and the reporting of
observations in scientific publications, that allows sharing of
knowledge to progressively improve the protocols of operations with RPAS
that involve the observation of the wild fauna.

\section{Discussion}\label{discussion}

Most of the sources analyzed focus on local-scale conservation projects
and feasibility studies of RPAS in the characterization of wildlife
populations and communities, especially in distribution and abundance
studies. Literature begins to be equally prolific in monitoring and
mapping activities in terrestrial and aquatic ecosystems, a niche
currently occupied by aerial and space platforms for environmental
remote sensing. Despite low number of scientific articles addressing the
use of RPAS in the control and surveillance of natural protected areas,
it is still one of the issues that more social debate generates and it
is not strange to find governmental initiatives or promoted by
environmental organizations in the fight against poaching. From the
economic point of view, expenses derived from the operation with RPAS
are hardly quantifiable. Also, not all studies consider the effort
required for the development of technical and analytical skills of the
staff involved. The computational requirements are demanding and certain
phases of information processing requires the acquisition of computer
programs whose price is generally high. Also, operations with RPAS are
not exempt from accidents, which has an negative impact on the budget
originally planned.

\subsection{Wildlife Studies}\label{wildlife-studies-1}

Most fixed-wing RPAS studies focus on population counts, with promising
results in the macrofauna. It is still early to generalize its use in
smaller species and areas of high vegetation coverage, although the
development of LIDAR technology and wide-spectrum sensors could help
overcome technical barriers. In addition, it is necessary to improve the
knowledge regarding the planning of the sampling performed with RPAS, to
avoid errors in estimation. Multicopters could cover some of the
limitations mentioned above, but there still seem to be no studies
combining both systems. On the other hand, the use of RPAS can increase
the complexity of research, requiring highly skilled work teams and
computational needs that are not available to many institutions. In any
case, RPAS could become an essential tool for ecologist and its use
could be justified as long as there are no advances in other traditional
techniques supporting wildlife research.

\subsection{Infrastructure and risk
assessment}\label{infrastructure-and-risk-assessment-1}

RPAS have demonstrated their capacity for the technical inspection of
industrial premises. The high cost of wildlife risk assessment using
manual methods in areas of high incidence of deaths could persuade
environmental authorities to encourage their use for preventive
purposes. As previously discussed, relative low operational cost of RPAS
make them an attractive alternative, which may foster such activities.
RPAS could also prevent accidents by applying dissuasive measures to
prevent the collision of birds in wind farms. Other uses include the
revision of natural areas facilities, by scheduling periodic flights.
Also RPAS are positioned as fundamental tools in the prevention and
evaluation of forest fires.

\subsection{Monitoring and mapping of terrestrial and aquatic
ecosystems}\label{monitoring-and-mapping-of-terrestrial-and-aquatic-ecosystems-1}

The integration of the classical remote sensing elements developed
during the last decades in the scope of the RPAS open new possibilities
in the observation of environmental phenomena at multiple scales. The
high resolution of images will allow the discrimination of plant
communities at the species level, observe the evolution of ecosystems in
shorter periods of time or more accurately quantify the volume and
structure of canopy. Also it will allow attending to urgent needs of
mapping in areas affected by natural and anthropic disasters. The
ability of computer systems to process massive amount of information is
closely linked to such applications.

\subsection{Surveillance and support for compliance with laws in natural
protected
areas}\label{surveillance-and-support-for-compliance-with-laws-in-natural-protected-areas}

The integration of RPAS in the fight against poaching and illegal
fishing in protected areas faces important technical and legal
constraints. In the first point, the reviewed literature mention the
need to design more efficient live vision systems. The low autonomy of
RPAS is especially critical in large natural parks, limiting the area
under surveillance. The issues concerning atmospheric conditions have
not yet been completely resolved. However, it is expected that main
barriers will appear in the legislative and social sphere. In some
countries it is forbidden to fly beyond the visual range of the
operator, limiting the effectiveness of the inspection in real time. The
work of RPAS in the surveillance of protected areas is also questioned
because it may affects civil liberties, especially with regards to
privacy. Some detractors are skeptical about the ability of RPAS to
persuade offenders, who in many cases face situations of greatest need.
Probably the success of such initiatives requires a greater consensus
among the parties involved and the development of strategies that seek
to solve the causes of poaching.

\subsection{Ecotourism}\label{ecotourism-1}

A permissive regularization of the use of RPAS in ecotourism activities
in natural parks could lead to unpredictable situations. On the one
hand, the constant presence of propeller and engine noise, the sensation
of invasion or lack of privacy and the visual impact of RPAS on the
landscape could negatively affect the tourist experience. It could also
significantly interfere with the health of ecosystems. Awareness of the
abuse of RPAS for recording wildlife has resulted in a ban on flying for
recreational purposes in natural parks in the United States and other
parts of the world. Not to mention the real risk of accidents that could
lead, for example, to the contamination of water reserves, due to the
toxicity of the materials. The abandonment or loss of damaged RPAS could
also increase the risk of fires in sensitive areas due to the presence
of flammable components. It does not appear that pilot or feasibility
studies or opinion polls have been published that respond to the issues
raised and to the ethical and legal implications derived from their use.
Even if the leisure possibilities are wide and recognized, it would be
advisable to be cautious in the face of the demand of the ecotourism
industry to incorporate RPAS in their activities.

\subsection{Impact of RPAS on
wildlife}\label{impact-of-rpas-on-wildlife-1}

The review of the literature suggests that there are still certain
niches that need more attention from the community. The ethical
implications of RPAS in wildlife studies have not yet been adequately
weighed. For example, most studies only marginally address the presence
or absence of reactions in species in the vicinity of RPAS. We believe
that experiments aimed at quantifying physiological and behavioral
changes are insufficient and that despite the emergence of some
initiatives and a greater degree of awareness, it would be necessary to
improve our current knowledge in order to include a set of good
practices and recommendations aimed at a wider group of wildlife
species. Other key factors include the specific professionalisation of
operators and the investment in RPAS models optimized to reduce the
impact on wildlife and facilitate their observation. We could mention
for example the reduction of propeller and engine noise and the design
of non-polluting components.

\section{Conclusions}\label{conclusions}

The consolidation of the RPAS as management and research tools in
natural protected areas is closely linked to the technological
development of the elements associated with the platform and to the
establishment of measures that favorably regulate its use, increasing
opportunities in the sector and stimulating innovation in priority
conservation areas. There are continually improvements in navigation
control and flight autonomy, while we are witnessing the progressive
miniaturization and diversification of sensors along with advances in
the field of artificial intelligence. This rapidly expanding confluence
of factors encourages the emergence of new scenarios with ethical and
legal implications. Most governments have reacted by setting constraints
that could have a negative impact on the capacity to integrate RPAS into
the civilian sphere. Given this situation, it is difficult to foresee
the actions that each country will adopt from now on in an attempt to
harmonize the advantages and disadvantages of these systems, reason why
it is probable that the future of the RPAS in protected areas is
conditioned fundamentally by political and social factors.

\hypertarget{refs}{}
\hypertarget{ref-abd-elrahman_development_2005}{}
Abd-Elrahman, Amr, Leonard Pearlstine, and Franklin Percival. 2005.
``Development of Pattern Recognition Algorithm for Automatic Bird .''
\emph{Surv. L. Inf. Sci.} 65 (1): 37.

\hypertarget{ref-Aedron2017}{}
AEDRON, Asociación Española de Drones y Afines. 2017. ``\emph{Borrador
de La Nueva Normativa (Pendiente Aprobación Y Publicación)}.''

\hypertarget{ref-Nacion2013}{}
Agencia EFE, La Nación. 2013. ``\emph{CIDH Alerta Del Creciente Uso de
'Drones' En América Latina}.''

\hypertarget{ref-Andrews2014}{}
Andrews, C. 2014. ``\emph{Wildlife Monitoring: Should Uav Drones Be
Banned?}''

\hypertarget{ref-barnosky_has_2011}{}
Barnosky, Anthony D, Nicholas Matzke, Susumu Tomiya, Guinevere O U
Wogan, Brian Swartz, Tiago B Quental, Charles Marshall, et al. 2011.
``Has the Earth's Sixth Mass Extinction Already Arrived?'' \emph{Nature}
470 (7336): 51--57.
doi:\href{https://doi.org/10.1038/nature09678}{10.1038/nature09678}.

\hypertarget{ref-bayram_active_2016}{}
Bayram, Haluk, Krishna Doddapaneni, Nikolaos Stefas, and Volkan Isler.
2016. ``Active Localization of VHF Collared Animals with Aerial
Robots.'' \emph{Bayram et Al. 2016}, no. 13: 74--75.
doi:\href{https://doi.org/10.1109/COASE.2016.7743503}{10.1109/COASE.2016.7743503}.

\hypertarget{ref-bravo_espacios_2008}{}
Bravo, Xavier Lastra. 2008. ``LOS ESPACIOS NATURALES PROTEGIDOS.
Concepto, Evolución Y Situación Actual En España.'' 1--25.

\hypertarget{ref-chape_measuring_2005}{}
Chape, S, J Harrison, M Spalding, and I Lysenko. 2005. ``Measuring the
Extent and Effectiveness of Protected Areas as an Indicator for Meeting
Global Biodiversity Targets.'' \emph{Philos. Trans. R. Soc. Lond. B.
Biol. Sci.} 360 (February 2005): 443--455.
doi:\href{https://doi.org/10.1098/rstb.2004.1592}{10.1098/rstb.2004.1592}.

\hypertarget{ref-christie_unmanned_2016}{}
Christie, Katherine S., Sophie L. Gilbert, Casey L. Brown, Michael
Hatfield, and Leanne Hanson. 2016. ``Unmanned Aircraft Systems in
Wildlife Research: Current and Future Applications of a Transformative
Technology.'' \emph{Front. Ecol. Environ.} 14 (5): 241--251.
doi:\href{https://doi.org/10.1002/fee.1281}{10.1002/fee.1281}.

\hypertarget{ref-cliff_online_2015}{}
Cliff, Oliver M, Robert Fitch, Salah Sukkarieh, Debra L Saunders, and
Robert Heinsohn. 2015. ``Online Localization of Radio-Tagged Wildlife
with an Autonomous Aerial Robot System.'' \emph{Robot. Sci. Syst.}, no.
November 2016: 1--9.
doi:\href{https://doi.org/10.15607/RSS.2015.XI.042}{10.15607/RSS.2015.XI.042}.

\hypertarget{ref-Conabio2017}{}
Conabio. 2017. ``\emph{Canarias Usará Drones Para Controlar La Pesca
Furtiva Y Mejorar Su Inspección}.''

\hypertarget{ref-ditmer_bears_2015}{}
Ditmer, Mark A., John B. Vincent, Leland K. Werden, Jessie C. Tanner,
Timothy G. Laske, Paul A. Iaizzo, David L. Garshelis, and John R.
Fieberg. 2015. ``Bears Show a Physiological but Limited Behavioral
Response to Unmanned Aerial Vehicles.'' \emph{Curr. Biol.} 25 (17):
2278--2283.
doi:\href{https://doi.org/10.1016/j.cub.2015.07.024}{10.1016/j.cub.2015.07.024}.

\hypertarget{ref-duffy_waging_2014}{}
Duffy, Rosaleen. 2014. ``Waging a War to Save Biodiversity: The Rise of
Militarized Conservation.'' \emph{Int. Aff.} 90 (4): 819--834.
doi:\href{https://doi.org/10.1111/1468-2346.12142}{10.1111/1468-2346.12142}.

\hypertarget{ref-forum_toward_2008}{}
Forum, Policy. 2008. ``Toward a Global Biodiversity Observing System,''
no. April.

\hypertarget{ref-Georg2016}{}
Gerster/Panos, Georg. 2017. ``\emph{Project Uses Drones to Monitor
Coastal Erosion in Ghana}.''

\hypertarget{ref-gini_aerial_2012}{}
Gini, R., D. Passoni, L. Pinto, and G. Sona. 2012. ``Aerial Images From
an Uav System: 3D Modeling and Tree Species Classification in a Park
Area.'' \emph{ISPRS - Int. Arch. Photogramm. Remote Sens. Spat. Inf.
Sci.} XXXIX-B1 (September): 361--366.
doi:\href{https://doi.org/10.5194/isprsarchives-XXXIX-B1-361-2012}{10.5194/isprsarchives-XXXIX-B1-361-2012}.

\hypertarget{ref-hansen_applying_2016}{}
Hansen, Andreas Skriver. 2016. ``Applying Visitor Monitoring Methods in
Coastal and Marine Areas Some Learnings and Critical Reflections from
Sweden.'' \emph{Scand. J. Hosp. Tour.} 2250 (June): 1--18.
doi:\href{https://doi.org/10.1080/15022250.2016.1155481}{10.1080/15022250.2016.1155481}.

\hypertarget{ref-hardin_small-scale_2013}{}
Hardin, Perry J, and Ryan R Jensen. 2013. ``Small-Scale Unmanned Aerial
Vehicles in Environmental Remote Sensing: Challenges and
Opportunities,'' no. October 2014: 37--41.
doi:\href{https://doi.org/10.2747/1548-1603.48.1.99}{10.2747/1548-1603.48.1.99}.

\hypertarget{ref-hodgson_unmanned_2013}{}
Hodgson, Amanda, Natalie Kelly, and David Peel. 2013. ``Unmanned Aerial
Vehicles (UAVs) for Surveying Marine Fauna: A Dugong Case Study.''
\emph{PLoS One} 8 (11): 1--15.
doi:\href{https://doi.org/10.1371/journal.pone.0079556}{10.1371/journal.pone.0079556}.

\hypertarget{ref-hodgson_best_2016}{}
Hodgson, Jarrod C., and Lian Pin Koh. 2016. ``Best Practice for
Minimising Unmanned Aerial Vehicle Disturbance to Wildlife in Biological
Field Research.'' \emph{Curr. Biol.} 26 (10).
doi:\href{https://doi.org/10.1016/j.cub.2016.04.001}{10.1016/j.cub.2016.04.001}.

\hypertarget{ref-Canarias2017}{}
INFORCASA. 2017. ``\emph{Canarias Usará Drones Para Controlar La Pesca
Furtiva Y Mejorar Su Inspección}.''

\hypertarget{ref-jain_unmanned_2013}{}
Jain, Mukesh. 2013. ``Unmanned Aerial Survey of Elephants.'' \emph{PLoS
One}.
doi:\href{https://doi.org/10.1371//\%0020journal.pone.0054700}{10.1371/\textbackslash{}\%0020journal.pone.0054700}.

\hypertarget{ref-juffe-bignoli_protected_2014}{}
Juffe-Bignoli, Diego, Neil David Burgess, H Bingham, E M S Belle, M G De
Lima, M Deguignet, B Bertzky, et al. 2014. ``Protected Planet Report
2014.'' \emph{Cambridge, UK UNEP-WCMC}.

\hypertarget{ref-king_will_2014}{}
King, Lisa M. 2014. ``Will Drones Revolutionise Ecotourism?'' \emph{J.
Ecotourism} 13 (1): 85--92.
doi:\href{https://doi.org/10.1080/14724049.2014.948448}{10.1080/14724049.2014.948448}.

\hypertarget{ref-knight_embracing_2007}{}
Knight, Andrew T., and Richard M. Cowling. 2007. ``Embracing Opportunism
in the Selection of Priority Conservation Areas.'' \emph{Conserv. Biol.}
21 (4): 1124--1126.
doi:\href{https://doi.org/10.1111/j.1523-1739.2007.00690.x}{10.1111/j.1523-1739.2007.00690.x}.

\hypertarget{ref-koh_dawn_2012}{}
Koh, Lian Pin, and Serge A. Wich. 2012. ``Dawn of Drone Ecology:
Low-Cost Autonomous Aerial Vehicles for Conservation.'' \emph{Trop.
Conserv. Sci.} 5 (2): 121--132.
doi:\href{https://doi.org/WOS:000310846600002}{WOS:000310846600002}.

\hypertarget{ref-korner_autonomous_2010}{}
Körner, Fabian, Raphael Speck, Ali Haydar, and Salah Sukkarieh. 2010.
``Autonomous Airborne Wildlife Tracking Using Radio Signal Strength,''
107--112.

\hypertarget{ref-leonardo_miniature_2013}{}
Leonardo, Miguel, Austin Jensen, Calvin Coopmans, Mac McKee, and
YangQuan Chen. 2013. ``A Miniature Wildlife Tracking UAV Payload System
Using Acoustic Biotelemetry.'' \emph{Proc. ASME Int. Des. Eng. Tech.
Conf. Comput. Inf. Eng. Conf.}, no. July 2015.
doi:\href{https://doi.org/10.1115/DETC2013-13267}{10.1115/DETC2013-13267}.

\hypertarget{ref-lhoest_how_2015}{}
Lhoest, S., J. Linchant, S. Quevauvillers, C. Vermeulen, and P. Lejeune.
2015. ``How Many Hippos (Homhip): Algorithm for Automatic Counts of
Animals with Infra-Red Thermal Imagery from UAV.'' \emph{Int. Arch.
Photogramm. Remote Sens. Spat. Inf. Sci. - ISPRS Arch.} 40 (3W3):
355--362.
doi:\href{https://doi.org/10.5194/isprsarchives-XL-3-W3-355-2015}{10.5194/isprsarchives-XL-3-W3-355-2015}.

\hypertarget{ref-linchant_are_2015}{}
Linchant, Julie, Jonathan Lisein, Jean Semeki, Philippe Lejeune, and
Cédric Vermeulen. 2015. ``Are Unmanned Aircraft Systems (UASs) the
Future of Wildlife Monitoring? A Review of Accomplishments and
Challenges.'' \emph{Mamm. Rev.} 45 (4): 239--252.
doi:\href{https://doi.org/10.1111/mam.12046}{10.1111/mam.12046}.

\hypertarget{ref-lisein_discrimination_2015}{}
Lisein, Jonathan, Adrien Michez, Hugues Claessens, and Philippe Lejeune.
2015. ``Discrimination of Deciduous Tree Species from Time Series of
Unmanned Aerial System Imagery.'' \emph{PLoS One} 10 (11).
doi:\href{https://doi.org/10.1371/journal.pone.0141006}{10.1371/journal.pone.0141006}.

\hypertarget{ref-lobermeier_mitigating_2015}{}
Lobermeier, Scott, Matthew Moldenhauer, Christopher Peter, Luke
Slominski, Richard Tedesco, Marcus Meer, James Dwyer, Richard Harness,
and Andrew Stewart. 2015. ``Mitigating Avian Collision with Power Lines:
A Proof of Concept for Installation of Line Markers via Unmanned Aerial
Vehicle.'' \emph{J. Unmanned Veh. Syst.} 3 (4): 252--258.
doi:\href{https://doi.org/10.1139/juvs-2015-0009}{10.1139/juvs-2015-0009}.

\hypertarget{ref-margarita_mulero-pazmany_juan_jose_negro_low_2014}{}
Margarita Mulero-Pázmány, Miguel Ferrer, Juan José Negro. 2014. ``A Low
Cost Way for Assessing Bird Risk Hazards in Power Lines: Fixed-Wing
Small Unmanned Aircraft Systems'' 2.

\hypertarget{ref-mascia_protected_2011}{}
Mascia, Michael B, and Sharon Pailler. 2011. ``Protected Area
Downgrading, Downsizing, and Degazettement (PADDD) and Its Conservation
Implications'' 4 (Dowie 2009): 9--20.
doi:\href{https://doi.org/10.1111/j.1755-263X.2010.00147.x}{10.1111/j.1755-263X.2010.00147.x}.

\hypertarget{ref-mateo_modelos_2011}{}
Mateo, Rubén G., Ángel M. Felicísimo, and Jesús Muñoz. 2011. ``Modelos
de Distribución de Especies: Una Revisión Sintética.'' \emph{Rev. Chil.
Hist. Nat.}, 217--240.
doi:\href{https://doi.org/10.4067/S0716-078X2011000200008}{10.4067/S0716-078X2011000200008}.

\hypertarget{ref-mulero-pazmany_m_aeromab_2011}{}
Mulero-Pázmány M., Negro JJ. 2011. ``AEROMAB: ``Small UAS for Montagus
Harriers Circus Pygargus Nests Monitoring.'' \emph{``AEROMAB ``Small UAS
Montagus Harriers Circus Pygargus Nests Monit. RED UAS Intenational
Congr. Univ. Eng. Seville, Spain. December 2011.}

\hypertarget{ref-mulero-pazmany_unmanned_2015}{}
Mulero-Pázmány, Margarita, Jose Ángel Barasona, Pelayo Acevedo, Joaquín
Vicente, and Juan José Negro. 2015. ``Unmanned Aircraft Systems
Complement Biologging in Spatial Ecology Studies.'' \emph{Ecol. Evol.} 5
(21): 4808--4818.
doi:\href{https://doi.org/10.1002/ece3.1744}{10.1002/ece3.1744}.

\hypertarget{ref-mulero-pazmany_remotely_2014}{}
Mulero-Pázmány, Margarita, Roel Stolper, L. D. Van Essen, Juan J. Negro,
and Tyrell Sassen. 2014. ``Remotely Piloted Aircraft Systems as a
Rhinoceros Anti-Poaching Tool in Africa.'' \emph{PLoS One} 9 (1): 1--10.
doi:\href{https://doi.org/10.1371/journal.pone.0083873}{10.1371/journal.pone.0083873}.

\hypertarget{ref-nugraha_urgency_2016}{}
Nugraha, Ridha Aditya, Deepika Jeyakodi, and Thitipon Mahem. 2016.
``Urgency for Legal Framework on Drones : Lessons for Indonesia , India
, and Thailand.'' \emph{Indones. Law Rev.} 6 (2): 137--157.

\hypertarget{ref-pereira_essential_2013}{}
Pereira, Henrique Miguel, Simon Ferrier, Michele Walters, Gary N Geller,
Rob H G Jongman, Robert J Scholes, Michael W Bruford, et al. 2013.
``Essential Biodiversity Variables.'' \emph{Science (80-. ).} 339
(6117): 277--278.
doi:\href{https://doi.org/10.1126/science.1229931}{10.1126/science.1229931}.

\hypertarget{ref-perroy_assessing_2017}{}
Perroy, Ryan L., Timo Sullivan, and Nathan Stephenson. 2017. ``Assessing
the Impacts of Canopy Openness and Flight Parameters on Detecting a
Sub-Canopy Tropical Invasive Plant Using a Small Unmanned Aerial
System.'' \emph{ISPRS J. Photogramm. Remote Sens.} 125: 174--183.
doi:\href{https://doi.org/10.1016/j.isprsjprs.2017.01.018}{10.1016/j.isprsjprs.2017.01.018}.

\hypertarget{ref-rodriguez_eye_2012}{}
Rodríguez, Airam, Juan J. Negro, Mara Mulero, Carlos Rodríguez, Jesús
Hernández-Pliego, and Javier Bustamante. 2012. ``The Eye in the Sky:
Combined Use of Unmanned Aerial Systems and GPS Data Loggers for
Ecological Research and Conservation of Small Birds.'' \emph{PLoS One} 7
(12).
doi:\href{https://doi.org/10.1371/journal.pone.0050336}{10.1371/journal.pone.0050336}.

\hypertarget{ref-sarda-palomera_fine-scale_2012}{}
Sardà-Palomera, Francesc, Gerard Bota, Carlos Viñolo, Oriol Pallarés,
Víctor Sazatornil, Lluís Brotons, Spartacus Gomáriz, and Francesc Sardà.
2012. ``Fine-Scale Bird Monitoring from Light Unmanned Aircraft
Systems.'' \emph{Ibis (Lond. 1859).} 154 (1): 177--183.
doi:\href{https://doi.org/10.1111/j.1474-919X.2011.01177.x}{10.1111/j.1474-919X.2011.01177.x}.

\hypertarget{ref-sasse_job-related_2003}{}
Sasse, D. Blake. 2003. ``Job-Related Mortality of Wildlife Workers in
the United States, 1937-2000.'' \emph{Wildl. Soc. Bull.} 31 (4):
1000--1003.

\hypertarget{ref-tomlins_remotely_1983}{}
Tomlins, G.F., and Y.J. Lee. 1983. ``Remotely Piloted Aircraft an
Inexpensive Option for Large-Scale Aerial Photography in Forestry
Applications.'' \emph{Can. J. Remote Sens.} 9 (2): 76--85.
doi:\href{https://doi.org/10.1080/07038992.1983.10855042}{10.1080/07038992.1983.10855042}.

\hypertarget{ref-USGS2014}{}
USGS. 2014. ``\emph{US Geological Survey National Unmanned Aircraft
Systems Project}.''

\hypertarget{ref-van_andel_locating_2015}{}
van Andel, Alexander C., Serge A. Wich, Christophe Boesch, Lian Pin Koh,
Martha M. Robbins, Joseph Kelly, and Hjalmar S. Kuehl. 2015. ``Locating
Chimpanzee Nests and Identifying Fruiting Trees with an Unmanned Aerial
Vehicle.'' \emph{Am. J. Primatol.} 77 (10): 1122--1134.
doi:\href{https://doi.org/10.1002/ajp.22446}{10.1002/ajp.22446}.

\hypertarget{ref-van_gemert_nature_2015}{}
van Gemert, Jan C., Camiel R. Verschoor, Pascal Mettes, Kitso Epema,
Lian Pin Koh, and Serge Wich. 2015. ``Nature Conservation Drones for
Automatic Localization and Counting of Animals.'' \emph{Lect. Notes
Comput. Sci. (Including Subser. Lect. Notes Artif. Intell. Lect. Notes
Bioinformatics)} 8925: 255--270.
doi:\href{https://doi.org/10.1007/978-3-319-16178-5_17}{10.1007/978-3-319-16178-5\_17}.

\hypertarget{ref-vas_approaching_2015}{}
Vas, E., A. Lescroel, O. Duriez, G. Boguszewski, and D. Gremillet. 2015.
``Approaching Birds with Drones: First Experiments and Ethical
Guidelines.'' \emph{Biol. Lett.} 11 (2): 20140754--20140754.
doi:\href{https://doi.org/10.1098/rsbl.2014.0754}{10.1098/rsbl.2014.0754}.

\hypertarget{ref-vincent_barriers_2015}{}
Vincent, John B, Leland K Werden, and Mark A Ditmer. 2015. ``Barriers to
Adding UAVs to the Ecologist's Toolbox.'' \emph{Front. Ecol. Environ.}
13 (2): 74--75.
doi:\href{https://doi.org/10.1890/15.WB.002}{10.1890/15.WB.002}.

\hypertarget{ref-weissensteiner_low-budget_2015}{}
Weissensteiner, M H, J W Poelstra, and J B W Wolf. 2015. ``Low-Budget
Ready-to-Fly Unmanned Aerial Vehicles: An Effective Tool for Evaluating
the Nesting Status of Canopy-Breeding Bird Species.'' \emph{J. Avian
Biol.} 46 (4): 425--430.
doi:\href{https://doi.org/10.1111/jav.00619}{10.1111/jav.00619}.

\hypertarget{ref-WichS2017}{}
Wich, S. 2017. ``\emph{Amazon River Dolphin Project}.''

\hypertarget{ref-wulder_high_2004}{}
Wulder, Michael A, Ronald J Hall, Nicholas C Coops, and Steven E
Franklin. 2004. ``High Spatial Resolution Remotely Sensed Data for
Ecosystem Characterization'' 54 (6): 511--521.
doi:\href{https://doi.org/10.1641/0006-3568(2004)054}{10.1641/0006-3568(2004)054}.

\hypertarget{ref-wwf_living_2016}{}
WWF. 2016. \emph{Living Planet Report 2016. Risk and Resilience in a New
Era.}

\hypertarget{ref-zahawi_using_2015}{}
Zahawi, Rakan A., Jonathan P. Dandois, Karen D. Holl, Dana Nadwodny, J.
Leighton Reid, and Erle C. Ellis. 2015. ``Using Lightweight Unmanned
Aerial Vehicles to Monitor Tropical Forest Recovery.'' \emph{Biol.
Conserv.} 186 (June): 287--295.
doi:\href{https://doi.org/10.1016/j.biocon.2015.03.031}{10.1016/j.biocon.2015.03.031}.

\hypertarget{ref-zhang_uav_2016}{}
Zhang, Yong, Xiuxiao Yuan, Yi Fang, and Shiyu Chen. 2016. ``UAV Low
Altitude Photogrammetry for Power Line Inspection,'' no. August.
doi:\href{https://doi.org/10.20944/preprints201608.0048.v1}{10.20944/preprints201608.0048.v1}.

\newpage
\singlespacing 
\end{document}
