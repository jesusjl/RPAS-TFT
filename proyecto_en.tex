% interactcadsample.tex mod using svm.latex.ms.texs
% v1.03 - April 2017
\documentclass[]{interact}
%\documentclass[a4paper]{article}
\usepackage{epstopdf}% To incorporate .eps illustrations using PDFLaTeX, etc.
\usepackage{subfigure}% Support for small, `sub' figures and tables
%\usepackage[nolists,tablesfirst]{endfloat}% To `separate' figures and tables from text if required
% 
\usepackage{natbib}% Citation support using natbib.sty
\bibpunct[, ]{(}{)}{;}{a}{}{,}% Citation support using natbib.sty
\renewcommand\bibfont{\fontsize{10}{12}\selectfont}% Bibliography support using natbib.sty

\theoremstyle{plain}% Theorem-like structures provided by amsthm.sty
\newtheorem{theorem}{Theorem}[section]
\newtheorem{lemma}[theorem]{Lemma}
\newtheorem{corollary}[theorem]{Corollary}
\newtheorem{proposition}[theorem]{Proposition}

\theoremstyle{definition}
\newtheorem{definition}[theorem]{Definition}
\newtheorem{example}[theorem]{Example}

\theoremstyle{remark}
\newtheorem{remark}{Remark}
\newtheorem{notation}{Notation}


\usepackage{rotating}
% \usepackage[]{geometry}


\usepackage{layout}
\setlength{\headsep}{15pt}
\usepackage{booktabs}
\usepackage{rotating}
\usepackage{graphicx}
\usepackage{setspace}
\usepackage{pdflscape}
\usepackage{longtable}
\usepackage{tabu}
\onehalfspacing
\usepackage{helvet}


\usepackage[T1]{fontenc}
\usepackage[utf8]{inputenc}
\usepackage[nottoc,notlof,notlot]{tocbibind} 

\usepackage{abstract}
\renewcommand{\abstractname}{}    % clear the title
\renewcommand{\absnamepos}{empty} % originally center

\renewenvironment{abstract}
 {{
    \setlength{\leftmargin}{10mm}
    \setlength{\rightmargin}{\leftmargin}%
  }
  \relax}
 {\endlist}


\newtheorem{hypothesis}{Hypothesis}
\usepackage{setspace}

\makeatletter
\@ifpackageloaded{hyperref}{}{%
\ifxetex
  \usepackage[setpagesize=false, % page size defined by xetex
              unicode=false, % unicode breaks when used with xetex
              xetex]{hyperref}
\else
  \usepackage[unicode=true]{hyperref}
\fi
}
\@ifpackageloaded{color}{
    \PassOptionsToPackage{usenames,dvipsnames}{color}
}{%
    \usepackage[usenames,dvipsnames]{color}
}
\makeatother
\hypersetup{breaklinks=true,
            bookmarks=true,
            pdfauthor={Jesús Jiménez López (University) and Margarita Mulero-Pázmány (University)},
             pdfkeywords = {RPAS, UAV, drones, natural protected areas, conservation, biodiversity},  
            pdftitle={Contribution of RPAS in research and conservation in natural protected
areas: present and future},
            colorlinks=true,
            citecolor=blue,
            urlcolor=blue,
            linkcolor=magenta,
            pdfborder={0 0 0}}
\urlstyle{same}  % don't use monospace font for urls


% notice vspace after title
\title{Contribution of RPAS in research and conservation in natural protected
areas: present and future\vspace{0.25in}  }


\author{\Large Jesús Jiménez López\vspace{0.05in} \newline\normalsize\emph{University }   \and \Large Margarita Mulero-Pázmány\vspace{0.05in} \newline\normalsize\emph{University }  }

\date{}

%%%%%%%%%%%%%%%%%%%%%%%%%%%%%%%%%%%%%%%%%%%%%%%%%%%%%%%%%%%%%%%%%%%%%%%%555

\begin{document}

\articletype{ARTICLE TEMPLATE}

% \pagenumbering{arabic}% resets `page` counter to 1 
%
\author{
\name{A.~N. Author\textsuperscript{a}\thanks{CONTACT A.~N. Author. Email: latex.helpdesk@tandf.co.uk} and John Smith\textsuperscript{b}}
\affil{\textsuperscript{a}Taylor \& Francis, 4 Park Square, Milton Park, Abingdon, UK; \textsuperscript{b}Institut f\"{u}r Informatik, Albert-Ludwigs-Universit\"{a}t, Freiburg, Germany}
}
% \author{\Large Jesús Jiménez López\vspace{0.05in} \newline\normalsize\emph{University}  \and \Large Margarita Mulero-Pázmány\vspace{0.05in} \newline\normalsize\emph{University} }
\title{Contribution of RPAS in research and conservation in natural protected
areas: present and future}
\maketitle  % title \par 
  




  \fontsize{9}{10}{\textbf{ABSTRACT}}: \par
  \begingroup
  \leftskip1em
  \rightskip\leftskip
  \noindent{\fontsize{9}{10}\selectfont Protected areas management has historically benefited from a wide range
of technological advances from remote sensing, camera traps or wildlife
tracking devices to a continuous development of analytical techniques to
cope with such amount of data collected. During the last decade, we have
witnesessed a growing interest in projects aimed to evaluate the
feasibility of RPAS for conservation purposes. So far, RPAS have been
tested or directly applied for a variety of research and management
activities including environmental and wildlife monitoring or
anti-poaching strategies. But there are technical, ethical, and legal
barriers that are currently limiting its effectiveness. However, as a
result of efforts to improve the features of these systems, followed by
a greater concern related to animal welfare and the parallel development
of novel surveys and statistical methods, their presence as an essential
tool for conservation in protected areas is increasingly justified.}
  \rightskip5em
  \par
  \endgroup
  \vspace{5mm}
  


    
 \fontsize{9}{10}{\textbf{KEYWORDS}}: \par
  \begingroup
  \leftskip1em
  \rightskip\leftskip
  \noindent{\fontsize{9}{10}\selectfont RPAS, UAV, drones, natural protected areas, conservation, biodiversity}
  \rightskip5em
  \par
  \endgroup
  \vspace{5mm}


% 

\vskip 6.5pt

\noindent  \section{Introduction}\label{introduction}

\subsection{Current context}\label{current-context}

Civil applications of remotely piloted aircraft systems (RPAS, also
known as unmanned aerial systems, UAS, drones) have been raised in an
increasingly number of scientific articles. During the last few years
there have been a significant amount of wildlife research projects in
natural protected areas using RPAS (J. Linchant et al. 2015; Chabot and
Bird 2015; Christie et al. 2016) and it is rapidly taking its place
within the broad set of technological solutions that support
conservation (Pimm et al. 2015). In most cases, feasibility studies were
carried out, assessing the capacity of RPAS in relation to traditional
conservation instruments by measuring the overall performance,
delimiting their strenghts and weakness, and establishing guidelines and
recommendations, resulting in new perspectives of application.

Although the potential of RPAS for mapping is tackled at the end of the
seventies ({\textbf{???}}), we found some references dating back to the
early 1980s, where first trials with RPAS on environmental issues began
with the objective of acquiring aerial photographs and demonstrating
their usefulness in forestry applications, the management of fish
resources or the coupling of sensors for atmospheric studies among
others (Tomlins and Lee 1983). Towards the end of the 20th century, the
first mapping surveys of vegetation in threatened species appeared
(Quilter, 1997), while with the arrival of the new millennium the number
of publications began to increase significantly (Hardin and Jensen
2013). At present there are some initiatives that seek to determine the
current state of the RPAS in the areas of ecology and conservation.
Recently, the journals \emph{Remote Sensing in Ecology and Conservation}
and the \emph{International Journal of Remote Sensing} made a call to
the scientific community for the sending of proposals in order to update
the current state of RPAS applied into the enviromental spere. As
result, a significant production of RPAS related papers on the matter is
expected. On the other hand, it is remarkable the greater presence of
web portals that center their activity around civil applications with
RPAS. In the field of research applied to conservation, the website
\url{http://conservationdrones.org/} is a worldwide reference, whose
contents illustrate recent pioneering projects, so they are not always
reflected in the scientific literature. The popularity of RPAS has
transcended the scientific-technical field, giving rise to the emergence
of user communities with a large presence on the Internet. One of the
most active portals is \url{http://diydrones.com/}, which brings
together fans of the do-it-yourself philosophy that encourages the use
of open platforms versus the traditional closed systems offered by the
traditional industry. This has unchained the reduction of costs of these
equipment and, together with the development of specialized open source
software, have led to the democratization of technology, bringing it
closer to a broad number of users and organizations. The scientific
community has probably benefited from this general trend. For some
authors, the flexibility in the assembly of RPAS offers in principle a
greater degree of customization, allowing to combine different sensors
and control systems according to the particular needs of each project
and within the research group itself (Koh and Wich 2012). In the
commercial field, more companies offer RPAS of high performance and
reliability along with professional services and software, so the sector
benefits from great dynamism.

\subsection{Protected areas}\label{protected-areas}

As defined by UICN, ``a protected area is a clearly defined geographical
space, recognised, dedicated and managed, through legal or other
effective means, to achieve the long term conservation of nature with
associated ecosystem services and cultural values'' (Dudley 2008).
Despite the fact that the number of protected areas has increased
considerably at a global level, with 15.4\% of the land area and 8.4\%
of the marine areas under some protection figure (Juffe-Bignoli et al.
2014) the size of wildlife populations has been estimated to have
decreased by 52\% in the period 1970 to 2012. Habitat fragmentation,
severe pollution particularly in freshwater ecosystems, overexploitation
of resources, environmental impacts of climate change and the impact of
invasive species on indigenous populations have been identified as the
main threats to biodiversity (Barnosky et al. 2011; Conabio 2017). To
address the current environmental crisis, the Convention on Biological
Diversity (CBD), as part of the United Nations Environment Program
(UNEP), established in Nagoya, Japan, a strategic plan for the period
2011-2020 which includes the so-called Aichi targets for biological
diversity. Among the goals raised is the increase in protected area
systems of special importance for biodiversity and ecosystem services
(target 11) following governance, equity, management, representativeness
and ecological connectivity criteria.

The Group on Earth Observations Biodiversity Observation Network (GEO
BON) as part of the Group on Earth Observations (GEOSS), has identified
a set of Essential Biodiversity Variables (Pereira et al. 2013) as key
components for the collection of environmental information to monitor
the global state of our ecosystems and support better decision-making on
biodiversity conservation (Forum 2008). As part of the large array of
observing systems monitoring biodiversity, RPAS can fill the gap at an
intermediate spatial scale, surpassing the financial and technological
constrains of remote sensing and ground / aerial manned vehicles based
surveys (Koh and Wich 2012; A. Rodríguez et al. 2012; Chabot and Bird
2015). First, while it is possible to acquire satellite images at low or
virtually zero cost (LandSat, MODIS, Sentinel, etc.), most of these
platforms operate on a global or regional scale. The limited spatial and
temporal resolution, along with the inconveniences of cloud presence,
especially noticeable in tropical areas, reduces the effectiveness of
remote sensing in the collection of data at fine-scale, according to the
requirements of ecological studies at the level of species, habitats or
populations (Wulder et al. 2004). Secondly, the large extent of these
protected areas significantly increases the costs of field work,
particularly in hadarzous and inaccessible areas. Finally, while manned
aerial vehicles offers an optimal alternative for covering much larger
areas, they suffer from excessively high operational costs and are also
subject to observer bias. In addition, air accidents are ranking as the
leading cause of death in wildlife specialists in the United States
(Sasse 2003). As a consequence, RPAS have been positioned as an
appropriate complement for conservation activities (Zahawi et al. 2015)
avoiding to a certain extent some of the above-mentioned drawbacks. In
developing countries, especially sensitive in terms of budgetary
allocations and technical capacities, monitoring and surveillance
programs are being successfully developed through the use of RPAS. For
example, by capturing aerial images in the Volta delta, Ghana, a team of
scientists measured the effects of climate change on coastal areas and
evaluates the effectiveness of prevention and restoration measures
against erosive processes (Gerster/Panos 2017).

\subsection{Legal barriers}\label{legal-barriers}

RPAS operations faces important legal barriers that undermine the true
potential in the civilian sphere (Stöcker et al. 2017). An overly
restrictive regulatory framework could limit the possibilities of use of
the RPAS in the field of conservation, which makes clear the urgent need
to harmonize the legislation . In the United States and in most of the
European countries consulted, interim legislation has been adopted
which, to a certain extent, equates the management of RPAS with that of
traditional aircraft. In general terms, the situation in Latin America
is uneven, however there is a general tendency to develop specific laws
to cope with the rise of the RPAS in both the civil and military sectors
(America 2017). Africa is one of the continents where the impact of
drones in conservation has had greater repercussions. However, in the
opinion of some conservationists, their use has not been without
problems, resulting in governments that have totally or partially
prohibited drone operations, arguing national security problems in
detriment of protection of natural areas (Andrews 2014). But RPAS have
also been generally welcomed in several developing countries in Asia,
where an array of related programs are being carried out (Nugraha,
Jeyakodi, and Mahem 2016). The uncertainty of the users along the world
has promoted the development of associations in order to advise on the
legal aspects to be taken into account during the operation, with the
International Association for Unmanned Vehicle Systems (AUVSI)
\url{http://www.auvsi.org} being the largest nonprofit organization in
the world dedicated to advancing the community of unmanned aerial
vehicles users.

\section{Methods}\label{methods}

To achieve the proposed objectives, a bibliographical review of
scientific articles, gray literature, postgraduate theses, websites and
specialized journals was carried out, following a similar line to other
studies (J. Linchant et al. 2015; Christie et al. 2016). The main tools
for selection of the cited bibliography include Google Schoolar,
Research Gate and Mendeley Desktop, while the use of Internet search
engines include other references outside the scientific scope. Key
search criteria for keywords included unmanned aerial vehicles in their
various meanings and acronyms (RPAS, UAV, drones, etc.), along with a
variety of terms referring to natural protected areas, primarily in
English. Last references revised was published on June, 2017.

The selected information was categorized according to the role played by
RPAS in direct or indirect relation to conservation in natural areas. It
is presented in tabular format, identifying where the study was
conducted, the expected accomplishments and technical specifications of
the aerial platform. After posing main results obtained, gaps are
indentified and possible scenarios for implementing RPAS as essential
tools to help achieve conservation plans in protected areas are
discussed, highlighting some trends and opportunities that apparently
have not yet been adequately exploited.

\section{Results}\label{results}

\subsection{Wildlife Monitoring and
Management}\label{wildlife-monitoring-and-management}

Manned aircraft have been traditional used to undertake a variety of
ecological surveys. As remarked by most papers reviewed such techniques
are risky, costly and despite several efforts to minimize error
estimation (Cook and Jacobson 1979) are subject to visibility bias since
a greater number of observers is required to guarantee an exhaustive
count of populations. RPAS have emerged as a feasible alternative to
surpass such drawbacks. Several studies addressed counting large
terrestrial mammals with positive contributions (Jain 2013, Lancia et
al. (2005), Mulero-Pázmány et al. (2015)). (Colefax, Butcher, and
Kelaher 2017) reviewed the potential of RPAS as surveys tools in species
relying on coastal and marine ecosystems . RPAS have as well been
applied to study population dynamics in bird colonies (Sardà-Palomera et
al. 2012), but also in the inspection and characterization of
inaccessible nesting sites using multicopters (Weissensteiner, Poelstra,
and Wolf 2015). In some cases, in order to overcome the barriers to
directly detect the species of interest, the studies focused on locating
and characterizing their breeding and nesting areas (Andel et al. 2015,
Szantoi et al. (2017), Andrew and Shephard (2017), Serge Wich et al.
(2016)).

One of the central themes in ecology is the development of surveys and
statistical models for estimating abundance and distribution of animals
in wild populations (Lancia et al. 2005; Mateo, Felicísimo, and Muñoz
2011). Such methods allow inferring the potential or suitable habitat of
organisms by collecting environmental information and species presence
data from different sources and techniques. Wildlife telemetry tracking
is one of the most common methods used to gather movement data.
(Mulero-Pázmány et al. 2015) compared the performance of RPAS as tools
for data collection against biologgers in cattle (\emph{Bos taurus}),
easily identifiable by high-resolution aerial images obtained by
photographic sensors on board. The authors obtained similar results
regarding the performance of the models, but they emphasized the
cost-benefit factor of RPAS as the main advantage. In general, the
relatively expensive purchase of electronic tracking devices limits
their availability for research purposes, reducing sample size. Added to
the risk of marking individuals under non-random criteria, the
robustness of the analysis can be seriously affected. However, main
advantage of wildlife telemetry is its ability to provide a large amount
of data for longer periods of time. Nevertheless, the authors pointed
out that both methodologies have the potential to complement each other
throughout all phases of the study. Other innovative techniques have
recently been illustrated in scientific papers evaluating the
feasibility of pairing radio locators in RPAS in the search for
individuals marked with VHF radio collars (Soriano, Caballero, and
Ollero 2009, Körner et al. (2010); Bayram et al. 2016; Cliff et al.
2015; Leonardo et al. 2013).

Given the large amount of information generated, it is not surprising
that novel software tools have been developed in the field of computer
vision and machine learning that allow the automatic detection,
recognition and counting of individuals captured in scenes acquired by
visible and thermal-infrared sensors, replacing otherwise time-consuming
manual tasks (Lhoest et al. 2015; Abd-Elrahman, Pearlstine, and Percival
2005; Gemert et al. 2015, Chabot and Francis (2016), Christiansen et al.
(2014)).

Outside the scientific literature, there are projects for monitoring
wildlife in both marine and terrestrial ecosystems, generally supported
by non-governmental organizations and research centers. Based on
information gathered at \url{https://conservationdrones.org} several
studies have been identified pursuing methods for registering
individuals in marine mammal populations, primates and macrofauna in
general, located in protected areas or frequently visited by wildlife
under some legal figure of threat. For instance, a work conducted in the
Amazon Basin in Brazil is testing the use of RPAS to improve the density
and abundance estimation of different species of dolphins, compared with
direct observation by specialists (S. Wich 2017). The main research aims
include the validation and harmonization of both methodologies and,
indirectly, evaluate the feasibility for its regular application in
monitoring projects with a similar purpose, taking into account the
cost-benefit of the execution.

\begin{landscape}
% \begin{sidewaystable}
% \caption{RPAS APPLICATIONS IN  NATURAL PROTECTED AREAS}
\tiny
\setlength\tabcolsep{1.5pt}

\begin{longtabu}{p{1cm}p{3cm}p{2cm}p{3cm}p{3cm}p{1cm}p{2cm}p{3cm}p{2cm}}

Study  & Aims & Country & Place & Species / Ecosystems & RPAS type &  RPAS model & Sensor & Costs \\ 

\hline

\multicolumn{9}{c}{} \\
\multicolumn{9}{c}{ {\bf ESTUDIOS DE FAUNA Y VIDA SILVESTRE}  } \\
\multicolumn{9}{c}{} \\

\cite{pazmany_mulero_unmanned_2015} & SDM comparatiive study & Spain & Doñana N.P. & Bos taurus  & Ala fija & Easy Fly plane, Ikarus autopilot, Eagletree GPS logger & Panasonic Lumix LX-3 11MP & 5700 euros \\ 

\cite{hodgson_unmanned_2017} & comparative survey RPAS / land based observation; abundance estimation  & Australia & North Stradbroke Island &  humback whales  & Fixed-wing &  ScanEagle & Nikon D90 12MP, Standard Definition Electro-Optical Camera & ?  \\ 

\cite{hodgson_unmanned_2013} & Dugongs detenction.  Test RPAS performance. Ideal h and res.  & Australia & Shark Bay Marine Park & Dugong & Fixed-wing &  ScanEagle & Nikon D90 12 megapixel digital SLR camera  & ?  \\ 

\cite{longmore_adapting_2017} & SofInfrared termic  species detecion software deveopment  & Englash & ? & Wildlife & Rotor-wing & 750mm carbon-folding Y6 multi-rotor APM 2 autopilot 3Drobotics & FLIR, Tau 2 LWIR Thermal Imaging Camera Core   & ?  \\ 

\cite{wilson_feasibility_2017}  & Monitoreo bioacústico con RPAS & USA & State Game Lands & Aves  & Multicóptero & DJI Phantom 2 & ZOOM H1 Handy Recorder  & ? \\ 

\cite{bayram_active_2016}  & Detección de collares VHF & ? & ? & Bears (Ursus)  & Rotor-wing & DJI F550 hexarotor, Pixhawk autopilot & Telonics MOD-500 VHF, Uniden handheld scanner  & ? \\ 

\cite{christie_unmanned_2016}   & Estimación abundancia & USA &  Aleutian Islands & León marino de Steller (Eumetopias jubatus) & Multicóptero & APH- 22 hexacopter & ?  & \$ 25.000 , \$ 3000 alquiler barco, or \$ 1700 por sitio \\ 

\cite{christie_unmanned_2016} & Abundace estimation & USA &  Monte Vista National Wildlife Refuge & Grus canadensis (sandhill cranes)  & Fixed-wing & Raven RQ- 11A & \$ 400 \\ 

\cite{koski_evaluation_2009}       &  ? & ? & ? &  ? & ?  & ? & ? & ?  \\ 

\cite{andrew_semi-automated_2017}  &  ? & ? & ? &  ? & ?  & ? & ? & ?   \\

\cite{martin_estimating_2012}      &  ? & ? & ? &  ? & ?  & ? & ? & ?   \\

\cite{colefax_potential_2017}      &  ? & ? & ? &  ? & ?  & ? & ? & ? \\

\multicolumn{9}{c}{} \\
\multicolumn{9}{c}{{\bf  MONITOREO DE ECOSISTEMAS TERRESTRES Y ACUÁTICOS  }} \\
\multicolumn{9}{c}{} \\

\cite{perroy_assessing_2017}  & Monitoreo de plantas invasoras & USA & Pahoa, Hawai & Miconia calvescens & Multicóptero & DJ Inspire-1 & DJI FC350 camera  & ?  \\ 

\cite{szantoi_mapping_2017}  & Mapeo de hábitat & Indonesia & Gunung Leuser National Park & Orangután (Pongo abelii)  & Ala fija & Skywalker & Canon S100  & \$ 4000 \\ 
  
\cite{ivosevic_use_2015}  & Monitoreo hábitats zonas restringidas; Modelos; RPAS performance test & South Korea & Chiaksan National Park;Taeanhaean National Park &  ? & Multicóptero & DJI Phantom 2 Vision+  & full HD videos 1080p/30fps and 720p/60fps, cámara 14 megapixels 4384x3288 & ? \\ 
  
\cite{lisein_discrimination_2015}  & Discriminación de especies de  hoja caduca, inventario forestal & Bélgica & Grand-Leez & English oak, birches, sycamore maple ,common ash and poplars & Fixed-wings & Gatewing X100  & Ricoh GR2 GR3 GR4 10 megapixels CCD  & ?  \\ 
  
\cite{puttock_aerial_2015}  & Caracterización ecosistemas afectados por la actividad del castor & UK & Devon Beaver Project site & Eurasian beaver (Castor fiber) & Multicóptero & 3D Robotics Y6 & Canon ELPH 520 HS  & ?  \\ 
  
\cite{zahawi_using_2015} & Caracterización estructura bosques tropicales para acciones de restauración & Costa Rica & Devon Beaver Project site & Varias especies & Multicóptero & 3D Robotics Y6 & Canon S100  & \$ 1500 \\ 
  
\cite{bustamante_forest_2015}  &  Monitoreo de bosques & Brasil & Riverine Forests (Permanent Protected Areas), Rio de Janeiro, Barrãcao do Mendes, Santa Cruz and São Lorenço & Bosques de rivera & Multicóptero & DJI Phantom Vision 2S   & RGB digital camera with 14 mega pixels & \$ 9700  \\ 
  
\cite{gini_aerial_2012}  & Modelamiento 3D, clasificación de especies arbóreas & Italy & Parco Adda Nord &  Varias especies &  Multicóptero  &  Microdrones TM MD4-200 & RGB CCD 12 megapixels Pentax Optio A40, modified NIR Sigma DP1 with a Foveon X3 sensor  & ?  \\

\cite{miyamoto_use_2004} & Clasificación de especies en humedales & Japón & Humedales de Kushiro &  Varias especies & Globo helio  & ? & NIKON F-801, NIKKOR 28 mm f/2.8  & Helio \$ 600, globo \$ 1000  \\ 

\cite{casella_mapping_2017}  &  ? & ? & ? &  ? & ?  & ? & ? & ?  \\

\multicolumn{9}{c}{} \\
\multicolumn{9}{c}{{\bf EVALUACIÓN DE INFRAESTRUCTURAS Y RIESGO, VIGILANCIA, ECOTURISMO, IMPACTO EN LA FAUNA }} \\
\multicolumn{9}{c}{} \\

\cite{lobermeier_mitigating_2015} & Mitigar el riesgo de colisión mediante la instalación de marcadores en líneas electríca & USA & ?  & Aves  & Multicóptero  & Mikrokopter Hexa XL  & KX 171 Microcam  & ? \\ 

\cite{pazmany_low_2014a}  & Evaluación riesgo riesgo eléctrico de nidos en postes de alta tensión & España & Parque Nacional de Doñana &  Aves  & Ala fija  & Easy fly St-330 & GoPro Hero 2 11 MP, Panasonic LX3 11MP & 7800 euros  \\ 

\cite{mulero-pazmany_remotely_2014}   & Vigilancia en áreas protegidas & Africa & KwaZulu-Nata & black rhinoceros (Diceros bicornis), white rhinoceros (Ceratotherium simum)  & Ala fija  & Easy Fly St-330 & Panasonic Lumix LX-3 11 MP, GoPro Hero2, Thermoteknix Micro CAM microbolometer & 13750 euros  \\ 

\cite{hansen_applying_2016}  & Monitoreo actividad visitantes  & Suecia & Kosterhavet National Park &  Humanos  & ?  & ? & ?   & ?  \\ 

\cite{king_will_2014}  & Aplicaciones RPAS en actividades ecoturismo & Suecia & Kosterhavet National Park &  Humanos  & ?  & ? & ?   & ?  \\ 
  
\cite{vas_approaching_2015}  & Impacto RPAS  especies aves lacustres  & Francia & e Zoo du Lunaret, Cros Martin Natural Area &  Anas platyrhyncho, Phoenicopterus roseus, Tringa nebularia  & Multicóptero &  Phantom & Hero3 GoPro  & ?  \\ 

\cite{ditmer_bears_2015}  & Impacto  RPAS oso negro americano   & USA & Kosterhavet National Park &  Oso negro americano (Ursus americanus) & Multicóptero & 3DR IRIS Pixhawk & GoPro HERO3+   & ?  \\ 

\multicolumn{9}{c}{} \\

\hline
\bibliographystyle{plainnat}
\bibliography{master}
\end{longtabu}
\end{landscape}


\subsection{Infrastructure and risk
assessment}\label{infrastructure-and-risk-assessment}

Other research projects highlight the convenience of RPAS in assessing
the risk that human infrastructure posed for wildlife, which results in
the implementation of more cost-effective preventive measures. For
instance, some species of birds nest on high voltage power lines poles,
making them especially vulnerable to death by electrocution. (Margarita
Mulero-Pázmány 2014) used a fixed-wing RPAS for the visual evaluation of
linear electrical structures in which low operational costs and enduring
flight time are crucial. On the other hand, one of the most common
causes of death in birds is due to collisions with the wiring.
(Lobermeier et al. 2015) proposed to install marks that are easily
visible through the use of robotics arms installed in multicopters. Due
to the ease of maneuvering of the platform, multicopters are more
suitable for precision work. Another possible use case is related to
birds nesting in the soil, especially in cereal crops. As a pre-harvest
activity, generally performed under mechanical procedures,
(Mulero-Pázmány Margarita 2011) suggested a flyby to identify possible
nests, and if necessary, take the appropriate actions to avoid their
destruction.

\subsection{Monitoring and mapping of terrestrial and aquatic
ecosystems}\label{monitoring-and-mapping-of-terrestrial-and-aquatic-ecosystems}

During the last decades, the emergence of remote sensors on board air or
space platforms has led to an increase in applications for the study of
ecosystems (Wulder et al. 2004). The data obtained have enabled the
development of vegetation and soil maps, enhance the characterization of
habitats or the understanding of the structure and function of forest
ecosystems, develop digital elevation models or geomorphological maps of
application in the modeling of species distribution. The emergence of
RPAS has led to the quantitative analysis of habitats at a level of
detail that had not been possible previously, either for economic
reasons or for technological limitations. This impulse has been
especially notable with the parallel development of multispectral and
hyperspectral sensors adapted to small aircraft, whose price is expected
to decrease according to trends in the sector. The United States
Geological Survey (USGS) has conducted flights to classify vegetation
cover in wetlands (USGS 2014). Other studies monitored the distribution
of invasive species under different flight conditions and vegetation
cover (Perroy, Sullivan, and Stephenson 2017), while the
characterization of forest stands constitutes an important section,
considering the number of articles facing the issue from different
perspectives. (Gini et al. 2012) employed a quadcopter model operated at
low-height and RGB and NIR cameras in small areas. Due to the reduced
reliability and autonomy of the platform and the difficulties to
increase the load capacity, the flight planning is reduced to three
passes with a percentage of 80\% and 30\% of longitudinal and transverse
overlap respectively. (Lisein et al. 2015) performs a multitemporal
analysis of the spectral response to phenological variations in
different species of deciduous trees and concluded that intraspecific
spectral variation is of maximum interest for the optimization of
classification algorithms and discrimination between species. During the
research, the authors operated a fixed wing RPAS model, used different
sensors in the visible and near infrared range and optimized the flight
parameters to cover the maximum surface area with the fewest possible
number of flights. (Zahawi et al. 2015) applies the Ecosynth
methodology, a toolkit for mapping and measuring 3D vegetation using
digital cameras and open source artificial vision software, in order to
evaluate the effectiveness of restoration actions in forests using RPAS
as a viable alternative for traditional field measurements and applying
different predictive models of the presence of frugivorous birds from
height and canopy structure data. Recently, shallow coastal habitats
were also mapped using cost-effective consumer grade RPAS (Casella et
al. 2017, Ventura et al. (2016)).

\subsection{Surveillance and support for compliance with laws in
protected
areas}\label{surveillance-and-support-for-compliance-with-laws-in-protected-areas}

RPAS have also relevance in the control and surveillance of protected
areas, documented through different experiences focused mainly on
controlling poaching. This type of study is characterized in giving
greater emphasis on improving first-person view methods (FPV) in order
to obtain a real-time view of the monitored area. Also, it is worth
mentioning the suitability of fixed-wing RPAS as provide longer flight
times to cover large areas, the convenience of using thermal cameras in
low visibility conditions, generally related to hours of greater furtive
activity, along with advances in computer vision systems programmed to
detect the presence of humans and target species under pressure from
illegal trade in protected areas. (Mulero-Pázmány et al. 2014) focused
on the African rhinoceros (\emph{Diceros bicornis}, \emph{Ceratotherium
simum}) and noted the advantages of real time video compared to still
photography, which despite the better overall quality, requires longer
post-processing time. In addition, authors emphasized the need to
improve the resolution of thermal sensors to increase the chances of
detecting suspicious activity at night time. (Franco et al. 2016)
suggested using RPAS to combat poaching and illegal fishing activities
in marine protected areas (MPA), claiming million-dollar economic losses
in the fishing sector. (Duffy 2014) analyzed the consequences of the
militarization of conservation practices as an increasing trend in
natural protected areas around the world and illustrates the use of RPAS
through several examples. With respect to coastal zones, a quick search
on Internet allows collecting several initiatives that try to optimize
the control of illegal fishing through RPAS. However we have not been
able to verify scientific studies that endorse such initiatives, so it
opens an interesting field of research. To illustrate some examples, a
pioneering survey was conducted in Belize for fisheries monitoring using
a fixed-wing model Skywalker. The Government of the Canary Islands is
considering the use of RPAS in hard to reach coastal areas to deal with
poaching (INFORCASA 2017). Finally \url{http://soarocean.org/} is an
initiative of National Geographic and Lindblad Expedition fostering the
use of low cost drones in the protection of the oceans and it looks a
good starting point to search for latest applications in this field.

\subsection{Ecotourism}\label{ecotourism}

The high degree of diversification offered by RPAS in the ecotourism
industry is summarized in a recent article, which shows possible
recreational activities, business opportunities, search and rescue
operations, mapping and formulas for granting RPAS flight permits in
designated areas (King 2014). Within the still scarce literature,
(Hansen 2016) values the effectiveness of RPAS in monitoring visitors in
marine and coastal areas, in combination with other innovative
solutions. According to the author the RPAS would theoretically allow to
operate under different environmental conditions, improving the level of
detail and offering a continuous coverage in the flow and behavior of
visitors, as opposed to other techniques of habitual use like the manual
observation or the installation of networks of surveillance cameras.

\subsection{Impact of RPAS on
wildlife}\label{impact-of-rpas-on-wildlife}

Animal welfare should be present on wildlife monitoring and ecological
research using RPAS , establishing ethical principles that complement
the current standards in research and conservation (Wilson and McMahon
2006). (Vas et al. 2015) analyzed the response of birds to RPAS,
assessing the impact of color, speed and angle of flight on the
behavioral responses of wetland birds to the approach of multicopters.
The latter factor is considered as the primary trigger for changes in
behavioral patterns, especially in vertical approaches at an angle of
90º. Finally, a core set of recommendations is included, and authors
encouraged to extend the trials to a wide range of RPAS and species.
(McEvoy, Hall, and McDonald 2016) accomplished the most intensive
disturbance assesment on waterfowl to date, by combining an array of
fixed wing and multirotor RPAS at various altitudes. ({\textbf{???}})
quantified noise detection by several representative species, suggesting
flying higher than 200 meters to minimise noise disturbance. (Ditmer et
al. 2015) measured physiological stress in American black bear
(\emph{Ursus americanus}) by electronic recording of cardiac activity in
the presence of RPAS. Although no changes in behavior patterns are
detected, the increase in beats per minute (bmp) is significant in most
cases observed. (Pomeroy, O'Connor, and Davies 2015) noted evidence of
variation in reactivity in seal populations based on a variety of
factors, from the RPAS platform, height and lateral distance to the
breeding or moulting season. No adverse reactions have been reported in
elephants (\emph{Loxodonta africana}) or cattle (\emph{Bos taurus}) on
flights at a minimum height of 100 meters (Jain 2013, Mulero-Pázmány et
al. (2015)), while could substitute more intrusive techniques when
inspecting the status of nesting sites (Weissensteiner, Poelstra, and
Wolf 2015). In the absence of further experiences to date, explicitly
addressing the phenomenon, (Hodgson and Koh 2016) suggested a series of
general recommendations as the basis for a code of good practice,
highlighting the adoption of the precautionary principle and respect for
aviation standards, the specific training of operators, the appropriate
selection of equipment, the cessation of operations in the case of
obvious disturbances in the populations studied and the reporting of
observations in scientific publications, that allows sharing of
knowledge to progressively improve the protocols of operations with RPAS
that involve the observation of the wild fauna.

\section{Environmental monitoring and decision
support}\label{environmental-monitoring-and-decision-support}

As service providers, protected areas are inherently subject to face
with periodic and unforeseen tasks, quality control procedures. RPAS
potential to easer decision making

(McCaldin, Johnston, and Rieker 2015)

(Zaman, Jensen, and McKee 2011)

({\textbf{???}})

(Fornace et al. 2014)

\section{Discussion}\label{discussion}

Most of the sources analyzed focus on local-scale conservation projects
and feasibility studies of RPAS in the characterization of distribution
and abundance of wildlife populations. Literature begins to be equally
prolific in monitoring and mapping activities in terrestrial and aquatic
ecosystems, a niche currently occupied by aerial and space platforms for
environmental remote sensing. Despite the low number of scientific
articles addressing the use of RPAS in the control and surveillance of
natural protected areas, it is still one of the issues that more social
debate generates and it is not strange to find governmental initiatives
or promoted by environmental organizations in the fight against
poaching. From the economic point of view, expenses derived from the
operation with RPAS are hardly quantifiable. Also, not all studies
consider the effort required for the development of technical and
analytical skills of the staff involved. The computational requirements
are demanding and certain phases of information processing requires the
acquisition of software whose price is generally high. Also, operations
with RPAS are not exempt from accidents, which has an negative impact on
the budget originally planned.

\subsection{Wildlife Monitoring and
Management}\label{wildlife-monitoring-and-management-1}

Most fixed-wing RPAS studies focus on population counts, obtaining
promising results in macrofauna. It is still early to generalize its use
in smaller species and areas of high vegetation coverage, although the
development of LIDAR technology and wide-spectrum sensors could help to
overcome technical barriers. If succesfully implemented, parks managers
could benefit from RPAS when estimating spatial distribution of foraging
domestic herbivores impacting food availability in natural areas
(Mulero-Pázmány et al. 2015). Also, periodic monitoring campaigns
usually undertaking by rangers could be overly simplified by RPAS
mapping capabilities. As a downsize, the use of RPAS can increase the
complexity of research and management, requiring highly skilled work
teams and computational resources not available to many institutions. In
addition, the lack of statistical methods to tackle the analisys of
data. Also further efforts should be made in order to refine the
planning of sampling performed with RPAS, to avoid errors in estimation.
Multicopters could cover some of the limitations mentioned above, but
there still seem to be a scarcity of studies combining both systems. In
any case, RPAS could become an essential tool for ecologist and its use
could be justified as long as there are no advances in other traditional
techniques supporting wildlife research.

\subsection{Infrastructure and risk
assessment}\label{infrastructure-and-risk-assessment-1}

RPAS have demonstrated their capacity for the technical inspection of
industrial premises {[}@{]}. They could be also of special interest in
buffer zones, where anthropic development may lead to conflict with the
surrounding fauna. Wildlife risk assessment may benefits from such
methods, promoting their use for preventive purposes in areas of high
incidence of deaths where otherwise high cost manual inspection where
applied. As previously discussed, relative low operational cost of RPAS
make them an attractive alternative, which may foster such activities.
RPAS could also prevent accidents by applying dissuasive measures to
avoid the collision of birds in wind farms. Other uses include the
revision of natural areas facilities, by scheduling periodic flights.
Also RPAS are positioned as fundamental tools in the prevention and
evaluation of forest fires and it could assist in environmental impact
assessment in sensitive areas.

\subsection{Monitoring and mapping of terrestrial and aquatic
ecosystems}\label{monitoring-and-mapping-of-terrestrial-and-aquatic-ecosystems-1}

The integration of the classical remote sensing elements developed
during the last decades in the scope of the RPAS open new possibilities
in the observation of environmental phenomena at multiple scales. The
high resolution of images will allow the discrimination of plant
communities at the species level, observe the evolution of ecosystems in
shorter periods of time or more accurately quantify the volume and
structure of canopy. Also it will allow attending to urgent needs of
mapping in areas affected by natural and anthropic disasters. The
ability of computer systems to process massive amount of information is
closely linked to such applications.

\subsection{Surveillance and support for compliance with laws in natural
protected
areas}\label{surveillance-and-support-for-compliance-with-laws-in-natural-protected-areas}

The integration of RPAS in the fight against poaching and illegal
fishing in protected areas faces important technical and legal
constraints. In the first point, the reviewed literature mention the
need to design more efficient live vision systems. The low autonomy of
RPAS is especially critical in large natural parks, limiting the area
under surveillance. The issues concerning atmospheric conditions have
not yet been completely resolved. (Banzi 2014) proposed a sensor based
economical feasible anti-poaching alternative, arguing that RPAS
fulfilling the suitable specifications are costly, especially in
developing countries. However, as technology becomes more accesible, it
is expected that main barriers will appear in the legislative and social
sphere. In some countries it is forbidden to fly beyond the visual range
of the operator, limiting the effectiveness of the inspection in real
time. RPAS applied to surveillance of protected areas is also questioned
arguing human right breaching (Banzi 2014). Some detractors are
skeptical about the ability of RPAS to persuade offenders, who in many
cases face situations of greatest need. Probably the success of such
initiatives requires a greater consensus among the parties involved and
the development of strategies that seek to solve the causes of poaching.
Surveillance of illegal logging activities or bonfires in unauthorized
areas have great potential and may be easier to implement.

\subsection{Ecotourism}\label{ecotourism-1}

A permissive regularization of the use of RPAS in ecotourism activities
in natural parks could lead to unpredictable situations. On the one
hand, the constant presence of propeller and engine noise, the sensation
of invasion or lack of privacy and the visual impact of RPAS on the
landscape could negatively affect the tourist experience. It remains to
be seen whether it could significantly alter the state of ecosystems.
Awareness of the abuse of RPAS for recording wildlife has resulted in a
ban on flying for recreational purposes in natural parks in the United
States and other parts of the world. As result of potential enviromental
impact due to the use of RPAS by tourist in Antarctica, (Leary 2017)
reported the partial prohibition of recreational RPAS in coastal areas
as part of a more extensive regulation promoted by stakeholders. Such
regulation could be the way forward for other protected areas to adapt
the allowed activities with RPAS. It seems obvious to think that in the
hands of non-professionals, the risk of accidents and losses would
increase. This may lead to disturbing widlife, contamination of water
supplies or triggering fires in sensitive areas due to the presence of
flammable components. It does not appear that feasibility studies or
opinion polls have been published that respond to the issues raised and
to the ethical and legal implications derived from their use. Even when
the leisure possibilities are wide and recognized, it would be advisable
to be cautious in the face of the demand of the ecotourism industry to
incorporate RPAS in their activities.

\subsection{Impact of RPAS on
wildlife}\label{impact-of-rpas-on-wildlife-1}

The review of the literature suggests that there are still certain
niches that need more attention from the research community. The ethical
implications of RPAS in wildlife studies have not yet been adequately
weighed since most studies only marginally address the presence or
absence of reactions in species in the vicinity of RPAS. Despite the
greater degree of awareness, we consider that further trials aimed at
quantifying physiological and behavioral changes should be carried out.
A set of best practices and recommendations targeting a wider group of
wild species could increase the chances of integrating the responsible
use of RPAS in conservation and management activities in natural parks.
Moreover, some authors mentioned the lack of commercial operators with
sufficient expertise to carry out such activities (McEvoy, Hall, and
McDonald 2016). Also, an optimal trade-off between benefits and
environmental costs should be pursued (Grémillet et al. 2012; Sepúlveda
et al. 2010). By designing quieter, non-polluting and safer components,
the impact on wildlife could be reduced and its objective observation
facilitated. Nonetheless we consider that, as far as further testing be
done, RPAS has great potential to replace more invasive monitoring
techniques, whose reliability is challenged by the potential to induce
conditions of unacceptable stress in wildlife that could ultimately
invalidate the results of the research (Z. Jewell 2013, Wilson and
McMahon (2006)). This should be taken into account by managers of
protected area when reluctant to allow RPAS to be essential part of
research and conservation activities.

\section{Conclusions}\label{conclusions}

The consolidation of the RPAS as management and research tools in
natural protected areas is closely linked to the technological
development of the elements associated with the platform and to the
establishment of measures that favorably regulate its use, increasing
opportunities in the sector and stimulating innovation in priority
conservation areas. There are continually improvements in navigation
control and flight autonomy, while we are witnessing the progressive
miniaturization and diversification of sensors along with advances in
the field of artificial intelligence. This rapidly expanding confluence
of factors encourages the emergence of new scenarios with ethical and
legal implications. Most governments have reacted by setting constraints
that could have a negative impact on the capacity to integrate RPAS into
the civilian sphere, despite some progress in this regard. As result, it
is difficult to foresee the actions that each country will adopt from
now on in an attempt to harmonize the contradictions presented by RPAS,
reason why it is probable that the future of the RPAS in protected areas
is conditioned fundamentally by political and social factors.

\section*{References}\label{references}
\addcontentsline{toc}{section}{References}

\hypertarget{refs}{}
\hypertarget{ref-abd-elrahman_development_2005}{}
Abd-Elrahman, Amr, Leonard Pearlstine, and Franklin Percival. 2005.
``Development of Pattern Recognition Algorithm for Automatic Bird .''
\emph{Surv. L. Inf. Sci.} 65 (1): 37.

\hypertarget{ref-Bworld_of_drones2017}{}
America, New. 2017. ``\emph{World of Drones}.''

\hypertarget{ref-van_andel_locating_2015}{}
Andel, Alexander C. van, Serge A. Wich, Christophe Boesch, Lian Pin Koh,
Martha M. Robbins, Joseph Kelly, and Hjalmar S. Kuehl. 2015. ``Locating
Chimpanzee Nests and Identifying Fruiting Trees with an Unmanned Aerial
Vehicle.'' \emph{Am. J. Primatol.} 77 (10): 1122--1134.
doi:\href{https://doi.org/10.1002/ajp.22446}{10.1002/ajp.22446}.

\hypertarget{ref-andrew_semi-automated_2017}{}
Andrew, Margaret E, and Jill M Shephard. 2017. ``Semi-Automated
Detection of Eagle Nests: An Application of Very High-Resolution Image
Data and Advanced Image Analyses to Wildlife Surveys.''
doi:\href{https://doi.org/10.1002/rse2.38}{10.1002/rse2.38}.

\hypertarget{ref-Andrews2014}{}
Andrews, C. 2014. ``\emph{Wildlife Monitoring: Should Uav Drones Be
Banned?}''

\hypertarget{ref-banzi_sensor_2014}{}
Banzi, Jamali Firmat. 2014. ``A Sensor Based Anti-Poaching System in
Tanzania National Parks.'' \emph{International Journal of Scientific and
Research Publications} 4 (4).

\hypertarget{ref-barnosky_has_2011}{}
Barnosky, Anthony D, Nicholas Matzke, Susumu Tomiya, Guinevere O U
Wogan, Brian Swartz, Tiago B Quental, Charles Marshall, et al. 2011.
``Has the Earth's Sixth Mass Extinction Already Arrived?'' \emph{Nature}
470 (7336): 51--57.
doi:\href{https://doi.org/10.1038/nature09678}{10.1038/nature09678}.

\hypertarget{ref-bayram_active_2016}{}
Bayram, Haluk, Krishna Doddapaneni, Nikolaos Stefas, and Volkan Isler.
2016. ``Active Localization of VHF Collared Animals with Aerial
Robots,'' no. 13: 74--75.
doi:\href{https://doi.org/10.1109/COASE.2016.7743503}{10.1109/COASE.2016.7743503}.

\hypertarget{ref-casella_mapping_2017}{}
Casella, Elisa, Antoine Collin, Daniel Harris, Sebastian Ferse, Sonia
Bejarano, Valeriano Parravicini, James L. Hench, and Alessio Rovere.
2017. ``Mapping Coral Reefs Using Consumer-Grade Drones and Structure
from Motion Photogrammetry Techniques.'' \emph{Coral Reefs} 36 (1):
269--275.
doi:\href{https://doi.org/10.1007/s00338-016-1522-0}{10.1007/s00338-016-1522-0}.

\hypertarget{ref-chabot_wildlife_2015}{}
Chabot, Dominique, and David M. Bird. 2015. ``Wildlife Research and
Management Methods in the 21st Century: Where Do Unmanned Aircraft Fit
in?'' \emph{J. Unmanned Veh. Syst.} 3 (4): 137--155.
doi:\href{https://doi.org/10.1139/juvs-2015-0021}{10.1139/juvs-2015-0021}.

\hypertarget{ref-chabot_computer-automated_2016}{}
Chabot, Dominique, and Charles M. Francis. 2016. ``Computer-Automated
Bird Detection and Counts in High-Resolution Aerial Images: A Review.''
\emph{Journal of Field Ornithology} 87 (4): 343--359.
doi:\href{https://doi.org/10.1111/jofo.12171}{10.1111/jofo.12171}.

\hypertarget{ref-christiansen_automated_2014}{}
Christiansen, Peter, Kim A rild Steen, Rasmus N yholm Jørgensen, and
Henrik Karstoft. 2014. ``Automated Detection and Recognition of Wildlife
Using Thermal Cameras.'' \emph{Sensors (Basel).} 14 (8): 13778--13793.
doi:\href{https://doi.org/10.3390/s140813778}{10.3390/s140813778}.

\hypertarget{ref-christie_unmanned_2016}{}
Christie, Katherine S., Sophie L. Gilbert, Casey L. Brown, Michael
Hatfield, and Leanne Hanson. 2016. ``Unmanned Aircraft Systems in
Wildlife Research: Current and Future Applications of a Transformative
Technology.'' \emph{Front. Ecol. Environ.} 14 (5): 241--251.
doi:\href{https://doi.org/10.1002/fee.1281}{10.1002/fee.1281}.

\hypertarget{ref-cliff_online_2015}{}
Cliff, Oliver M, Robert Fitch, Salah Sukkarieh, Debra L Saunders, and
Robert Heinsohn. 2015. ``Online Localization of Radio-Tagged Wildlife
with an Autonomous Aerial Robot System.'' \emph{Robot. Sci. Syst.}, no.
November 2016: 1--9.
doi:\href{https://doi.org/10.15607/RSS.2015.XI.042}{10.15607/RSS.2015.XI.042}.

\hypertarget{ref-colefax_potential_2017}{}
Colefax, Andrew P., Paul A. Butcher, and Brendan P. Kelaher. 2017. ``The
Potential for Unmanned Aerial Vehicles (UAVs) to Conduct Marine Fauna
Surveys in Place of Manned Aircraft.'' \emph{ICES Journal of Marine
Science}, June.
doi:\href{https://doi.org/10.1093/icesjms/fsx100}{10.1093/icesjms/fsx100}.

\hypertarget{ref-Conabio2017}{}
Conabio. 2017. ``\emph{Canarias Usará Drones Para Controlar La Pesca
Furtiva Y Mejorar Su Inspección}.''

\hypertarget{ref-cook_design_1979}{}
Cook, R. Dennis, and Jerald O. Jacobson. 1979. ``A Design for Estimating
Visibility Bias in Aerial Surveys.'' \emph{Biometrics} 35 (4): 735.
doi:\href{https://doi.org/10.2307/2530104}{10.2307/2530104}.

\hypertarget{ref-ditmer_bears_2015}{}
Ditmer, Mark A., John B. Vincent, Leland K. Werden, Jessie C. Tanner,
Timothy G. Laske, Paul A. Iaizzo, David L. Garshelis, and John R.
Fieberg. 2015. ``Bears Show a Physiological but Limited Behavioral
Response to Unmanned Aerial Vehicles.'' \emph{Curr. Biol.} 25 (17):
2278--2283.
doi:\href{https://doi.org/10.1016/j.cub.2015.07.024}{10.1016/j.cub.2015.07.024}.

\hypertarget{ref-dudley_guidelines_2008}{}
Dudley, Nigel. 2008. \emph{Guidelines for Applying Protected Area
Management Categories}. IUCN.

\hypertarget{ref-duffy_waging_2014}{}
Duffy, Rosaleen. 2014. ``Waging a War to Save Biodiversity: The Rise of
Militarized Conservation.'' \emph{Int. Aff.} 90 (4): 819--834.
doi:\href{https://doi.org/10.1111/1468-2346.12142}{10.1111/1468-2346.12142}.

\hypertarget{ref-fornace_mapping_2014}{}
Fornace, Kimberly M., Chris J. Drakeley, Timothy William, Fe Espino, and
Jonathan Cox. 2014. ``Mapping Infectious Disease Landscapes: Unmanned
Aerial Vehicles and Epidemiology.'' \emph{Trends Parasitol.}, October,
1--6.
doi:\href{https://doi.org/10.1016/j.pt.2014.09.001}{10.1016/j.pt.2014.09.001}.

\hypertarget{ref-forum_toward_2008}{}
Forum, Policy. 2008. ``Toward a Global Biodiversity Observing System,''
no. April.

\hypertarget{ref-franco_five_2016}{}
Franco, Antonio Di, Pierre Thiriet, Giuseppe Di Carlo, Charalampos
Dimitriadis, Patrice Francour, Nicolas L Gutiérrez, Alain Jeudy De
Grissac, et al. 2016. ``Five Key Attributes Can Increase Marine
Protected Areas Performance for Small-Scale Fisheries Management.''
\emph{Nat. Publ. Gr.}, no. November: 1--9.
doi:\href{https://doi.org/10.1038/srep38135}{10.1038/srep38135}.

\hypertarget{ref-van_gemert_nature_2015}{}
Gemert, Jan C. van, Camiel R. Verschoor, Pascal Mettes, Kitso Epema,
Lian Pin Koh, and Serge Wich. 2015. ``Nature Conservation Drones for
Automatic Localization and Counting of Animals.'' \emph{Lect. Notes
Comput. Sci. (Including Subser. Lect. Notes Artif. Intell. Lect. Notes
Bioinformatics)} 8925: 255--270.
doi:\href{https://doi.org/10.1007/978-3-319-16178-5_17}{10.1007/978-3-319-16178-5\_17}.

\hypertarget{ref-Georg2016}{}
Gerster/Panos, Georg. 2017. ``\emph{Project Uses Drones to Monitor
Coastal Erosion in Ghana}.''

\hypertarget{ref-gini_aerial_2012}{}
Gini, R., D. Passoni, L. Pinto, and G. Sona. 2012. ``Aerial Images from
an Uav System: 3D Modeling and Tree Species Classification in a Park
Area.'' \emph{ISPRS - Int. Arch. Photogramm. Remote Sens. Spat. Inf.
Sci.} XXXIX-B1 (September): 361--366.
doi:\href{https://doi.org/10.5194/isprsarchives-XXXIX-B1-361-2012}{10.5194/isprsarchives-XXXIX-B1-361-2012}.

\hypertarget{ref-gremillet_robots_2012}{}
Grémillet, David, William Puech, Véronique Garçon, Thierry Boulinier,
and Yvon Le Maho. 2012. ``Robots in Ecology: Welcome to the Machine.''
\emph{Open J. Ecol.} 02 (2): 49--57.
doi:\href{https://doi.org/10.4236/oje.2012.22006}{10.4236/oje.2012.22006}.

\hypertarget{ref-hansen_applying_2016}{}
Hansen, Andreas Skriver. 2016. ``Applying Visitor Monitoring Methods in
Coastal and Marine Areas -- Some Learnings and Critical Reflections from
Sweden.'' \emph{Scand. J. Hosp. Tour.} 2250 (June): 1--18.
doi:\href{https://doi.org/10.1080/15022250.2016.1155481}{10.1080/15022250.2016.1155481}.

\hypertarget{ref-hardin_small-scale_2013}{}
Hardin, Perry J, and Ryan R Jensen. 2013. ``Small-Scale Unmanned Aerial
Vehicles in Environmental Remote Sensing: Challenges and
Opportunities,'' no. October 2014: 37--41.
doi:\href{https://doi.org/10.2747/1548-1603.48.1.99}{10.2747/1548-1603.48.1.99}.

\hypertarget{ref-hodgson_best_2016}{}
Hodgson, Jarrod C., and Lian Pin Koh. 2016. ``Best Practice for
Minimising Unmanned Aerial Vehicle Disturbance to Wildlife in Biological
Field Research.'' \emph{Curr. Biol.} 26 (10).
doi:\href{https://doi.org/10.1016/j.cub.2016.04.001}{10.1016/j.cub.2016.04.001}.

\hypertarget{ref-Canarias2017}{}
INFORCASA. 2017. ``\emph{Canarias Usará Drones Para Controlar La Pesca
Furtiva Y Mejorar Su Inspección}.''

\hypertarget{ref-jain_unmanned_2013}{}
Jain, Mukesh. 2013. ``Unmanned Aerial Survey of Elephants.'' \emph{PLoS
One}.
doi:\href{https://doi.org/10.1371/\%20journal.pone.0054700}{10.1371/ journal.pone.0054700}.

\hypertarget{ref-jewell_effect_2013}{}
Jewell, Zoe. 2013. ``Effect of Monitoring Technique on Quality of
Conservation Science: Ethics and Science in Conservation.''
\emph{Conservation Biology} 27 (3): 501--508.
doi:\href{https://doi.org/10.1111/cobi.12066}{10.1111/cobi.12066}.

\hypertarget{ref-juffe-bignoli_protected_2014}{}
Juffe-Bignoli, Diego, Neil David Burgess, H Bingham, E M S Belle, M G De
Lima, M Deguignet, B Bertzky, et al. 2014. ``Protected Planet Report
2014.'' \emph{Cambridge, UK UNEP-WCMC}.

\hypertarget{ref-king_will_2014}{}
King, Lisa M. 2014. ``Will Drones Revolutionise Ecotourism?'' \emph{J.
Ecotourism} 13 (1): 85--92.
doi:\href{https://doi.org/10.1080/14724049.2014.948448}{10.1080/14724049.2014.948448}.

\hypertarget{ref-koh_dawn_2012}{}
Koh, Lian Pin, and Serge A. Wich. 2012. ``Dawn of Drone Ecology:
Low-Cost Autonomous Aerial Vehicles for Conservation.'' \emph{Trop.
Conserv. Sci.} 5 (2): 121--132.
doi:\href{https://doi.org/WOS:000310846600002}{WOS:000310846600002}.

\hypertarget{ref-korner_autonomous_2010}{}
Körner, Fabian, Raphael Speck, Ali Haydar, and Salah Sukkarieh. 2010.
``Autonomous Airborne Wildlife Tracking Using Radio Signal Strength,''
107--112.

\hypertarget{ref-lancia_estimating_2005}{}
Lancia, Richard A, William L Kendall, Kenneth H Pollock, and James D
Nichols. 2005. ``Estimating the Number of Animals in Wildlife
Populations.''

\hypertarget{ref-leary_drones_2017}{}
Leary, David. 2017. ``Drones on Ice: An Assessment of the Legal
Implications of the Use of Unmanned Aerial Vehicles in Scientific
Research and by the Tourist Industry in Antarctica.'' \emph{Polar
Record}, May, 1--15.
doi:\href{https://doi.org/10.1017/S0032247417000262}{10.1017/S0032247417000262}.

\hypertarget{ref-leonardo_miniature_2013}{}
Leonardo, Miguel, Austin Jensen, Calvin Coopmans, Mac McKee, and
YangQuan Chen. 2013. ``A Miniature Wildlife Tracking UAV Payload System
Using Acoustic Biotelemetry.'' \emph{Proc. ASME Int. Des. Eng. Tech.
Conf. Comput. Inf. Eng. Conf.}, no. July 2015.
doi:\href{https://doi.org/10.1115/DETC2013-13267}{10.1115/DETC2013-13267}.

\hypertarget{ref-lhoest_how_2015}{}
Lhoest, S., J. Linchant, S. Quevauvillers, C. Vermeulen, and P. Lejeune.
2015. ``How Many Hippos (Homhip): Algorithm for Automatic Counts of
Animals with Infra-Red Thermal Imagery from UAV.'' \emph{Int. Arch.
Photogramm. Remote Sens. Spat. Inf. Sci. - ISPRS Arch.} 40 (3):
355--362.
doi:\href{https://doi.org/10.5194/isprsarchives-XL-3-W3-355-2015}{10.5194/isprsarchives-XL-3-W3-355-2015}.

\hypertarget{ref-linchant_are_2015}{}
Linchant, Julie, Jonathan Lisein, Jean Semeki, Philippe Lejeune, and
Cédric Vermeulen. 2015. ``Are Unmanned Aircraft Systems (UASs) the
Future of Wildlife Monitoring? A Review of Accomplishments and
Challenges.'' \emph{Mamm. Rev.} 45 (4): 239--252.
doi:\href{https://doi.org/10.1111/mam.12046}{10.1111/mam.12046}.

\hypertarget{ref-lisein_discrimination_2015}{}
Lisein, Jonathan, Adrien Michez, Hugues Claessens, and Philippe Lejeune.
2015. ``Discrimination of Deciduous Tree Species from Time Series of
Unmanned Aerial System Imagery.'' \emph{PLoS One} 10 (11).
doi:\href{https://doi.org/10.1371/journal.pone.0141006}{10.1371/journal.pone.0141006}.

\hypertarget{ref-lobermeier_mitigating_2015}{}
Lobermeier, Scott, Matthew Moldenhauer, Christopher Peter, Luke
Slominski, Richard Tedesco, Marcus Meer, James Dwyer, Richard Harness,
and Andrew Stewart. 2015. ``Mitigating Avian Collision with Power Lines:
A Proof of Concept for Installation of Line Markers via Unmanned Aerial
Vehicle.'' \emph{J. Unmanned Veh. Syst.} 3 (4): 252--258.
doi:\href{https://doi.org/10.1139/juvs-2015-0009}{10.1139/juvs-2015-0009}.

\hypertarget{ref-margarita_mulero-pazmany_juan_jose_negro_low_2014}{}
Margarita Mulero-Pázmány, Miguel Ferrer, Juan José Negro. 2014. ``A Low
Cost Way for Assessing Bird Risk Hazards in Power Lines: Fixed-Wing
Small Unmanned Aircraft Systems'' 2.

\hypertarget{ref-mateo_modelos_2011}{}
Mateo, Rubén G., Ángel M. Felicísimo, and Jesús Muñoz. 2011. ``Modelos
de Distribución de Especies: Una Revisión Sintética.'' \emph{Rev. Chil.
Hist. Nat.}, 217--240.
doi:\href{https://doi.org/10.4067/S0716-078X2011000200008}{10.4067/S0716-078X2011000200008}.

\hypertarget{ref-mccaldin_use_2015}{}
McCaldin, Guy, Michael Johnston, and Andrew Rieker. 2015. \emph{Use of
Unmanned Aircraft Systems to Assist with Decision Support for Land
Managers on Christmas Island (Indian Ocean)}. October. Australia: V-TOL
Aerospace; Department of parks; Wildlife, Western Australia.

\hypertarget{ref-mcevoy_evaluation_2016}{}
McEvoy, John F., Graham P. Hall, and Paul G. McDonald. 2016.
``Evaluation of Unmanned Aerial Vehicle Shape, Flight Path and Camera
Type for Waterfowl Surveys: Disturbance Effects and Species
Recognition.'' \emph{PeerJ} 4 (March): e1831.
doi:\href{https://doi.org/10.7717/peerj.1831}{10.7717/peerj.1831}.

\hypertarget{ref-mulero-pazmany_margarita_aeromab_2011}{}
Mulero-Pázmány Margarita, Negro JJ. 2011. ``AEROMAB Small UAS for
Montagu's Harrier's Circus Pygargus Nests Monitoring.'' \emph{AEROMAB
Small UAS Montagu's Harrier's Circus Pygargus Nests Monit. RED UAS
Intenational Congr. Univ. Eng. Seville, Spain. December 2011.}

\hypertarget{ref-mulero-pazmany_unmanned_2015}{}
Mulero-Pázmány, Margarita, Jose Ángel Barasona, Pelayo Acevedo, Joaquín
Vicente, and Juan José Negro. 2015. ``Unmanned Aircraft Systems
Complement Biologging in Spatial Ecology Studies.'' \emph{Ecol. Evol.} 5
(21): 4808--4818.
doi:\href{https://doi.org/10.1002/ece3.1744}{10.1002/ece3.1744}.

\hypertarget{ref-mulero-pazmany_remotely_2014}{}
Mulero-Pázmány, Margarita, Roel Stolper, L. D. Van Essen, Juan J. Negro,
and Tyrell Sassen. 2014. ``Remotely Piloted Aircraft Systems as a
Rhinoceros Anti-Poaching Tool in Africa.'' \emph{PLoS One} 9 (1): 1--10.
doi:\href{https://doi.org/10.1371/journal.pone.0083873}{10.1371/journal.pone.0083873}.

\hypertarget{ref-nugraha_urgency_2016}{}
Nugraha, Ridha Aditya, Deepika Jeyakodi, and Thitipon Mahem. 2016.
``Urgency for Legal Framework on Drones : Lessons for Indonesia , India
, and Thailand.'' \emph{Indones. Law Rev.} 6 (2): 137--157.

\hypertarget{ref-pereira_essential_2013}{}
Pereira, Henrique Miguel, Simon Ferrier, Michele Walters, Gary N Geller,
Rob H G Jongman, Robert J Scholes, Michael W Bruford, et al. 2013.
``Essential Biodiversity Variables.'' \emph{Science (80-. ).} 339
(6117): 277--278.
doi:\href{https://doi.org/10.1126/science.1229931}{10.1126/science.1229931}.

\hypertarget{ref-perroy_assessing_2017}{}
Perroy, Ryan L., Timo Sullivan, and Nathan Stephenson. 2017. ``Assessing
the Impacts of Canopy Openness and Flight Parameters on Detecting a
Sub-Canopy Tropical Invasive Plant Using a Small Unmanned Aerial
System.'' \emph{ISPRS J. Photogramm. Remote Sens.} 125: 174--183.
doi:\href{https://doi.org/10.1016/j.isprsjprs.2017.01.018}{10.1016/j.isprsjprs.2017.01.018}.

\hypertarget{ref-pimm_emerging_2015}{}
Pimm, Stuart L, Sky Alibhai, Richard Bergl, Alex Dehgan, Chandra Giri,
Zoë Jewell, Lucas Joppa, Roland Kays, and Scott Loarie. 2015. ``Emerging
Technologies to Conserve Biodiversity.'' \emph{Trends Ecol. Evol.} 30
(11): 685--696.
doi:\href{https://doi.org/10.1016/j.tree.2015.08.008}{10.1016/j.tree.2015.08.008}.

\hypertarget{ref-pomeroy_assessing_2015}{}
Pomeroy, P., L. O'Connor, and P. Davies. 2015. ``Assessing Use of and
Reaction to Unmanned Aerial Systems in Gray and Harbor Seals During
Breeding and Molt in the UK \(^{\textrm{1}}\).'' \emph{Journal of
Unmanned Vehicle Systems} 3 (3): 102--113.
doi:\href{https://doi.org/10.1139/juvs-2015-0013}{10.1139/juvs-2015-0013}.

\hypertarget{ref-rodriguez_eye_2012}{}
Rodríguez, Airam, Juan J. Negro, Mara Mulero, Carlos Rodríguez, Jesús
Hernández-Pliego, and Javier Bustamante. 2012. ``The Eye in the Sky:
Combined Use of Unmanned Aerial Systems and GPS Data Loggers for
Ecological Research and Conservation of Small Birds.'' \emph{PLoS One} 7
(12).
doi:\href{https://doi.org/10.1371/journal.pone.0050336}{10.1371/journal.pone.0050336}.

\hypertarget{ref-sarda-palomera_fine-scale_2012}{}
Sardà-Palomera, Francesc, Gerard Bota, Carlos Viñolo, Oriol Pallarés,
Víctor Sazatornil, Lluís Brotons, Spartacus Gomáriz, and Francesc Sardà.
2012. ``Fine-Scale Bird Monitoring from Light Unmanned Aircraft
Systems.'' \emph{Ibis (Lond. 1859).} 154 (1): 177--183.
doi:\href{https://doi.org/10.1111/j.1474-919X.2011.01177.x}{10.1111/j.1474-919X.2011.01177.x}.

\hypertarget{ref-sasse_job-related_2003}{}
Sasse, D. Blake. 2003. ``Job-Related Mortality of Wildlife Workers in
the United States, 1937-2000.'' \emph{Wildl. Soc. Bull.} 31 (4):
1000--1003.

\hypertarget{ref-sepulveda_review_2010}{}
Sepúlveda, Alejandra, Mathias Schluep, Fabrice G. Renaud, Martin
Streicher, Ruediger Kuehr, Christian Hagelüken, and Andreas C. Gerecke.
2010. ``A Review of the Environmental Fate and Effects of Hazardous
Substances Released from Electrical and Electronic Equipments During
Recycling: Examples from China and India.'' \emph{Environmental Impact
Assessment Review} 30 (1): 28--41.
doi:\href{https://doi.org/10.1016/j.eiar.2009.04.001}{10.1016/j.eiar.2009.04.001}.

\hypertarget{ref-soriano_rf-based_2009}{}
Soriano, P, F Caballero, and A Ollero. 2009. ``RF-Based Particle Filter
Localization for Wildlife Tracking by Using an UAV.'' \emph{40 Th Int.
Symp. Robot.}, 239--244.
\url{http://grvc.us.es/publica/congresosint/documentos/isr_soriano.pdf}.

\hypertarget{ref-stocker_review_2017}{}
Stöcker, Claudia, Rohan Bennett, Francesco Nex, Markus Gerke, and Jaap
Zevenbergen. 2017. ``Review of the Current State of UAV Regulations.''
\emph{Remote Sensing} 9 (5): 459.
doi:\href{https://doi.org/10.3390/rs9050459}{10.3390/rs9050459}.

\hypertarget{ref-szantoi_mapping_2017}{}
Szantoi, Zoltan, Scot E. Smith, Giovanni Strona, Lian Pin Koh, and Serge
A. Wich. 2017. ``Mapping Orangutan Habitat and Agricultural Areas Using
Landsat OLI Imagery Augmented with Unmanned Aircraft System Aerial
Photography.'' \emph{Int. J. Remote Sens.} 38 (8): 1--15.
doi:\href{https://doi.org/10.1080/01431161.2017.1280638}{10.1080/01431161.2017.1280638}.

\hypertarget{ref-tomlins_remotely_1983}{}
Tomlins, G.F., and Y.J. Lee. 1983. ``Remotely Piloted Aircraft --- an
Inexpensive Option for Large-Scale Aerial Photography in Forestry
Applications.'' \emph{Can. J. Remote Sens.} 9 (2): 76--85.
doi:\href{https://doi.org/10.1080/07038992.1983.10855042}{10.1080/07038992.1983.10855042}.

\hypertarget{ref-USGS2014}{}
USGS. 2014. ``\emph{US Geological Survey National Unmanned Aircraft
Systems Project}.''

\hypertarget{ref-vas_approaching_2015}{}
Vas, E., A. Lescroel, O. Duriez, G. Boguszewski, and D. Gremillet. 2015.
``Approaching Birds with Drones: First Experiments and Ethical
Guidelines.'' \emph{Biol. Lett.} 11 (2): 20140754--20140754.
doi:\href{https://doi.org/10.1098/rsbl.2014.0754}{10.1098/rsbl.2014.0754}.

\hypertarget{ref-ventura_low-cost_2016}{}
Ventura, Daniele, Michele Bruno, Giovanna Jona Lasinio, Andrea
Belluscio, and Giandomenico Ardizzone. 2016. ``A Low-Cost Drone Based
Application for Identifying and Mapping of Coastal Fish Nursery
Grounds.'' \emph{Estuar. Coast. Shelf Sci.} 171.
doi:\href{https://doi.org/10.1016/j.ecss.2016.01.030}{10.1016/j.ecss.2016.01.030}.

\hypertarget{ref-weissensteiner_low-budget_2015}{}
Weissensteiner, M H, J W Poelstra, and J B W Wolf. 2015. ``Low-Budget
Ready-to-Fly Unmanned Aerial Vehicles: An Effective Tool for Evaluating
the Nesting Status of Canopy-Breeding Bird Species.'' \emph{J. Avian
Biol.} 46 (4): 425--430.
doi:\href{https://doi.org/10.1111/jav.00619}{10.1111/jav.00619}.

\hypertarget{ref-WichS2017}{}
Wich, S. 2017. ``\emph{Amazon River Dolphin Project}.''

\hypertarget{ref-wich_preliminary_2016}{}
Wich, Serge, David Dellatore, Max Houghton, Rio Ardi, and Lian Pin Koh.
2016. ``A Preliminary Assessment of Using Conservation Drones for
Sumatran Orang-Utan (Pongo Abelii) Distribution and Density.'' \emph{J.
Unmanned Veh. Syst.} 4 (1): 45--52.
doi:\href{https://doi.org/10.1139/juvs-2015-0015}{10.1139/juvs-2015-0015}.

\hypertarget{ref-wilson_measuring_2006}{}
Wilson, Rory P., and Clive R. McMahon. 2006. ``Measuring Devices on Wild
Animals: What Constitutes Acceptable Practice?'' \emph{Frontiers in
Ecology and the Environment} 4 (3): 147--154.
doi:\href{https://doi.org/10.1890/1540-9295(2006)004\%5B0147:MDOWAW\%5D2.0.CO;2}{10.1890/1540-9295(2006)004{[}0147:MDOWAW{]}2.0.CO;2}.

\hypertarget{ref-wulder_high_2004}{}
Wulder, Michael A, Ronald J Hall, Nicholas C Coops, and Steven E
Franklin. 2004. ``High Spatial Resolution Remotely Sensed Data for
Ecosystem Characterization'' 54 (6): 511--521.
doi:\href{https://doi.org/10.1641/0006-3568(2004)054}{10.1641/0006-3568(2004)054}.

\hypertarget{ref-zahawi_using_2015}{}
Zahawi, Rakan A., Jonathan P. Dandois, Karen D. Holl, Dana Nadwodny, J.
Leighton Reid, and Erle C. Ellis. 2015. ``Using Lightweight Unmanned
Aerial Vehicles to Monitor Tropical Forest Recovery.'' \emph{Biol.
Conserv.} 186 (June): 287--295.
doi:\href{https://doi.org/10.1016/j.biocon.2015.03.031}{10.1016/j.biocon.2015.03.031}.

\hypertarget{ref-zaman_use_2011}{}
Zaman, Bushra, Austin M. Jensen, and Mac McKee. 2011. ``Use of
High-Resolution Multispectral Imagery Acquired with an Autonomous
Unmanned Aerial Vehicle to Quantify the Spread of an Invasive Wetlands
Species.'' \emph{Int. Geosci. Remote Sens. Symp.}, 803--806.
doi:\href{https://doi.org/10.1109/IGARSS.2011.6049252}{10.1109/IGARSS.2011.6049252}.

\newpage
\singlespacing 
\end{document}
