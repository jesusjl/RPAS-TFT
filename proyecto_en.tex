% interactcadsample.tex mod using svm.latex.ms.texs
% v1.03 - April 2017
\documentclass[]{interact}
%\documentclass[a4paper]{article}
\usepackage{epstopdf}% To incorporate .eps illustrations using PDFLaTeX, etc.
\usepackage{subfigure}% Support for small, `sub' figures and tables
%\usepackage[nolists,tablesfirst]{endfloat}% To `separate' figures and tables from text if required
% 
\usepackage{natbib}% Citation support using natbib.sty
\bibpunct[, ]{(}{)}{;}{a}{}{,}% Citation support using natbib.sty
\renewcommand\bibfont{\fontsize{10}{12}\selectfont}% Bibliography support using natbib.sty

\theoremstyle{plain}% Theorem-like structures provided by amsthm.sty
\newtheorem{theorem}{Theorem}[section]
\newtheorem{lemma}[theorem]{Lemma}
\newtheorem{corollary}[theorem]{Corollary}
\newtheorem{proposition}[theorem]{Proposition}

\theoremstyle{definition}
\newtheorem{definition}[theorem]{Definition}
\newtheorem{example}[theorem]{Example}

\theoremstyle{remark}
\newtheorem{remark}{Remark}
\newtheorem{notation}{Notation}


\usepackage{rotating}
% \usepackage[]{geometry}
\usepackage{url}




\usepackage{layout}
\setlength{\headsep}{15pt}
\usepackage{booktabs}
\usepackage{rotating}
\usepackage{graphicx}
\usepackage{setspace}
\usepackage{pdflscape}
\usepackage{longtable}
\usepackage{tabu}
\onehalfspacing
\usepackage{helvet}


\usepackage[T1]{fontenc}
\usepackage[utf8]{inputenc}
\usepackage[nottoc,notlof,notlot]{tocbibind} 

\usepackage{abstract}
\renewcommand{\abstractname}{}    % clear the title
\renewcommand{\absnamepos}{empty} % originally center

\renewenvironment{abstract}
 {{
    \setlength{\leftmargin}{10mm}
    \setlength{\rightmargin}{\leftmargin}%
  }
  \relax}
 {\endlist}


\newtheorem{hypothesis}{Hypothesis}
\usepackage{setspace}

\makeatletter
\@ifpackageloaded{hyperref}{}{%
\ifxetex
  \usepackage[setpagesize=false, % page size defined by xetex
              unicode=false, % unicode breaks when used with xetex
              xetex]{hyperref}
\else
  \usepackage[unicode=true]{hyperref}
\fi
}
\@ifpackageloaded{color}{
    \PassOptionsToPackage{usenames,dvipsnames}{color}
}{%
    \usepackage[usenames,dvipsnames]{color}
}
\makeatother
\hypersetup{breaklinks=true,
            bookmarks=true,
            pdfauthor={Jesús Jiménez López (University) and Margarita Mulero-Pázmány (University)},
             pdfkeywords = {RPAS, UAV, drones, natural protected areas, conservation, biodiversity},  
            pdftitle={Contribution of RPAS in research and conservation in protected areas:
present and future},
            colorlinks=true,
            citecolor=blue,
            urlcolor=blue,
            linkcolor=magenta,
            pdfborder={0 0 0}}
\urlstyle{same}  % don't use monospace font for urls


% notice vspace after title
\title{Contribution of RPAS in research and conservation in protected areas:
present and future\vspace{0.25in}  }


\author{\Large Jesús Jiménez López\vspace{0.05in} \newline\normalsize\emph{University }   \and \Large Margarita Mulero-Pázmány\vspace{0.05in} \newline\normalsize\emph{University }  }

\date{}


\usepackage{lscape}
\newcommand{\blandscape}{\begin{landscape}}
\newcommand{\elandscape}{\end{landscape}}

%%%%%%%%%%%%%%%%%%%%%%%%%%%%%%%%%%%%%%%%%%%%%%%%%%%%%%%%%%%%%%%%%%%%%%%%555

\begin{document}

\articletype{ARTICLE TEMPLATE}

% \pagenumbering{arabic}% resets `page` counter to 1 
%
\author{
\name{A.~N. Author\textsuperscript{a}\thanks{CONTACT L.~Jimenez. Jesus. Email: foo@foo.com} and Margarita Múlero-Pázmani\textsuperscript{b}}
\affil{\textsuperscript{a}Taylor \& Francis, 4 Park Square, Milton Park, Abingdon, UK; \textsuperscript{b}Institut f\"{u}rDepartment of Evolutionary Ecology, Doñana Biological Station, CSIC\"{a}t,  Seville, Spain}
}
% \author{\Large Jesús Jiménez López\vspace{0.05in} \newline\normalsize\emph{University}  \and \Large Margarita Mulero-Pázmány\vspace{0.05in} \newline\normalsize\emph{University} }
\title{Contribution of RPAS in research and conservation in protected areas:
present and future}
\maketitle  % title \par 
  




  \fontsize{9}{10}{\textbf{ABSTRACT}}: \par
  \begingroup
  \leftskip1em
  \rightskip\leftskip
  \noindent{\fontsize{9}{10}\selectfont Nature conservation has benefited from a wide range of technological
advances, from remote sensing, camera traps or wildlife tracking devices
to a continuous development of computational tools and analytical
techniques to deal with large amounts of data collected. All of them
play an important role in safeguarding biodiversity and better
understanding of ecosystem functions. In the last few decades, we have
witnessed a growing interest in projects aimed to evaluate the
feasibility of RPAS for conservation purposes. So far, RPAS have been
tested or directly applied for a variety of research and management
activities including environmental and wildlife monitoring or
anti-poaching strategies. It remains to be seen how well RPAS complement
conservation actions in protected areas, since there are technical,
ethical and legal barriers that currently limit their effectiveness.
However, through the development of novel solutions and continuous
improvement of performance and sensor characteristics, the parallel
design of computer intensive statistical methods and a greater concern
for animal welfare issues, there are high expectations in the research
community and RPAS applications in protected areas are increasingly
justified.}
  \rightskip5em
  \par
  \endgroup
  \vspace{5mm}
  


    
 \fontsize{9}{10}{\textbf{KEYWORDS}}: \par
  \begingroup
  \leftskip1em
  \rightskip\leftskip
  \noindent{\fontsize{9}{10}\selectfont RPAS, UAV, drones, natural protected areas, conservation, biodiversity}
  \rightskip5em
  \par
  \endgroup
  \vspace{5mm}


% 

\vskip 6.5pt

\noindent  \section{Introduction}\label{introduction}

\subsection{Current context}\label{current-context}

Civil applications of remotely piloted aircraft systems (RPAS, also
known as unmanned aerial systems, UAS, drones) have been raised in an
growing number of scientific articles. During the last few years there
have been a significant amount of wildlife research projects in natural
protected areas using RPAS (Linchant et al. 2015; Chabot and Bird 2015;
Christie et al. 2016) and it is rapidly taking its place within the
broad set of technological solutions that support conservation (Pimm et
al. 2015). In most cases, feasibility studies were carried out,
assessing the capacity of RPAS in relation to traditional conservation
instruments by measuring the overall performance, delimiting their
strengths and weakness, and establishing guidelines and recommendations,
resulting in new perspectives of application.

Although the potential of RPAS for mapping is tackled at the end of the
seventies (Colomina and Molina 2014), we found some references dating
back to the early 1980s, where first trials with RPAS applied on
environmental issues began with the objective of acquiring aerial
photographs and demonstrating their usefulness in forestry applications,
fish resource management or the coupling of sensors for atmospheric
studies (Tomlins and Lee 1983). At the beginning of the 21st century,
first mapping surveys of vegetation in threatened species appeared
(Quilter and Anderson 2000), while accross the first decade the number
of publications began to increase significantly (Hardin and Jensen
2013). At present there are some initiatives that seek to determine the
current state of the RPAS in ecology and conservation. Recently, the
journals \emph{Remote Sensing in Ecology and Conservation} and the
\emph{International Journal of Remote Sensing} made a call to the
scientific community for the sending of proposals in order to update the
current state of RPAS applied into the environmental sphere. As result,
a significant production of related papers on the matter is expected. On
the other hand, it is remarkable the greater presence of web portals
that center their activity around civil applications with RPAS. In the
field of research applied to conservation, the website
\url{http://conservationdrones.org/} is a worldwide reference, whose
contents illustrate recent pioneering projects, so they are not always
reflected in the scientific literature. The popularity of RPAS has
transcended the scientific-technical field, giving rise to the emergence
of user communities with a large presence on the Internet. One of the
most active portals is \url{http://diydrones.com/}, which brings
together fans of the do-it-yourself philosophy that encourages the use
of open platforms versus the traditional closed systems offered by the
traditional industry. This has unchained the reduction of costs of these
equipment and, together with the development of specialized open source
software, have led to the democratization of technology, bringing it
closer to a broad number of users and organizations. The scientific
community has probably benefited from this general trend. For some
authors, the flexibility in the assembly of RPAS offers in principle a
greater degree of customization, allowing to combine different sensors
and control systems, according to the particular needs of each project
and without having to depend on others beyond the research group itself
(Koh and Wich 2012). In the commercial field, more companies offer RPAS
of high performance and reliability together with professional services
and software, so the sector benefits from great dynamism.

\subsection{Protected areas}\label{protected-areas}

As defined by UICN, ``a protected area is a clearly defined geographical
space, recognized, dedicated and managed, through legal or other
effective means, to achieve the long term conservation of nature with
associated ecosystem services and cultural values'' (Dudley 2008).
Despite the fact that the number of protected areas has raised
considerably at a global level, with 15.4\% of the land area and 8.4\%
of the marine areas under some protection figure (Juffe-Bignoli et al.
2014) the size of wildlife populations has been estimated to have
decreased by 52\% in the period 1970 to 2012. Habitat change and
fragmentation, severe pollution particularly in freshwater ecosystems,
overexploitation of resources, climate change and the impact of invasive
species on indigenous populations have been identified as the main
threats to biodiversity (Barnosky et al. 2011; Secretariat of the
Convention on Biological Diversity and UNEP World Conservation
Monitoring Centre 2006). To address the current environmental crisis,
the Convention on Biological Diversity (CBD) established in Nagoya,
Japan, a strategic plan for the period 2011-2020 which includes the
so-called Aichi targets for biological diversity. Among the stated
objectives is the increase of protected area systems of particular
importance for biodiversity and ecosystem services (target 11).

Protected areas are reference sites for monitoring and managing
biodiversity. As part of the large array of observing systems monitoring
ecosystems and measuring human impacts (Forum 2008, Pereira et al.
(2013)), RPAS can fill the gap at an intermediate spatial scale,
surpassing the financial and technological constrains of remote sensing
and ground / aerial manned vehicles based surveys (Koh and Wich 2012;
Rodríguez et al. 2012; Chabot and Bird 2015). First, while it is
possible to acquire satellite images at low or virtually zero cost
(LandSat, MODIS, Sentinel, etc.), most of these platforms operate on a
global or regional scale. The limited spatial and temporal resolution,
added to the inconveniences of cloud presence especially noticeable in
tropical areas, reduces the effectiveness of remote sensing in the
collection of data at fine-scale, according to the requirements of
ecological studies at the level of species, habitats or populations
(Wulder et al. 2004). Secondly, the large extent of these protected
areas significantly increases the costs of field work on foot,
particularly in hadarzous and inaccessible spots. Finally, while manned
aerial vehicles offers an optimal alternative for covering much larger
areas, they suffer from excessively high operational costs and when not
used for aerial mapping, they could be affected by observer bias. In
addition, air accidents are ranking as the leading cause of death in
wildlife specialists in the United States (Sasse 2003). As a
consequence, RPAS have been positioned as an appropriate complement for
conservation activities (Zahawi et al. 2015) avoiding to a certain
extent some of the above-mentioned drawbacks. In developing countries,
especially sensitive in terms of budgetary allocations and technical
capacities, monitoring and surveillance programs are being successfully
developed through the use of RPAS. For instance, by capturing aerial
images in the Volta delta, Ghana, a team of scientists measured the
effects of climate change on coastal areas and evaluates the
effectiveness of prevention and restoration measures against erosive
processes (Gerster/Panos 2017).

\subsection{Legal barriers}\label{legal-barriers}

RPAS operations faces important legal barriers that undermine the true
potential in the civilian sphere (Stöcker et al. 2017). An overly
restrictive regulatory framework could limit the possibilities of use of
the RPAS in the field of conservation, which makes clear the urgent need
to harmonize legislation . In the United States and in most of the
European countries consulted, interim legislation has been adopted
which, to a certain extent, equates the management of RPAS with that of
traditional aircraft. In general terms, the situation in Latin America
is uneven, however there is a general tendency to develop specific laws
to cope with the rise of the RPAS in both the civil and military sectors
(America 2017). Africa is one of the continents where the impact of RPAS
in conservation has had greater repercussions. However, in the opinion
of some conservationists, their use has not been without problems,
resulting in governments that have totally or partially prohibited drone
operations, arguing national security problems in detriment of
protection of natural areas (Andrews 2014). But RPAS have also been
generally welcomed in several developing countries in Asia, where an
array of related programs are being carried out (Nugraha, Jeyakodi, and
Mahem 2016). (Consulting 2017) is a relative accurate database informing
RPAS regulations by country. The uncertainty of the users along the
world has promoted the development of associations in order to advise on
the legal aspects to be taken into account during the operation, with
the International Association for Unmanned Vehicle Systems (AUVSI)
(UAVSI 2017) being the largest nonprofit organization in the world
dedicated to advancing the community of unmanned aerial vehicles users.
Also, a relative up-to-date database has been published online where
users can consult the regulatory framework of RPAS by country
(Consulting 2017).

\section{Methods}\label{methods}

To achieve the proposed objectives, a bibliographical review of
scientific articles, gray literature, postgraduate theses, websites and
specialized journals was carried out, following a similar line to other
studies (Linchant et al. 2015; Christie et al. 2016). The main tools for
selection of the cited bibliography include Google Schoolar, Research
Gate and Mendeley Desktop, while the use of Internet search engines
include other references outside the scientific scope. Key search
criteria for keywords included unmanned aerial vehicles in their various
meanings and acronyms (RPAS, UAV, UAS, drones), along with a variety of
terms referring to natural protected areas, primarily in English. Last
references revised were published on June, 2017.

The selected information was categorized according to the role played by
RPAS in direct or indirect relation to conservation in protected areas.
It is presented in tabular format, identifying where the study was
conducted, the expected accomplishments and technical specifications of
the aerial platform. After posing main results obtained, gaps are
identified and possible scenarios for implementing RPAS as essential
tools to help achieve conservation plans in protected areas are
discussed, highlighting some trends and opportunities that apparently
have not yet been adequately exploited.

\section{Results}\label{results}

\subsection{Wildlife Monitoring and
Management}\label{wildlife-monitoring-and-management}

Manned aircraft have been traditional used to undertake a variety of
ecological surveys. As remarked by most papers reviewed such techniques
are risky, costly and despite several efforts to minimize error
estimation (Cook and Jacobson 1979) are subject to visibility bias since
a greater number of observers is required to guarantee an exhaustive
count of populations. RPAS have emerged as a feasible alternative to
surpass such inconveniences. Several studies addressed counting large
terrestrial mammals with positive contributions (Jain 2013, Lancia et
al. (2005), Mulero-Pázmány et al. (2015)). (Colefax, Butcher, and
Kelaher 2017) conducted an extensive bibliographic revision to unveil
the potential of RPAS to monitor species relying on coastal and marine
ecosystems in placed on manned aircraft. In this context, (Hodgson,
Peel, and Kelly 2017) performed a comparative study using both RPAS and
traditional methods to survey humpback whales (\emph{Megaptera
novaeangliae}), while described a novel statistical method to estimate
marine mammal populations. RPAS have as well been applied to study
population dynamics in bird colonies (Sardà-Palomera et al. 2012), but
also in the inspection and characterization of inaccessible nesting
sites using multi-rotors (Weissensteiner, Poelstra, and Wolf 2015). In
some cases, in order to overcome the barriers to directly detect the
species of interest, the studies focused on locating and characterizing
their breeding and nesting areas (Andel et al. 2015, Szantoi et al.
(2017), Andrew and Shephard (2017), Wich et al. (2016)).

On the other hand, one of the central themes in ecology is the
development of surveys and statistical models for estimating abundance
and distribution of animals in wild populations (Lancia et al. 2005;
Mateo, Felicísimo, and Muñoz 2011). Such methods allow inferring the
potential or suitable habitat of organisms by collecting environmental
information and species presence data from different sources and
techniques. Wildlife telemetry tracking is one of the most common
methods used to gather movement data. (Mulero-Pázmány et al. 2015)
compared the performance of RPAS as tools for data collection against
biologgers when estimating spatial distribution of foraging domestic
herbivores impacting food availability in natural areas, easily
identifiable by high-resolution aerial images obtained by photographic
sensors on board. Targeting cattle (\emph{Bos taurus}), the authors
obtained similar results regarding the performance of the models, but
they emphasized the cost-benefit factor of RPAS as the main advantage.
In general, the relatively expensive purchase of electronic tracking
devices limits their availability for research purposes, reducing sample
size. Added to the risk of marking individuals under non-random
criteria, it is argued that robustness of the analysis can be seriously
affected. However, main advantage of wildlife telemetry is its ability
to provide a large amount of data for longer periods of time.
Nevertheless, the authors pointed out that both methodologies have the
potential to complement each other throughout all phases of the study.
Other innovative techniques have recently been illustrated in scientific
papers evaluating the feasibility of pairing radio locators in RPAS in
the search for individuals marked with VHF radio collars (Soriano,
Caballero, and Ollero 2009, Körner et al. (2010); Bayram et al. 2016;
Cliff et al. 2015; Leonardo et al. 2013).

Given the large amount of information generated, it is not surprising
that software have been developed in the field of computer vision and
machine learning to handle the automatic detection, recognition and
counting of individuals captured in scenes acquired by visible and
thermal-infrared sensors, replacing otherwise time-consuming manual
tasks (Lhoest et al. 2015; Abd-Elrahman, Pearlstine, and Percival 2005;
Gemert et al. 2015, Chabot and Francis (2016), Christiansen et al.
(2014)).

Outside the scientific literature, there are projects for monitoring
wildlife in both marine and terrestrial ecosystems, generally supported
by non-governmental organizations and research centers. Based on
information gathered at \url{https://conservationdrones.org} several
studies have been identified pursuing methods for registering
individuals in marine mammal populations, primates and macrofauna in
general. For instance, a work conducted in the Amazon Basin in Brazil is
testing the use of RPAS to improve the density and abundance estimation
of different species of dolphins, compared with direct observation by
specialists (Wich 2017). The main research aims include the validation
and harmonization of both methodologies and, indirectly, evaluate the
feasibility for its regular application in monitoring projects with a
similar purpose, taking into account the cost-benefit of the execution.

\newpage

\blandscape

\tiny

\begin{longtable}[]{@{}llllllll@{}}
\caption{WILDLIFE MONITORING AND MANAGEMENT}\tabularnewline
\toprule
\begin{minipage}[b]{0.11\columnwidth}\raggedright\strut
Study\strut
\end{minipage} & \begin{minipage}[b]{0.18\columnwidth}\raggedright\strut
Aims\strut
\end{minipage} & \begin{minipage}[b]{0.03\columnwidth}\raggedright\strut
Country\strut
\end{minipage} & \begin{minipage}[b]{0.14\columnwidth}\raggedright\strut
Place\strut
\end{minipage} & \begin{minipage}[b]{0.10\columnwidth}\raggedright\strut
Target\strut
\end{minipage} & \begin{minipage}[b]{0.09\columnwidth}\raggedright\strut
RPAS platform\strut
\end{minipage} & \begin{minipage}[b]{0.11\columnwidth}\raggedright\strut
Payload\strut
\end{minipage} & \begin{minipage}[b]{0.01\columnwidth}\raggedright\strut
Costs\strut
\end{minipage}\tabularnewline
\midrule
\endfirsthead
\toprule
\begin{minipage}[b]{0.11\columnwidth}\raggedright\strut
Study\strut
\end{minipage} & \begin{minipage}[b]{0.18\columnwidth}\raggedright\strut
Aims\strut
\end{minipage} & \begin{minipage}[b]{0.03\columnwidth}\raggedright\strut
Country\strut
\end{minipage} & \begin{minipage}[b]{0.14\columnwidth}\raggedright\strut
Place\strut
\end{minipage} & \begin{minipage}[b]{0.10\columnwidth}\raggedright\strut
Target\strut
\end{minipage} & \begin{minipage}[b]{0.09\columnwidth}\raggedright\strut
RPAS platform\strut
\end{minipage} & \begin{minipage}[b]{0.11\columnwidth}\raggedright\strut
Payload\strut
\end{minipage} & \begin{minipage}[b]{0.01\columnwidth}\raggedright\strut
Costs\strut
\end{minipage}\tabularnewline
\midrule
\endhead
\begin{minipage}[t]{0.11\columnwidth}\raggedright\strut
(Mulero-Pázmány et al. 2015)\strut
\end{minipage} & \begin{minipage}[t]{0.18\columnwidth}\raggedright\strut
Telemetry/RPAS SDM comparative study\strut
\end{minipage} & \begin{minipage}[t]{0.03\columnwidth}\raggedright\strut
Spain\strut
\end{minipage} & \begin{minipage}[t]{0.14\columnwidth}\raggedright\strut
Doñana N.P.\strut
\end{minipage} & \begin{minipage}[t]{0.10\columnwidth}\raggedright\strut
\emph{Bos taurus}\strut
\end{minipage} & \begin{minipage}[t]{0.09\columnwidth}\raggedright\strut
Fixed-wing: Easy Fly plane, Ikarus autopilot, Eagletree GPS logger\strut
\end{minipage} & \begin{minipage}[t]{0.11\columnwidth}\raggedright\strut
Panasonic Lumix LX-3 11MP\strut
\end{minipage} & \begin{minipage}[t]{0.01\columnwidth}\raggedright\strut
\$ 6500\strut
\end{minipage}\tabularnewline
\begin{minipage}[t]{0.11\columnwidth}\raggedright\strut
(Hodgson, Peel, and Kelly 2017)\strut
\end{minipage} & \begin{minipage}[t]{0.18\columnwidth}\raggedright\strut
Comparative survey RPAS/land based observation; abundance
estimation\strut
\end{minipage} & \begin{minipage}[t]{0.03\columnwidth}\raggedright\strut
Australia\strut
\end{minipage} & \begin{minipage}[t]{0.14\columnwidth}\raggedright\strut
North Stradbroke Island\strut
\end{minipage} & \begin{minipage}[t]{0.10\columnwidth}\raggedright\strut
humback whales\strut
\end{minipage} & \begin{minipage}[t]{0.09\columnwidth}\raggedright\strut
Fixed-wing: ScanEagle\strut
\end{minipage} & \begin{minipage}[t]{0.11\columnwidth}\raggedright\strut
Nikon D90 12MP, Standard Definition Electro-Optical Camera\strut
\end{minipage} & \begin{minipage}[t]{0.01\columnwidth}\raggedright\strut
?\strut
\end{minipage}\tabularnewline
\begin{minipage}[t]{0.11\columnwidth}\raggedright\strut
(Hodgson, Kelly, and Peel 2013)\strut
\end{minipage} & \begin{minipage}[t]{0.18\columnwidth}\raggedright\strut
Dugongs detection\strut
\end{minipage} & \begin{minipage}[t]{0.03\columnwidth}\raggedright\strut
Australia\strut
\end{minipage} & \begin{minipage}[t]{0.14\columnwidth}\raggedright\strut
Shark Bay Marine Park\strut
\end{minipage} & \begin{minipage}[t]{0.10\columnwidth}\raggedright\strut
Dugong\strut
\end{minipage} & \begin{minipage}[t]{0.09\columnwidth}\raggedright\strut
Fixed-wing: ScanEagle\strut
\end{minipage} & \begin{minipage}[t]{0.11\columnwidth}\raggedright\strut
Nikon D90 12MP\strut
\end{minipage} & \begin{minipage}[t]{0.01\columnwidth}\raggedright\strut
?\strut
\end{minipage}\tabularnewline
\begin{minipage}[t]{0.11\columnwidth}\raggedright\strut
(Wilson, Barr, and Zagorski 2017)\strut
\end{minipage} & \begin{minipage}[t]{0.18\columnwidth}\raggedright\strut
Bioacustic monitoring\strut
\end{minipage} & \begin{minipage}[t]{0.03\columnwidth}\raggedright\strut
USA\strut
\end{minipage} & \begin{minipage}[t]{0.14\columnwidth}\raggedright\strut
State Game Lands\strut
\end{minipage} & \begin{minipage}[t]{0.10\columnwidth}\raggedright\strut
Birds\strut
\end{minipage} & \begin{minipage}[t]{0.09\columnwidth}\raggedright\strut
Rotor-wing: DJI Phantom 2\strut
\end{minipage} & \begin{minipage}[t]{0.11\columnwidth}\raggedright\strut
ZOOM H1 Handy Recorder\strut
\end{minipage} & \begin{minipage}[t]{0.01\columnwidth}\raggedright\strut
?\strut
\end{minipage}\tabularnewline
\begin{minipage}[t]{0.11\columnwidth}\raggedright\strut
(Bayram et al. 2016)\strut
\end{minipage} & \begin{minipage}[t]{0.18\columnwidth}\raggedright\strut
VHF collars tracking\strut
\end{minipage} & \begin{minipage}[t]{0.03\columnwidth}\raggedright\strut
?\strut
\end{minipage} & \begin{minipage}[t]{0.14\columnwidth}\raggedright\strut
?\strut
\end{minipage} & \begin{minipage}[t]{0.10\columnwidth}\raggedright\strut
Bears (Ursus)\strut
\end{minipage} & \begin{minipage}[t]{0.09\columnwidth}\raggedright\strut
Rotor-wing: DJI F550\strut
\end{minipage} & \begin{minipage}[t]{0.11\columnwidth}\raggedright\strut
Telonics MOD-500 VHF, Uniden handheld scanner\strut
\end{minipage} & \begin{minipage}[t]{0.01\columnwidth}\raggedright\strut
?\strut
\end{minipage}\tabularnewline
\begin{minipage}[t]{0.11\columnwidth}\raggedright\strut
(Christie et al. 2016)\strut
\end{minipage} & \begin{minipage}[t]{0.18\columnwidth}\raggedright\strut
Abundance estimation\strut
\end{minipage} & \begin{minipage}[t]{0.03\columnwidth}\raggedright\strut
USA\strut
\end{minipage} & \begin{minipage}[t]{0.14\columnwidth}\raggedright\strut
Aleutian Islands\strut
\end{minipage} & \begin{minipage}[t]{0.10\columnwidth}\raggedright\strut
Steller Sea Lion (Eumetopias jubatus)\strut
\end{minipage} & \begin{minipage}[t]{0.09\columnwidth}\raggedright\strut
Rotor-wing: APH-22\strut
\end{minipage} & \begin{minipage}[t]{0.11\columnwidth}\raggedright\strut
?\strut
\end{minipage} & \begin{minipage}[t]{0.01\columnwidth}\raggedright\strut
\$ 25.000\strut
\end{minipage}\tabularnewline
\begin{minipage}[t]{0.11\columnwidth}\raggedright\strut
(Christie et al. 2016\})\strut
\end{minipage} & \begin{minipage}[t]{0.18\columnwidth}\raggedright\strut
Abundace estimation\strut
\end{minipage} & \begin{minipage}[t]{0.03\columnwidth}\raggedright\strut
USA\strut
\end{minipage} & \begin{minipage}[t]{0.14\columnwidth}\raggedright\strut
Monte Vista National Wildlife Refuge\strut
\end{minipage} & \begin{minipage}[t]{0.10\columnwidth}\raggedright\strut
Grus canadensis (sandhill cranes)\strut
\end{minipage} & \begin{minipage}[t]{0.09\columnwidth}\raggedright\strut
Fixed-wing: Raven RQ- 11A\strut
\end{minipage} & \begin{minipage}[t]{0.11\columnwidth}\raggedright\strut
?\strut
\end{minipage} & \begin{minipage}[t]{0.01\columnwidth}\raggedright\strut
\$ 400\strut
\end{minipage}\tabularnewline
\begin{minipage}[t]{0.11\columnwidth}\raggedright\strut
(Wich et al. 2016)\strut
\end{minipage} & \begin{minipage}[t]{0.18\columnwidth}\raggedright\strut
Sumatran orangutan nest detection\strut
\end{minipage} & \begin{minipage}[t]{0.03\columnwidth}\raggedright\strut
?\strut
\end{minipage} & \begin{minipage}[t]{0.14\columnwidth}\raggedright\strut
?\strut
\end{minipage} & \begin{minipage}[t]{0.10\columnwidth}\raggedright\strut
?\strut
\end{minipage} & \begin{minipage}[t]{0.09\columnwidth}\raggedright\strut
Fixed-wing: Skywalker 2013\strut
\end{minipage} & \begin{minipage}[t]{0.11\columnwidth}\raggedright\strut
Canon S100\strut
\end{minipage} & \begin{minipage}[t]{0.01\columnwidth}\raggedright\strut
?\strut
\end{minipage}\tabularnewline
\begin{minipage}[t]{0.11\columnwidth}\raggedright\strut
(Andel et al. 2015)\strut
\end{minipage} & \begin{minipage}[t]{0.18\columnwidth}\raggedright\strut
Chimpanzee nest detection\strut
\end{minipage} & \begin{minipage}[t]{0.03\columnwidth}\raggedright\strut
Africa\strut
\end{minipage} & \begin{minipage}[t]{0.14\columnwidth}\raggedright\strut
Loango National Park\strut
\end{minipage} & \begin{minipage}[t]{0.10\columnwidth}\raggedright\strut
Chimpanzee (Pan troglodytes)\strut
\end{minipage} & \begin{minipage}[t]{0.09\columnwidth}\raggedright\strut
Fixed-wing: Maja\strut
\end{minipage} & \begin{minipage}[t]{0.11\columnwidth}\raggedright\strut
Canon Powershot SX230 HS\strut
\end{minipage} & \begin{minipage}[t]{0.01\columnwidth}\raggedright\strut
\$ 5000\strut
\end{minipage}\tabularnewline
\begin{minipage}[t]{0.11\columnwidth}\raggedright\strut
(Koski et al. 2009)\strut
\end{minipage} & \begin{minipage}[t]{0.18\columnwidth}\raggedright\strut
Marine mammals monitoring\strut
\end{minipage} & \begin{minipage}[t]{0.03\columnwidth}\raggedright\strut
USA\strut
\end{minipage} & \begin{minipage}[t]{0.14\columnwidth}\raggedright\strut
Admiralty Bay\strut
\end{minipage} & \begin{minipage}[t]{0.10\columnwidth}\raggedright\strut
Marine mammals\strut
\end{minipage} & \begin{minipage}[t]{0.09\columnwidth}\raggedright\strut
Fixed-wing: ScanEagle\strut
\end{minipage} & \begin{minipage}[t]{0.11\columnwidth}\raggedright\strut
NTSC Video Camera\strut
\end{minipage} & \begin{minipage}[t]{0.01\columnwidth}\raggedright\strut
?\strut
\end{minipage}\tabularnewline
\begin{minipage}[t]{0.11\columnwidth}\raggedright\strut
(Andrew and Shephard 2017)\strut
\end{minipage} & \begin{minipage}[t]{0.18\columnwidth}\raggedright\strut
Semi-automated image processing tools to detect and map sea eagle
nests\strut
\end{minipage} & \begin{minipage}[t]{0.03\columnwidth}\raggedright\strut
Australia\strut
\end{minipage} & \begin{minipage}[t]{0.14\columnwidth}\raggedright\strut
Houtman Abrolhos Islands\strut
\end{minipage} & \begin{minipage}[t]{0.10\columnwidth}\raggedright\strut
White-bellied sea eagle (Haliaeetus leucogaster)\strut
\end{minipage} & \begin{minipage}[t]{0.09\columnwidth}\raggedright\strut
?\strut
\end{minipage} & \begin{minipage}[t]{0.11\columnwidth}\raggedright\strut
?\strut
\end{minipage} & \begin{minipage}[t]{0.01\columnwidth}\raggedright\strut
?\strut
\end{minipage}\tabularnewline
\begin{minipage}[t]{0.11\columnwidth}\raggedright\strut
(Longmore et al. 2017)\strut
\end{minipage} & \begin{minipage}[t]{0.18\columnwidth}\raggedright\strut
Software development to help detect animals in thermal images\strut
\end{minipage} & \begin{minipage}[t]{0.03\columnwidth}\raggedright\strut
UK\strut
\end{minipage} & \begin{minipage}[t]{0.14\columnwidth}\raggedright\strut
Arrowe Brook Farm Wirral\strut
\end{minipage} & \begin{minipage}[t]{0.10\columnwidth}\raggedright\strut
Wildlife\strut
\end{minipage} & \begin{minipage}[t]{0.09\columnwidth}\raggedright\strut
Rotor-wing: 3DR robotics Y6\strut
\end{minipage} & \begin{minipage}[t]{0.11\columnwidth}\raggedright\strut
FLIR, Tau 2 LWIR Thermal Imaging Camera Core\strut
\end{minipage} & \begin{minipage}[t]{0.01\columnwidth}\raggedright\strut
?\strut
\end{minipage}\tabularnewline
\begin{minipage}[t]{0.11\columnwidth}\raggedright\strut
(Martin et al. 2012)\strut
\end{minipage} & \begin{minipage}[t]{0.18\columnwidth}\raggedright\strut
Estimate the distribution of organisms using statistical models\strut
\end{minipage} & \begin{minipage}[t]{0.03\columnwidth}\raggedright\strut
USA\strut
\end{minipage} & \begin{minipage}[t]{0.14\columnwidth}\raggedright\strut
?\strut
\end{minipage} & \begin{minipage}[t]{0.10\columnwidth}\raggedright\strut
Manatee (\emph{Trichechus manatus latirostris})\strut
\end{minipage} & \begin{minipage}[t]{0.09\columnwidth}\raggedright\strut
Fixed-wing: Nova 2.1\strut
\end{minipage} & \begin{minipage}[t]{0.11\columnwidth}\raggedright\strut
Olympus H E-420\strut
\end{minipage} & \begin{minipage}[t]{0.01\columnwidth}\raggedright\strut
?\strut
\end{minipage}\tabularnewline
\bottomrule
\end{longtable}

\begin{longtable}[]{@{}llllllll@{}}
\caption{MONITORING OF TERRESTRIAL AND AQUATIC
ECOSYSTEMS}\tabularnewline
\toprule
\begin{minipage}[b]{0.11\columnwidth}\raggedright\strut
Study\strut
\end{minipage} & \begin{minipage}[b]{0.18\columnwidth}\raggedright\strut
Aims\strut
\end{minipage} & \begin{minipage}[b]{0.03\columnwidth}\raggedright\strut
Country\strut
\end{minipage} & \begin{minipage}[b]{0.14\columnwidth}\raggedright\strut
Place\strut
\end{minipage} & \begin{minipage}[b]{0.10\columnwidth}\raggedright\strut
Target\strut
\end{minipage} & \begin{minipage}[b]{0.09\columnwidth}\raggedright\strut
RPAS platform\strut
\end{minipage} & \begin{minipage}[b]{0.11\columnwidth}\raggedright\strut
Payload\strut
\end{minipage} & \begin{minipage}[b]{0.01\columnwidth}\raggedright\strut
Costs\strut
\end{minipage}\tabularnewline
\midrule
\endfirsthead
\toprule
\begin{minipage}[b]{0.11\columnwidth}\raggedright\strut
Study\strut
\end{minipage} & \begin{minipage}[b]{0.18\columnwidth}\raggedright\strut
Aims\strut
\end{minipage} & \begin{minipage}[b]{0.03\columnwidth}\raggedright\strut
Country\strut
\end{minipage} & \begin{minipage}[b]{0.14\columnwidth}\raggedright\strut
Place\strut
\end{minipage} & \begin{minipage}[b]{0.10\columnwidth}\raggedright\strut
Target\strut
\end{minipage} & \begin{minipage}[b]{0.09\columnwidth}\raggedright\strut
RPAS platform\strut
\end{minipage} & \begin{minipage}[b]{0.11\columnwidth}\raggedright\strut
Payload\strut
\end{minipage} & \begin{minipage}[b]{0.01\columnwidth}\raggedright\strut
Costs\strut
\end{minipage}\tabularnewline
\midrule
\endhead
\begin{minipage}[t]{0.11\columnwidth}\raggedright\strut
(Perroy, Sullivan, and Stephenson 2017)\strut
\end{minipage} & \begin{minipage}[t]{0.18\columnwidth}\raggedright\strut
Tropical invasive plants\strut
\end{minipage} & \begin{minipage}[t]{0.03\columnwidth}\raggedright\strut
USA\strut
\end{minipage} & \begin{minipage}[t]{0.14\columnwidth}\raggedright\strut
Pahoa, Hawai\strut
\end{minipage} & \begin{minipage}[t]{0.10\columnwidth}\raggedright\strut
\emph{Miconia calvescens}\strut
\end{minipage} & \begin{minipage}[t]{0.09\columnwidth}\raggedright\strut
Rotor-wing: DJ Inspire-1\strut
\end{minipage} & \begin{minipage}[t]{0.11\columnwidth}\raggedright\strut
DJI FC350 camera\strut
\end{minipage} & \begin{minipage}[t]{0.01\columnwidth}\raggedright\strut
?\strut
\end{minipage}\tabularnewline
\begin{minipage}[t]{0.11\columnwidth}\raggedright\strut
(Szantoi et al. 2017)\strut
\end{minipage} & \begin{minipage}[t]{0.18\columnwidth}\raggedright\strut
Habitat Mapping\strut
\end{minipage} & \begin{minipage}[t]{0.03\columnwidth}\raggedright\strut
Indonesia\strut
\end{minipage} & \begin{minipage}[t]{0.14\columnwidth}\raggedright\strut
Gunung Leuser National Park\strut
\end{minipage} & \begin{minipage}[t]{0.10\columnwidth}\raggedright\strut
Orangutan (Pongo abelii)\strut
\end{minipage} & \begin{minipage}[t]{0.09\columnwidth}\raggedright\strut
Fixed-wing: Skywalker\strut
\end{minipage} & \begin{minipage}[t]{0.11\columnwidth}\raggedright\strut
Canon S100\strut
\end{minipage} & \begin{minipage}[t]{0.01\columnwidth}\raggedright\strut
\$ 4000\strut
\end{minipage}\tabularnewline
\begin{minipage}[t]{0.11\columnwidth}\raggedright\strut
(Casella et al. 2017)\strut
\end{minipage} & \begin{minipage}[t]{0.18\columnwidth}\raggedright\strut
Coral reef mapping\strut
\end{minipage} & \begin{minipage}[t]{0.03\columnwidth}\raggedright\strut
French Polynesia\strut
\end{minipage} & \begin{minipage}[t]{0.14\columnwidth}\raggedright\strut
Tiahura,; Moorea\strut
\end{minipage} & \begin{minipage}[t]{0.10\columnwidth}\raggedright\strut
Coral reef\strut
\end{minipage} & \begin{minipage}[t]{0.09\columnwidth}\raggedright\strut
Rotor-wing: DJI Phantom 2\strut
\end{minipage} & \begin{minipage}[t]{0.11\columnwidth}\raggedright\strut
Modified GoPro HERO4\strut
\end{minipage} & \begin{minipage}[t]{0.01\columnwidth}\raggedright\strut
\$ 1678\strut
\end{minipage}\tabularnewline
\begin{minipage}[t]{0.11\columnwidth}\raggedright\strut
(Casella et al. 2016)\strut
\end{minipage} & \begin{minipage}[t]{0.18\columnwidth}\raggedright\strut
Monitoring coastal erosion dynamics in shorelines\strut
\end{minipage} & \begin{minipage}[t]{0.03\columnwidth}\raggedright\strut
French Polynesia\strut
\end{minipage} & \begin{minipage}[t]{0.14\columnwidth}\raggedright\strut
Tiahura; Moorea\strut
\end{minipage} & \begin{minipage}[t]{0.10\columnwidth}\raggedright\strut
Coral reef\strut
\end{minipage} & \begin{minipage}[t]{0.09\columnwidth}\raggedright\strut
Rotor-wing: Mikrokopter Okto XL\strut
\end{minipage} & \begin{minipage}[t]{0.11\columnwidth}\raggedright\strut
Canon G11\strut
\end{minipage} & \begin{minipage}[t]{0.01\columnwidth}\raggedright\strut
\$ 7500\strut
\end{minipage}\tabularnewline
\begin{minipage}[t]{0.11\columnwidth}\raggedright\strut
(Müllerová et al. 2016)\strut
\end{minipage} & \begin{minipage}[t]{0.18\columnwidth}\raggedright\strut
Monitoring plant invasion\strut
\end{minipage} & \begin{minipage}[t]{0.03\columnwidth}\raggedright\strut
?\strut
\end{minipage} & \begin{minipage}[t]{0.14\columnwidth}\raggedright\strut
?\strut
\end{minipage} & \begin{minipage}[t]{0.10\columnwidth}\raggedright\strut
Exotic species\strut
\end{minipage} & \begin{minipage}[t]{0.09\columnwidth}\raggedright\strut
Fixed-wing: VUT 712 713 720\strut
\end{minipage} & \begin{minipage}[t]{0.11\columnwidth}\raggedright\strut
Canon S100\strut
\end{minipage} & \begin{minipage}[t]{0.01\columnwidth}\raggedright\strut
?\strut
\end{minipage}\tabularnewline
\begin{minipage}[t]{0.11\columnwidth}\raggedright\strut
(Ventura et al. 2016)\strut
\end{minipage} & \begin{minipage}[t]{0.18\columnwidth}\raggedright\strut
Marine fish nursery areas mapping\strut
\end{minipage} & \begin{minipage}[t]{0.03\columnwidth}\raggedright\strut
Italy\strut
\end{minipage} & \begin{minipage}[t]{0.14\columnwidth}\raggedright\strut
Giglio Island\strut
\end{minipage} & \begin{minipage}[t]{0.10\columnwidth}\raggedright\strut
Marine fish nursery areas\strut
\end{minipage} & \begin{minipage}[t]{0.09\columnwidth}\raggedright\strut
Rotor-wing: homemade prototype\strut
\end{minipage} & \begin{minipage}[t]{0.11\columnwidth}\raggedright\strut
Mobius HD, GoPro HERO3 Black Edition\strut
\end{minipage} & \begin{minipage}[t]{0.01\columnwidth}\raggedright\strut
\$ 100\strut
\end{minipage}\tabularnewline
\begin{minipage}[t]{0.11\columnwidth}\raggedright\strut
(Ivošević et al. 2015)\strut
\end{minipage} & \begin{minipage}[t]{0.18\columnwidth}\raggedright\strut
Habitat monitoring and modeling in restricted areas; RPAS performance
test \& South Korea \& Chiaksan National Park;Taeanhaean National
Park\strut
\end{minipage} & \begin{minipage}[t]{0.03\columnwidth}\raggedright\strut
South Korea\strut
\end{minipage} & \begin{minipage}[t]{0.14\columnwidth}\raggedright\strut
Chiaksan National Park;Taeanhaean National Park\strut
\end{minipage} & \begin{minipage}[t]{0.10\columnwidth}\raggedright\strut
\strut
\end{minipage} & \begin{minipage}[t]{0.09\columnwidth}\raggedright\strut
Rotor-wing: DJI Phantom 2 Vision+\strut
\end{minipage} & \begin{minipage}[t]{0.11\columnwidth}\raggedright\strut
Full HD videos 1080p/30fps and 720p/60fps\strut
\end{minipage} & \begin{minipage}[t]{0.01\columnwidth}\raggedright\strut
?\strut
\end{minipage}\tabularnewline
\begin{minipage}[t]{0.11\columnwidth}\raggedright\strut
(Lisein et al. 2015)\strut
\end{minipage} & \begin{minipage}[t]{0.18\columnwidth}\raggedright\strut
Discrimination of deciduous species; Forest inventory\strut
\end{minipage} & \begin{minipage}[t]{0.03\columnwidth}\raggedright\strut
Belgium\strut
\end{minipage} & \begin{minipage}[t]{0.14\columnwidth}\raggedright\strut
Grand-Leez\strut
\end{minipage} & \begin{minipage}[t]{0.10\columnwidth}\raggedright\strut
English oak, birches, sycamore maple ,common ash and poplars\strut
\end{minipage} & \begin{minipage}[t]{0.09\columnwidth}\raggedright\strut
Fixed-wings: Gatewing X100\strut
\end{minipage} & \begin{minipage}[t]{0.11\columnwidth}\raggedright\strut
Ricoh GR2 GR3 GR4 10 megapixels CCD\strut
\end{minipage} & \begin{minipage}[t]{0.01\columnwidth}\raggedright\strut
?\strut
\end{minipage}\tabularnewline
\begin{minipage}[t]{0.11\columnwidth}\raggedright\strut
(Puttock et al. 2015)\strut
\end{minipage} & \begin{minipage}[t]{0.18\columnwidth}\raggedright\strut
Characterization of ecosystems affected by beaver activity\strut
\end{minipage} & \begin{minipage}[t]{0.03\columnwidth}\raggedright\strut
UK\strut
\end{minipage} & \begin{minipage}[t]{0.14\columnwidth}\raggedright\strut
Devon Beaver Project site\strut
\end{minipage} & \begin{minipage}[t]{0.10\columnwidth}\raggedright\strut
Eurasian beaver (\emph{Castor fiber})\strut
\end{minipage} & \begin{minipage}[t]{0.09\columnwidth}\raggedright\strut
Rotor-wing: 3D Robotics Y6\strut
\end{minipage} & \begin{minipage}[t]{0.11\columnwidth}\raggedright\strut
anon ELPH 520 HS\strut
\end{minipage} & \begin{minipage}[t]{0.01\columnwidth}\raggedright\strut
?\strut
\end{minipage}\tabularnewline
\begin{minipage}[t]{0.11\columnwidth}\raggedright\strut
(Zahawi et al. 2015)\strut
\end{minipage} & \begin{minipage}[t]{0.18\columnwidth}\raggedright\strut
Characterization of tropical forest structure for restoration
actions\strut
\end{minipage} & \begin{minipage}[t]{0.03\columnwidth}\raggedright\strut
Costa Rica\strut
\end{minipage} & \begin{minipage}[t]{0.14\columnwidth}\raggedright\strut
Devon Beaver Project site\strut
\end{minipage} & \begin{minipage}[t]{0.10\columnwidth}\raggedright\strut
Several species\strut
\end{minipage} & \begin{minipage}[t]{0.09\columnwidth}\raggedright\strut
Rotor-wing: 3D Robotics Y6\strut
\end{minipage} & \begin{minipage}[t]{0.11\columnwidth}\raggedright\strut
Canon S100\strut
\end{minipage} & \begin{minipage}[t]{0.01\columnwidth}\raggedright\strut
\$ 1500\strut
\end{minipage}\tabularnewline
\begin{minipage}[t]{0.11\columnwidth}\raggedright\strut
(Bustamante 2015)\strut
\end{minipage} & \begin{minipage}[t]{0.18\columnwidth}\raggedright\strut
Forest monitoring\strut
\end{minipage} & \begin{minipage}[t]{0.03\columnwidth}\raggedright\strut
Brasil\strut
\end{minipage} & \begin{minipage}[t]{0.14\columnwidth}\raggedright\strut
Riverine Forests (Permanent Protected Areas), Rio de Janeiro, Barrãcao
do Mendes, Santa Cruz and São Lorenço\strut
\end{minipage} & \begin{minipage}[t]{0.10\columnwidth}\raggedright\strut
Riverbank forests\strut
\end{minipage} & \begin{minipage}[t]{0.09\columnwidth}\raggedright\strut
Rotor-wing: DJI Phantom Vision 2S\strut
\end{minipage} & \begin{minipage}[t]{0.11\columnwidth}\raggedright\strut
RGB digital camera 14MP\strut
\end{minipage} & \begin{minipage}[t]{0.01\columnwidth}\raggedright\strut
\$ 9700\strut
\end{minipage}\tabularnewline
\begin{minipage}[t]{0.11\columnwidth}\raggedright\strut
(Gini et al. 2012)\strut
\end{minipage} & \begin{minipage}[t]{0.18\columnwidth}\raggedright\strut
3D modeling and classification of tree species\strut
\end{minipage} & \begin{minipage}[t]{0.03\columnwidth}\raggedright\strut
Italy\strut
\end{minipage} & \begin{minipage}[t]{0.14\columnwidth}\raggedright\strut
Parco Adda Nord\strut
\end{minipage} & \begin{minipage}[t]{0.10\columnwidth}\raggedright\strut
Several species\strut
\end{minipage} & \begin{minipage}[t]{0.09\columnwidth}\raggedright\strut
Rotor-wing: Microdrones TM MD4-200\strut
\end{minipage} & \begin{minipage}[t]{0.11\columnwidth}\raggedright\strut
RGB CCD 12 megapixels Pentax Optio A40; modified NIR Sigma DP1 with a
Foveon X3 sensor\strut
\end{minipage} & \begin{minipage}[t]{0.01\columnwidth}\raggedright\strut
?\strut
\end{minipage}\tabularnewline
\begin{minipage}[t]{0.11\columnwidth}\raggedright\strut
(Miyamoto et al. 2004)\strut
\end{minipage} & \begin{minipage}[t]{0.18\columnwidth}\raggedright\strut
Classification of species in wetlands\strut
\end{minipage} & \begin{minipage}[t]{0.03\columnwidth}\raggedright\strut
Japan\strut
\end{minipage} & \begin{minipage}[t]{0.14\columnwidth}\raggedright\strut
Kushiro Wetlands\strut
\end{minipage} & \begin{minipage}[t]{0.10\columnwidth}\raggedright\strut
Several species\strut
\end{minipage} & \begin{minipage}[t]{0.09\columnwidth}\raggedright\strut
Helium balloon\strut
\end{minipage} & \begin{minipage}[t]{0.11\columnwidth}\raggedright\strut
NIKON F-801, NIKKOR 28 mm f/2.8\strut
\end{minipage} & \begin{minipage}[t]{0.01\columnwidth}\raggedright\strut
\$ 1600\strut
\end{minipage}\tabularnewline
\begin{minipage}[t]{0.11\columnwidth}\raggedright\strut
(Casella et al. 2017)\strut
\end{minipage} & \begin{minipage}[t]{0.18\columnwidth}\raggedright\strut
Mapping coral reefs\strut
\end{minipage} & \begin{minipage}[t]{0.03\columnwidth}\raggedright\strut
?\strut
\end{minipage} & \begin{minipage}[t]{0.14\columnwidth}\raggedright\strut
?\strut
\end{minipage} & \begin{minipage}[t]{0.10\columnwidth}\raggedright\strut
?\strut
\end{minipage} & \begin{minipage}[t]{0.09\columnwidth}\raggedright\strut
Rotor.wing: DJI Phantom 2\strut
\end{minipage} & \begin{minipage}[t]{0.11\columnwidth}\raggedright\strut
GoPro HERO4\strut
\end{minipage} & \begin{minipage}[t]{0.01\columnwidth}\raggedright\strut
?\strut
\end{minipage}\tabularnewline
\bottomrule
\end{longtable}

\begin{longtable}[]{@{}llllllll@{}}
\caption{INFRASTRUCTURES AND RISK ASSESSMENT, ECOTOURISM, IMPACT ON
WILDLIFE AND ECOSYSTEMS}\tabularnewline
\toprule
\begin{minipage}[b]{0.11\columnwidth}\raggedright\strut
Study\strut
\end{minipage} & \begin{minipage}[b]{0.18\columnwidth}\raggedright\strut
Aims\strut
\end{minipage} & \begin{minipage}[b]{0.03\columnwidth}\raggedright\strut
Country\strut
\end{minipage} & \begin{minipage}[b]{0.14\columnwidth}\raggedright\strut
Place\strut
\end{minipage} & \begin{minipage}[b]{0.10\columnwidth}\raggedright\strut
Target\strut
\end{minipage} & \begin{minipage}[b]{0.09\columnwidth}\raggedright\strut
RPAS platform\strut
\end{minipage} & \begin{minipage}[b]{0.11\columnwidth}\raggedright\strut
Payload\strut
\end{minipage} & \begin{minipage}[b]{0.01\columnwidth}\raggedright\strut
Costs\strut
\end{minipage}\tabularnewline
\midrule
\endfirsthead
\toprule
\begin{minipage}[b]{0.11\columnwidth}\raggedright\strut
Study\strut
\end{minipage} & \begin{minipage}[b]{0.18\columnwidth}\raggedright\strut
Aims\strut
\end{minipage} & \begin{minipage}[b]{0.03\columnwidth}\raggedright\strut
Country\strut
\end{minipage} & \begin{minipage}[b]{0.14\columnwidth}\raggedright\strut
Place\strut
\end{minipage} & \begin{minipage}[b]{0.10\columnwidth}\raggedright\strut
Target\strut
\end{minipage} & \begin{minipage}[b]{0.09\columnwidth}\raggedright\strut
RPAS platform\strut
\end{minipage} & \begin{minipage}[b]{0.11\columnwidth}\raggedright\strut
Payload\strut
\end{minipage} & \begin{minipage}[b]{0.01\columnwidth}\raggedright\strut
Costs\strut
\end{minipage}\tabularnewline
\midrule
\endhead
\begin{minipage}[t]{0.11\columnwidth}\raggedright\strut
(Lobermeier et al. 2015)\strut
\end{minipage} & \begin{minipage}[t]{0.18\columnwidth}\raggedright\strut
Mitigate the risk of collision by installing markers on electrical
lines\strut
\end{minipage} & \begin{minipage}[t]{0.03\columnwidth}\raggedright\strut
USA\strut
\end{minipage} & \begin{minipage}[t]{0.14\columnwidth}\raggedright\strut
?\strut
\end{minipage} & \begin{minipage}[t]{0.10\columnwidth}\raggedright\strut
Birds\strut
\end{minipage} & \begin{minipage}[t]{0.09\columnwidth}\raggedright\strut
Rotor-wing: Mikrokopter Hexa XL\strut
\end{minipage} & \begin{minipage}[t]{0.11\columnwidth}\raggedright\strut
KX 171 Microcam\strut
\end{minipage} & \begin{minipage}[t]{0.01\columnwidth}\raggedright\strut
?\strut
\end{minipage}\tabularnewline
\begin{minipage}[t]{0.11\columnwidth}\raggedright\strut
(Margarita Mulero-Pázmány 2014\})\strut
\end{minipage} & \begin{minipage}[t]{0.18\columnwidth}\raggedright\strut
Bird risk hazards in power lines\strut
\end{minipage} & \begin{minipage}[t]{0.03\columnwidth}\raggedright\strut
Spain\strut
\end{minipage} & \begin{minipage}[t]{0.14\columnwidth}\raggedright\strut
Doñana National Park\strut
\end{minipage} & \begin{minipage}[t]{0.10\columnwidth}\raggedright\strut
Birds\strut
\end{minipage} & \begin{minipage}[t]{0.09\columnwidth}\raggedright\strut
ixed-wing: Easy fly St-330\strut
\end{minipage} & \begin{minipage}[t]{0.11\columnwidth}\raggedright\strut
GoPro HERO2 11MP;Panasonic LX3 11MP\strut
\end{minipage} & \begin{minipage}[t]{0.01\columnwidth}\raggedright\strut
\$ 8863\strut
\end{minipage}\tabularnewline
\begin{minipage}[t]{0.11\columnwidth}\raggedright\strut
(Mulero-Pázmány et al. 2014)\strut
\end{minipage} & \begin{minipage}[t]{0.18\columnwidth}\raggedright\strut
Anti-poaching\strut
\end{minipage} & \begin{minipage}[t]{0.03\columnwidth}\raggedright\strut
Africa\strut
\end{minipage} & \begin{minipage}[t]{0.14\columnwidth}\raggedright\strut
KwaZulu-Nata\strut
\end{minipage} & \begin{minipage}[t]{0.10\columnwidth}\raggedright\strut
Black rhinocero (\emph{Diceros bicornis}), white rhinocero
(\emph{Ceratotherium simum})\strut
\end{minipage} & \begin{minipage}[t]{0.09\columnwidth}\raggedright\strut
Fixed-wing: Easy Fly St-330\strut
\end{minipage} & \begin{minipage}[t]{0.11\columnwidth}\raggedright\strut
Panasonic Lumix LX-3 11 MP, GoPro Hero2; Thermoteknix Micro CAM
microbolometer\strut
\end{minipage} & \begin{minipage}[t]{0.01\columnwidth}\raggedright\strut
\$ 15700\strut
\end{minipage}\tabularnewline
\begin{minipage}[t]{0.11\columnwidth}\raggedright\strut
(Hansen 2016)\strut
\end{minipage} & \begin{minipage}[t]{0.18\columnwidth}\raggedright\strut
Visitors Surveillance\strut
\end{minipage} & \begin{minipage}[t]{0.03\columnwidth}\raggedright\strut
Sweden\strut
\end{minipage} & \begin{minipage}[t]{0.14\columnwidth}\raggedright\strut
Sweden \& Kosterhavet National Park\strut
\end{minipage} & \begin{minipage}[t]{0.10\columnwidth}\raggedright\strut
Humans\strut
\end{minipage} & \begin{minipage}[t]{0.09\columnwidth}\raggedright\strut
?\strut
\end{minipage} & \begin{minipage}[t]{0.11\columnwidth}\raggedright\strut
?\strut
\end{minipage} & \begin{minipage}[t]{0.01\columnwidth}\raggedright\strut
?\strut
\end{minipage}\tabularnewline
\begin{minipage}[t]{0.11\columnwidth}\raggedright\strut
(Sabella et al. 2017)\strut
\end{minipage} & \begin{minipage}[t]{0.18\columnwidth}\raggedright\strut
Visitors Surveillance\strut
\end{minipage} & \begin{minipage}[t]{0.03\columnwidth}\raggedright\strut
Italy\strut
\end{minipage} & \begin{minipage}[t]{0.14\columnwidth}\raggedright\strut
R.N.O. Oasi faunistica di Vendicari\strut
\end{minipage} & \begin{minipage}[t]{0.10\columnwidth}\raggedright\strut
Humans\strut
\end{minipage} & \begin{minipage}[t]{0.09\columnwidth}\raggedright\strut
Rotor-wing: DJI Phantom 3\strut
\end{minipage} & \begin{minipage}[t]{0.11\columnwidth}\raggedright\strut
?\strut
\end{minipage} & \begin{minipage}[t]{0.01\columnwidth}\raggedright\strut
?\strut
\end{minipage}\tabularnewline
\begin{minipage}[t]{0.11\columnwidth}\raggedright\strut
(King 2014)\strut
\end{minipage} & \begin{minipage}[t]{0.18\columnwidth}\raggedright\strut
RPAS applications in ecotourism activities\strut
\end{minipage} & \begin{minipage}[t]{0.03\columnwidth}\raggedright\strut
Sweeden\strut
\end{minipage} & \begin{minipage}[t]{0.14\columnwidth}\raggedright\strut
Sweeden \& Kosterhavet National Park\strut
\end{minipage} & \begin{minipage}[t]{0.10\columnwidth}\raggedright\strut
Humans\strut
\end{minipage} & \begin{minipage}[t]{0.09\columnwidth}\raggedright\strut
?\strut
\end{minipage} & \begin{minipage}[t]{0.11\columnwidth}\raggedright\strut
?\strut
\end{minipage} & \begin{minipage}[t]{0.01\columnwidth}\raggedright\strut
?\strut
\end{minipage}\tabularnewline
\begin{minipage}[t]{0.11\columnwidth}\raggedright\strut
(Vas et al. 2015)\strut
\end{minipage} & \begin{minipage}[t]{0.18\columnwidth}\raggedright\strut
RPAS impact\strut
\end{minipage} & \begin{minipage}[t]{0.03\columnwidth}\raggedright\strut
France\strut
\end{minipage} & \begin{minipage}[t]{0.14\columnwidth}\raggedright\strut
Zoo du Lunaret, Cros Martin Natural Area\strut
\end{minipage} & \begin{minipage}[t]{0.10\columnwidth}\raggedright\strut
\emph{Anas platyrhyncho}, \emph{Phoenicopterus roseus}, \emph{Tringa
nebularia}\strut
\end{minipage} & \begin{minipage}[t]{0.09\columnwidth}\raggedright\strut
Rotor-wing: Phantom\strut
\end{minipage} & \begin{minipage}[t]{0.11\columnwidth}\raggedright\strut
GoPro HERO3\strut
\end{minipage} & \begin{minipage}[t]{0.01\columnwidth}\raggedright\strut
?\strut
\end{minipage}\tabularnewline
\begin{minipage}[t]{0.11\columnwidth}\raggedright\strut
(Weissensteiner, Poelstra, and Wolf 2015)\strut
\end{minipage} & \begin{minipage}[t]{0.18\columnwidth}\raggedright\strut
RPAS Impact\strut
\end{minipage} & \begin{minipage}[t]{0.03\columnwidth}\raggedright\strut
Sweeden\strut
\end{minipage} & \begin{minipage}[t]{0.14\columnwidth}\raggedright\strut
?\strut
\end{minipage} & \begin{minipage}[t]{0.10\columnwidth}\raggedright\strut
Hooded crow (\emph{Corvus corone cornix})\strut
\end{minipage} & \begin{minipage}[t]{0.09\columnwidth}\raggedright\strut
Rotor-wing: DJI Phantom 2 Vision\strut
\end{minipage} & \begin{minipage}[t]{0.11\columnwidth}\raggedright\strut
?\strut
\end{minipage} & \begin{minipage}[t]{0.01\columnwidth}\raggedright\strut
\$ 1000\strut
\end{minipage}\tabularnewline
\bottomrule
\end{longtable}

\begin{longtable}[]{@{}llllllll@{}}
\caption{ENVIRONMENTAL MONITORING AND DECISION SUPPORT}\tabularnewline
\toprule
\begin{minipage}[b]{0.11\columnwidth}\raggedright\strut
Study\strut
\end{minipage} & \begin{minipage}[b]{0.18\columnwidth}\raggedright\strut
Aims\strut
\end{minipage} & \begin{minipage}[b]{0.03\columnwidth}\raggedright\strut
Country\strut
\end{minipage} & \begin{minipage}[b]{0.14\columnwidth}\raggedright\strut
Place\strut
\end{minipage} & \begin{minipage}[b]{0.10\columnwidth}\raggedright\strut
Target\strut
\end{minipage} & \begin{minipage}[b]{0.09\columnwidth}\raggedright\strut
RPAS platform\strut
\end{minipage} & \begin{minipage}[b]{0.11\columnwidth}\raggedright\strut
Payload\strut
\end{minipage} & \begin{minipage}[b]{0.01\columnwidth}\raggedright\strut
Costs\strut
\end{minipage}\tabularnewline
\midrule
\endfirsthead
\toprule
\begin{minipage}[b]{0.11\columnwidth}\raggedright\strut
Study\strut
\end{minipage} & \begin{minipage}[b]{0.18\columnwidth}\raggedright\strut
Aims\strut
\end{minipage} & \begin{minipage}[b]{0.03\columnwidth}\raggedright\strut
Country\strut
\end{minipage} & \begin{minipage}[b]{0.14\columnwidth}\raggedright\strut
Place\strut
\end{minipage} & \begin{minipage}[b]{0.10\columnwidth}\raggedright\strut
Target\strut
\end{minipage} & \begin{minipage}[b]{0.09\columnwidth}\raggedright\strut
RPAS platform\strut
\end{minipage} & \begin{minipage}[b]{0.11\columnwidth}\raggedright\strut
Payload\strut
\end{minipage} & \begin{minipage}[b]{0.01\columnwidth}\raggedright\strut
Costs\strut
\end{minipage}\tabularnewline
\midrule
\endhead
\begin{minipage}[t]{0.11\columnwidth}\raggedright\strut
(Zang et al. 2012)\strut
\end{minipage} & \begin{minipage}[t]{0.18\columnwidth}\raggedright\strut
Pollution monitoring\strut
\end{minipage} & \begin{minipage}[t]{0.03\columnwidth}\raggedright\strut
China\strut
\end{minipage} & \begin{minipage}[t]{0.14\columnwidth}\raggedright\strut
Several cities\strut
\end{minipage} & \begin{minipage}[t]{0.10\columnwidth}\raggedright\strut
Rivers\strut
\end{minipage} & \begin{minipage}[t]{0.09\columnwidth}\raggedright\strut
Fixed-wing\strut
\end{minipage} & \begin{minipage}[t]{0.11\columnwidth}\raggedright\strut
Canon 50D, ACD multispectral camera\strut
\end{minipage} & \begin{minipage}[t]{0.01\columnwidth}\raggedright\strut
?\strut
\end{minipage}\tabularnewline
\begin{minipage}[t]{0.11\columnwidth}\raggedright\strut
(Cornell, Herman, and Ontiveros 2016)\strut
\end{minipage} & \begin{minipage}[t]{0.18\columnwidth}\raggedright\strut
Water sampling\strut
\end{minipage} & \begin{minipage}[t]{0.03\columnwidth}\raggedright\strut
USA\strut
\end{minipage} & \begin{minipage}[t]{0.14\columnwidth}\raggedright\strut
Lake Ontari0\strut
\end{minipage} & \begin{minipage}[t]{0.10\columnwidth}\raggedright\strut
?\strut
\end{minipage} & \begin{minipage}[t]{0.09\columnwidth}\raggedright\strut
Rotor-wing: DJI Phantom 3\strut
\end{minipage} & \begin{minipage}[t]{0.11\columnwidth}\raggedright\strut
50mL Falcon tube\strut
\end{minipage} & \begin{minipage}[t]{0.01\columnwidth}\raggedright\strut
?\strut
\end{minipage}\tabularnewline
\begin{minipage}[t]{0.11\columnwidth}\raggedright\strut
(McCaldin, Johnston, and Rieker 2015)\strut
\end{minipage} & \begin{minipage}[t]{0.18\columnwidth}\raggedright\strut
Aerial baiting\strut
\end{minipage} & \begin{minipage}[t]{0.03\columnwidth}\raggedright\strut
Australia\strut
\end{minipage} & \begin{minipage}[t]{0.14\columnwidth}\raggedright\strut
Christmas Island\strut
\end{minipage} & \begin{minipage}[t]{0.10\columnwidth}\raggedright\strut
Cat (\emph{Felis catus})\strut
\end{minipage} & \begin{minipage}[t]{0.09\columnwidth}\raggedright\strut
Rotor-wing: V-TOL Hornet I-II\strut
\end{minipage} & \begin{minipage}[t]{0.11\columnwidth}\raggedright\strut
Canon S100; Drop mechanism with HD Video Recorder\strut
\end{minipage} & \begin{minipage}[t]{0.01\columnwidth}\raggedright\strut
?\strut
\end{minipage}\tabularnewline
\begin{minipage}[t]{0.11\columnwidth}\raggedright\strut
(Fornace et al. 2014)\strut
\end{minipage} & \begin{minipage}[t]{0.18\columnwidth}\raggedright\strut
Spatial epidemiology\strut
\end{minipage} & \begin{minipage}[t]{0.03\columnwidth}\raggedright\strut
Malaysia / Philippines\strut
\end{minipage} & \begin{minipage}[t]{0.14\columnwidth}\raggedright\strut
Sabah / Palawan\strut
\end{minipage} & \begin{minipage}[t]{0.10\columnwidth}\raggedright\strut
?\strut
\end{minipage} & \begin{minipage}[t]{0.09\columnwidth}\raggedright\strut
Fixed-wing: Sensefly eBee\strut
\end{minipage} & \begin{minipage}[t]{0.11\columnwidth}\raggedright\strut
16mp\strut
\end{minipage} & \begin{minipage}[t]{0.01\columnwidth}\raggedright\strut
\$ 25000\strut
\end{minipage}\tabularnewline
\begin{minipage}[t]{0.11\columnwidth}\raggedright\strut
(Van Tilburg 2017)\strut
\end{minipage} & \begin{minipage}[t]{0.18\columnwidth}\raggedright\strut
Search and Rescue (SAR)\strut
\end{minipage} & \begin{minipage}[t]{0.03\columnwidth}\raggedright\strut
USA\strut
\end{minipage} & \begin{minipage}[t]{0.14\columnwidth}\raggedright\strut
Columbia Gorge National Scenic Area\strut
\end{minipage} & \begin{minipage}[t]{0.10\columnwidth}\raggedright\strut
Humans\strut
\end{minipage} & \begin{minipage}[t]{0.09\columnwidth}\raggedright\strut
Rotor-wing: Phantom 3, SAR Bot, Inspire 1\strut
\end{minipage} & \begin{minipage}[t]{0.11\columnwidth}\raggedright\strut
DJI 12MP; VUE PRO 640 thermal imager\strut
\end{minipage} & \begin{minipage}[t]{0.01\columnwidth}\raggedright\strut
?\strut
\end{minipage}\tabularnewline
\begin{minipage}[t]{0.11\columnwidth}\raggedright\strut
(Schwarzbach et al. 2014)\strut
\end{minipage} & \begin{minipage}[t]{0.18\columnwidth}\raggedright\strut
Water sampling\strut
\end{minipage} & \begin{minipage}[t]{0.03\columnwidth}\raggedright\strut
Spain\strut
\end{minipage} & \begin{minipage}[t]{0.14\columnwidth}\raggedright\strut
Doñana N.P.\strut
\end{minipage} & \begin{minipage}[t]{0.10\columnwidth}\raggedright\strut
Freshwater ecosystems\strut
\end{minipage} & \begin{minipage}[t]{0.09\columnwidth}\raggedright\strut
Rotor-wing: Helicopter\strut
\end{minipage} & \begin{minipage}[t]{0.11\columnwidth}\raggedright\strut
Water sampling mechanism\strut
\end{minipage} & \begin{minipage}[t]{0.01\columnwidth}\raggedright\strut
?\strut
\end{minipage}\tabularnewline
\begin{minipage}[t]{0.11\columnwidth}\raggedright\strut
(Schmale, Dingus, and Reinholtz 2008)\strut
\end{minipage} & \begin{minipage}[t]{0.18\columnwidth}\raggedright\strut
Aerobiological sampling\strut
\end{minipage} & \begin{minipage}[t]{0.03\columnwidth}\raggedright\strut
USA\strut
\end{minipage} & \begin{minipage}[t]{0.14\columnwidth}\raggedright\strut
Virginia Tech's Kentland Farm\strut
\end{minipage} & \begin{minipage}[t]{0.10\columnwidth}\raggedright\strut
Prokaryotic and eukaryotic microorganisms\strut
\end{minipage} & \begin{minipage}[t]{0.09\columnwidth}\raggedright\strut
Fixed-wing: Senior Telemaster\strut
\end{minipage} & \begin{minipage}[t]{0.11\columnwidth}\raggedright\strut
Aerobiological sampling devices\strut
\end{minipage} & \begin{minipage}[t]{0.01\columnwidth}\raggedright\strut
?\strut
\end{minipage}\tabularnewline
\bottomrule
\end{longtable}

\elandscape
\normalsize

\subsection{Infrastructure and risk
assessment}\label{infrastructure-and-risk-assessment}

Other research projects highlight the convenience of RPAS in assessing
the risk that human infrastructure posed for wildlife, which results in
the implementation of more cost-effective preventive measures. For
instance, some species of birds nest on high voltage power lines poles,
making them especially vulnerable to death by electrocution. (Margarita
Mulero-Pázmány 2014) used a long endurance fixed-wing RPAS as a low cost
alternative for visual evaluation of linear electrical structures.
Collisions with the wiring is one of the most common causes of death in
birds. (Lobermeier et al. 2015) proposed to install marks that are
easily visible through the use of robotics arms installed in
multicopters. Due to the ease of maneuvering of the platform,
multi-rotors are more suitable for precision work. As a pre-harvest
activity, generally performed under mechanical procedures,
(Mulero-Pázmány Margarita 2011) suggested a flyby to identify possible
nests on the ground, and if necessary, take appropriate actions to avoid
their destruction.

\subsection{Monitoring and mapping of terrestrial and aquatic
ecosystems}\label{monitoring-and-mapping-of-terrestrial-and-aquatic-ecosystems}

The maturity of remote sensors on board air or space platforms has led
to an increase in applications for the study of ecosystems (Wulder et
al. 2004). The data obtained have enabled the development of vegetation
and soil maps, enhance the characterization of habitats or the
understanding of the structure and function of forest ecosystems,
develop digital elevation models or geomorphological maps. The emergence
of RPAS permits the quantitative analysis of habitats to a level of
detail ever reached, either for economic reasons or for technological
limitations. This momentum has been especially notable with the parallel
development of affordable multispectral and hyperspectral sensors
adapted to small aircraft (Bareth et al. 2015). RPAS has been used by
The United States Geological Survey (USGS) to classify vegetation cover
in wetlands (USGS 2014). (Zaman, Jensen, and McKee 2011) monitored the
spread of invasive species in such ecosystems while (Müllerová et al.
2016) assessed the importance of spectral and spatial resolution to
detect plant invasion using satellite, aerial and RPAS data. (Perroy,
Sullivan, and Stephenson 2017) conducted several flight to quantify the
detection rate of invasive species under different flight parameters,
environmental conditions and vegetation cover. The characterization of
forest stands constitutes an important section, considering the number
of papers facing the issue from different perspectives. (Gini et al.
2012) employed a quadcopter model operated at low-height and RGB and NIR
cameras in small areas. Due to the reduced reliability and autonomy of
the platform and the difficulties to increase the load capacity, the
flight planning is reduced to three passes with a percentage of 80\% and
30\% of longitudinal and transverse overlap respectively. (Lisein et al.
2015) performs a multitemporal analysis of the spectral response to
phenological variations in different species of deciduous trees and
concluded that intraspecific spectral variation is of maximum interest
for the optimization of classification algorithms and discrimination
between species. During the research, the authors operated a fixed wing
RPAS model, used different sensors in the visible and near infrared
range and optimized the flight parameters to cover the maximum surface
area with the fewest possible number of flights. (Zahawi et al. 2015)
applies the Ecosynth methodology, a toolkit for mapping and measuring 3D
vegetation using digital cameras and artificial vision open source
software. Such project was aimed to evaluate the effectiveness of
restoration actions in forests using RPAS as a viable alternative for
traditional field measurements and applying different predictive models
of frugivorous birds presence by means of height and canopy structure
data. Recently, shallow coastal habitats were also mapped using consumer
grade RPAS (Casella et al. 2017, Ventura et al. (2016)), including
monitoring erosion dynamics in shorelines. (Casella et al. 2016).

\subsection{Surveillance and support for compliance with laws in
protected
areas}\label{surveillance-and-support-for-compliance-with-laws-in-protected-areas}

RPAS have also relevance in the control and surveillance of protected
areas, documented through different experiences focused mainly on
controlling poaching. As a result of these preliminary results, emphasis
has been placed on improving first-person vision (FPV) methods to obtain
a clear, real-time view of the monitored area. Also, fixed-wing RPAS are
prefered, since they offered longer flight times and consequently cover
greater areas. Other requisites include the convenience of using thermal
cameras in low visibility conditions, generally related to hours of
greater furtive activity, along with advances in computer vision systems
programmed to detect the presence of humans and target species under
pressure from illegal trade in protected areas. (Mulero-Pázmány et al.
2014) focused on the African rhinoceros (\emph{Diceros bicornis},
\emph{Ceratotherium simum}) and noted the advantages of real time video
compared to still photography, which despite the better overall quality,
requires longer post-processing time. In addition, authors emphasized
the need to improve the resolution of thermal sensors to increase the
chances of detecting suspicious activity at night time. (Franco et al.
2016) suggested using RPAS to combat poaching and illegal fishing
activities in marine protected areas (MPA), claiming million-dollar
economic losses in the fishing sector. (Duffy 2014) analyzed the
consequences of the militarization of conservation practices as an
increasing trend in natural protected areas around the world and
illustrates the use of RPAS through several examples. Surveillance of
other less contentious illegal activities in natural parks using an
affordable small rotor-wing were also recently described (Sabella et al.
2017).

\subsection{Ecotourism}\label{ecotourism}

The high degree of diversification offered by RPAS in the ecotourism
industry is summarized in a recent article, which shows possible
recreational activities, business opportunities, search and rescue
operations, mapping and formulas for granting RPAS flight permits in
designated areas (King 2014). Within the still scarce literature,
(Hansen 2016) values the effectiveness of RPAS in monitoring visitors in
marine and coastal areas, in combination with other innovative
solutions. According to the author the RPAS would theoretically allow to
operate under different environmental conditions, improving the level of
detail and offering a continuous coverage in the flow and behavior of
visitors, as opposed to other techniques of habitual use like the manual
observation or the installation of networks of surveillance cameras.

\subsection{Impact of RPAS on wildlife and
ecosystems}\label{impact-of-rpas-on-wildlife-and-ecosystems}

Animal welfare in wildlife management practices and ecological research
is a sensitive issue from which ethical issues arise (Wilson and McMahon
2006). Not surprisingly, RPAS are not exempt of discussion. (Vas et al.
2015) assessed the impact of color, speed and angle of flight on the
behavioral responses of wetland birds. The latter factor was considered
the primary trigger for changes in behavioral patterns, especially in
vertical approaches at an angle of 90º. Finally, a core set of
recommendations is included, and authors encouraged to extend the trials
to a wide range of RPAS and species. (McEvoy, Hall, and McDonald 2016)
accomplished the most intensive disturbance assesment on waterfowl to
date, by combining an array of fixed-wing and rotor-wings RPAS at
various altitudes, while (BORRELLE and FLETCHER 2017) reviewed the
published literature tackling the impact of RPAS on surface-nesting
seabird. (Scobie and Hugenholtz 2016) quantified noise detection by
several representative species, suggesting flying higher than 200 meters
to minimise noise disturbance. (Ditmer et al. 2015) measured
physiological stress in American black bear (\emph{Ursus americanus}) by
electronic recording of cardiac activity in the presence of RPAS.
Although no changes in behavior patterns are detected, heart rate
(measure in beats per minute) was significant higher in most cases
observed. (Pomeroy, O'Connor, and Davies 2015) noticed evidence of
variation in reactivity in seal populations based on a variety of
factors, from the RPAS platform, flight height and lateral distance to
the breeding or moulting season. No adverse reactions were reported in
elephants (\emph{Loxodonta africana}) or cattle (\emph{Bos taurus}) on
flights at a minimum height of 100 meters (Jain 2013, Mulero-Pázmány et
al. (2015)). (Weissensteiner, Poelstra, and Wolf 2015) observed similar
disturbance patterns using RPAS with respect to manual inspection when
documenting nesting status in hooded crows (\emph{Corvus corone
cornix}), but they emphasized the shorter period of stress to which
birds are exposed, along with reducing or eliminating risk in climbers
and avoiding damages to trees. In the absence of further experiences to
date explicitly addressing animal welfare, (Hodgson and Koh 2016)
suggested a series of general recommendations as the basis for a code of
good practice, highlighting the adoption of the precautionary principle
and respect for aviation standards, the specific training of operators,
the appropriate selection of equipment, the cessation of operations in
the case of obvious disturbances in the populations studied and the
reporting of observations in scientific publications, that allows
sharing of knowledge to progressively improve the protocols of
operations with RPAS that involve the observation of fauna.
(Mulero-Pázmány et al. 2017) carried out an up-to-date systematic review
of literature documenting RPAS disturbance effects in wildlife and
outlined some findings, while complementing the previously mentioned
recommendations. The authors agreed that noise level and intensity are
the main, but not the only factor inducing animal reactions in presence
of RPAS, where birds are the most sensitive species, followed by
terrestrial mammals and aquatic animals. These response patterns are
also influenced by life-history stage and level of aggregation of
targeted species.

\section{Environmental management and decision
support}\label{environmental-management-and-decision-support}

Planning in protected areas is reflected through a variety of management
programs that are difficult to fit into some of the previously discussed
categories. An extensive report based on Christmas Island proposed
dropping poisoned baits from RPAS to eradicate feral cats disturbing
threatened native species (McCaldin, Johnston, and Rieker 2015).
(Cornell, Herman, and Ontiveros 2016) obtained ground truth data by
adapting RPAS to take water samples for comparison with hyperspectral
measurements of Landsat 8 Operational Land Imager (OLI). (Zang et al.
2012) identified several pollution agents in riparian areas using RPAS
imagery. (Schwarzbach et al. 2014) goes further away by performing
several aerial water sampling methods using an unmaned helicopter to
monitor water pollution, while (Schmale, Dingus, and Reinholtz 2008)
collected a broad spectrum of both prokaryotic and eukaryotic
microorganisms using a fixed-wing aircraft equiped with a custom made
aerial sampling device. (Fornace et al. 2014) considered mapping
enviromental risk factors for predicting zoonotic diseases as part of a
extensive epidemiological study carried out in Philippines and so on.
Literature citing RPAS for search and rescue activities is profuse and
an in-depth revision is beyond the scope of this article, but a recent
publication ilustrate several examples where RPAS were succesfully
operated to assist rescue teams (Van Tilburg 2017). A google scholar
search sorted by relevance using disaster management and drones keywords
throws at first place a complete report describing a complex framework
for decision support using RPAS (Maza et al. 2011).

\section{Discussion}\label{discussion}

Most of the sources analyzed focus on local-scale conservation projects
and feasibility studies of RPAS monitoring distribution and abundance of
wildlife populations. Literature begins to be equally prolific in
mapping activities in terrestrial and aquatic ecosystems, a niche until
recently entirely occupied by aerial and space platforms for
environmental remote sensing. Despite the low number of scientific
articles addressing the use of RPAS in the control and surveillance of
natural protected areas, it is still one of the issues that more social
debate generates and it is not strange to find governmental initiatives
or promoted by environmental organizations in the fight against
poaching. From the economic point of view, expenses derived from the
operation with RPAS are hardly quantifiable. Not all studies consider
the effort required for the development of technical and analytical
skills of work teams. Computational requirements are demanding and
certain phases of information processing requires the acquisition of
software whose price is generally high. Also, operations with RPAS are
not exempt from accidents, which has an negative impact on the budget
originally planned. In addition, statistical and sampling methods
approaching the analysis of data are mostly in its infancy and further
research should be accompased to assess the overall performance of these
methods.

\subsection{Wildlife Monitoring and
Management}\label{wildlife-monitoring-and-management-1}

Both fixed-wing and rotor-wing RPAS might become very handy tools for
conservation practitioners. Wildlife census campaigns, usually carried
out by going in on foot, terrestrial vehicles or by vessel deployment in
aquatic environments, could be supplemented or replaced by RPAS mapping
and monitoring capabilities. As becoming easier to operate, there are
sufficient grounds to encourage park rangers training in the use of
RPAS, which are subject in many cases to time-consuming and often
dangerous raids. If operated responsibly, it could be closer of being
considered a non-invasive and reliable monitoring technique (Z. Jewell
2013a). From the technological point of view, ``Follow-me'' capabilities
of RPAS constitute a promising advance in animal movement and remote
sensing disciplines, by having high-resolution aerial imagery from
places frequently visited or crossed by electronically tagged species.

\subsection{Infrastructure and risk
assessment}\label{infrastructure-and-risk-assessment-1}

Wildlife risk assessment could benefit from RPAS by promoting their use
for preventive purposes in conflicting areas where human factor is
indirectly causing the killing of many species. Relative low operational
cost of RPAS make them an attractive alternative to manual inspection,
which may foster such practices. Since the literature citing RPAS for
such purposes is limited, we propose that they could serve to reduce
fatalities by scheduling periodic flights monitoring facilities, roads
crossing sensitive areas and coastal ecosystems where vessels strikes
with aquatic species is frequent. RPAS might also help to persuade birds
from approaching power lines, wind turbines and other potential hazards,
just as it has been applied to keep airports safe.

\subsection{Monitoring and mapping of terrestrial and aquatic
ecosystems}\label{monitoring-and-mapping-of-terrestrial-and-aquatic-ecosystems-1}

The integration of classical remote sensing elements developed during
the last decades in the scope of RPAS opens new possibilities in the
observation of environmental phenomena at local scale, complementing
other systems of Earth observation. Protected area managers should be
aware of the benefits of having information on demand. As requirements
change, Information Technology (IT) departments in protected areas must
be ready to integrate data into effective conservation strategies and
better decision-making. The inclusion of RPAS in monitoring activities
should be weighed against the major complexity of such systems.

\subsection{Surveillance and support for compliance with laws in natural
protected
areas}\label{surveillance-and-support-for-compliance-with-laws-in-natural-protected-areas}

The convenience of using RPAS in the fight against poaching and illegal
fishing in protected areas faces important technical and legal
constraints. First, the reviewed literature mention the need to design
more efficient live vision systems. Low autonomy of RPAS is especially
critical in large natural parks, limiting the area under surveillance.
Issues concerning atmospheric conditions have not yet been completely
resolved. (Banzi 2014) proposed a sensor based economical feasible
anti-poaching alternative, arguing that RPAS fulfilling suitable
specifications are costly, especially in developing countries. However,
as technology becomes more accessible, it is expected that main barriers
will appear in the legislative and social sphere. In some countries it
is forbidden to fly beyond the visual range of the operator, limiting
the effectiveness of the inspection in real time. RPAS applied to
surveillance of protected areas is also questioned arguing human right
breaching (Banzi 2014). Some detractors are skeptical about the ability
of RPAS to persuade offenders, who in many cases face situations of
greatest need. Probably the success of such initiatives requires a
greater consensus among the parties involved and the development of
strategies that seek to solve the causes of poaching. Surveillance of
illegal logging activities or bonfires detection in unauthorized areas
have great potential and may be convenient to implement.

\subsection{Ecotourism}\label{ecotourism-1}

A permissive regularization of RPAS in ecotourism activities in natural
parks could lead to unpredictable situations. On the one hand, the
constant presence of sources of noise coming from propellers and
engines, the sensation of invasion or lack of privacy, security issues
and the visual impact of RPAS on the landscape could negatively affect
the tourist experience. It is well known that the consequences triggered
by RPAS disturbing wildlife have led to the ban on flying them in
national parks in the United States and other protected areas of the
world. As result of potential enviromental impact due to the use of RPAS
by tourist in Antarctica, (Leary 2017) reported the partial prohibition
of recreational RPAS in coastal areas as part of a more extensive
regulation promoted by stakeholders. Such regulation looks reasonable
and could be the way forward for other protected areas to adapt the
allowed activities with RPAS. It seems obvious to deduce that in the
hands of non-skilled operators, the risk of accidents and losses would
increase. This may lead to the aforementioned wildlife disturbance, but
they also pose a risk for contamination of water supplies or the
triggering of fires in sensitive areas due to the presence of flammable
and toxic components, fueling the low popularity of RPAS to the
detriment of the benefits they bring. It does not appear that
feasibility studies or opinion polls have been published that respond to
the issues raised and to the ethical and legal implications derived from
their use. Even when the leisure possibilities are wide and recognized,
it would be advisable to be cautious in the face of the demand of the
ecotourism industry to incorporate RPAS in their activities.

\subsection{Impact of RPAS on wildlife and
ecosystems}\label{impact-of-rpas-on-wildlife-and-ecosystems-1}

The review of the literature suggests that there are still certain
niches that need more attention from the research community. The ethical
implications of RPAS in wildlife studies have not yet been adequately
weighed since most studies only marginally address the presence or
absence of reactions in species in the vicinity of RPAS. Despite the
greater degree of awareness reflected in a emergent set of guidelines
(Hodgson and Koh 2016; Mulero-Pázmány et al. 2017) , we consider that
further trials aimed at quantifying physiological and behavioral changes
targeting a broader group of wild species should be carried out. The
establishment of a best practices and recommendations manual could
increase the chances of integrating the responsible use of RPAS in
conservation and management activities in natural areas. Moreover, some
authors mentioned the lack of commercial operators with sufficient
expertise to carry out such activities (McEvoy, Hall, and McDonald
2016). Also, an optimal trade-off between benefits and environmental
costs should be pursued (Grémillet et al. 2012; Sepúlveda et al. 2010).
By designing quieter, non-polluting and safer components, the impact on
wildlife and ecosystems could be reduced and its objective observation
facilitated. Nonetheless we trust that, as far as further testing be
done, RPAS has great potential to replace more invasive monitoring
techniques, whose reliability is challenged by the potential to induce
conditions of unacceptable stress in wildlife that could ultimately
invalidate the results of the research (Z. Jewell 2013b, Wilson and
McMahon (2006)). This should be consciously taken into account by the
many actors involved in protected areas activities when they are
reluctant to allow RPAS to be an essential part of research and
conservation activities.

\section{Environmental monitoring and decision
support}\label{environmental-monitoring-and-decision-support}

Protected areas are subject to periodic environmental quality control
procedures where RPAS could play a major role. Also, RPAS are suitable
to assist decision making where rapid response is crucial by offering
valuable information at real time to handle natural and man-made
disasters. Wildfires is a major concern in natural parks and is not rare
that RPAS have been put forward to assist in prevention, fighting and
evaluation phases. In most cases, such applications have operational
requirements which eventually are costly. For instance, sophisticated
on-board instruments, gas powered engines for longer endurance and
higher payloads or robotics arms and containers designed to assist
sampling, hold cargo or deliver assistance. RPAS could leverage wildlife
capture procedures by carrying dart guns for chemical immobilization
where otherwise manual approaching free-ranging animals is considered
inefficient or dangerous.

\section{Conclusions}\label{conclusions}

The consolidation of the RPAS as management and research tools in
protected areas is closely linked to the technological development of
the elements associated with the platform and to the establishment of
measures that favorably regulate its use, increasing opportunities in
the sector and stimulating innovation in priority conservation areas.
There are continually improvements in navigation control and flight
autonomy, while we are witnessing the progressive miniaturization and
diversification of sensors along with advances in the field of
artificial intelligence. This rapidly expanding confluence of factors
encourages the emergence of new scenarios with ethical and legal
implications. Most governments have reacted by setting constraints that
could have a negative impact on the capacity to integrate RPAS into the
civilian sphere, despite some progress in this regard. As a result, it
is difficult to foresee the actions that each country will adopt from
now on in an attempt to harmonize the contradictions presented by RPAS,
reason why it is probable that the future of the RPAS in protected areas
is conditioned fundamentally by political and social factors.

\section*{References}\label{references}
\addcontentsline{toc}{section}{References}

\hypertarget{refs}{}
\hypertarget{ref-abd-elrahman_development_2005}{}
Abd-Elrahman, Amr, Leonard Pearlstine, and Franklin Percival. 2005.
``Development of Pattern Recognition Algorithm for Automatic Bird .''
\emph{Surv. L. Inf. Sci.} 65 (1): 37.

\hypertarget{ref-Bworld_of_drones2017}{}
America, New. 2017. ``\emph{World of Drones}.''
\url{http://drones.newamerica.org/\#regulations} {[}Accessed: 17 June,
2017{]}.

\hypertarget{ref-van_andel_locating_2015}{}
Andel, Alexander C. van, Serge A. Wich, Christophe Boesch, Lian Pin Koh,
Martha M. Robbins, Joseph Kelly, and Hjalmar S. Kuehl. 2015. ``Locating
Chimpanzee Nests and Identifying Fruiting Trees with an Unmanned Aerial
Vehicle.'' \emph{Am. J. Primatol.} 77 (10): 1122--1134.
doi:\href{https://doi.org/10.1002/ajp.22446}{10.1002/ajp.22446}.

\hypertarget{ref-andrew_semi-automated_2017}{}
Andrew, Margaret E, and Jill M Shephard. 2017. ``Semi-Automated
Detection of Eagle Nests: An Application of Very High-Resolution Image
Data and Advanced Image Analyses to Wildlife Surveys.''
doi:\href{https://doi.org/10.1002/rse2.38}{10.1002/rse2.38}.

\hypertarget{ref-Andrews2014}{}
Andrews, C. 2014. ``\emph{Wildlife Monitoring: Should Uav Drones Be
Banned?}''
\url{https://prod-eandt.theiet.org/content/articles/2014/07/wildlife-monitoring-should-uav-drones-be-banned/}
{[}Accessed: 07 Abril, 2017{]}.

\hypertarget{ref-banzi_sensor_2014}{}
Banzi, Jamali Firmat. 2014. ``A Sensor Based Anti-Poaching System in
Tanzania National Parks.'' \emph{International Journal of Scientific and
Research Publications} 4 (4).

\hypertarget{ref-bareth_low-weight_2015}{}
Bareth, Georg, Helge Aasen, Juliane Bendig, Martin Leon Gnyp, Andreas
Bolten, Andr?s Jung, Ren? Michels, and Jussi Soukkam?ki. 2015.
``Low-Weight and UAV-Based Hyperspectral Full-Frame Cameras for
Monitoring Crops: Spectral Comparison with Portable Spectroradiometer
Measurements.'' \emph{Photogrammetrie - Fernerkundung - Geoinformation}
2015 (1): 69--79.
doi:\href{https://doi.org/10.1127/pfg/2015/0256}{10.1127/pfg/2015/0256}.

\hypertarget{ref-barnosky_has_2011}{}
Barnosky, Anthony D, Nicholas Matzke, Susumu Tomiya, Guinevere O U
Wogan, Brian Swartz, Tiago B Quental, Charles Marshall, et al. 2011.
``Has the Earth's Sixth Mass Extinction Already Arrived?'' \emph{Nature}
470 (7336): 51--57.
doi:\href{https://doi.org/10.1038/nature09678}{10.1038/nature09678}.

\hypertarget{ref-bayram_active_2016}{}
Bayram, Haluk, Krishna Doddapaneni, Nikolaos Stefas, and Volkan Isler.
2016. ``Active Localization of VHF Collared Animals with Aerial
Robots,'' no. 13: 74--75.
doi:\href{https://doi.org/10.1109/COASE.2016.7743503}{10.1109/COASE.2016.7743503}.

\hypertarget{ref-borrelle_will_2017}{}
BORRELLE, SB, and AT FLETCHER. 2017. ``Will Drones Reduce Investigator
Disturbance to Surface-Nesting Seabirds?'' bibtex: borrelle2017will.
\emph{Marine Ornithology} 45: 89--94.

\hypertarget{ref-bustamante_forest_2015}{}
Bustamante, Luis Antonio Esquivel. 2015. ``Forest Monitoring with Drones
: Application Strategies for Protected Riverine Forest Ecosystems in the
Atlantic Forest of Rio de,'' 96.

\hypertarget{ref-casella_mapping_2017}{}
Casella, Elisa, Antoine Collin, Daniel Harris, Sebastian Ferse, Sonia
Bejarano, Valeriano Parravicini, James L. Hench, and Alessio Rovere.
2017. ``Mapping Coral Reefs Using Consumer-Grade Drones and Structure
from Motion Photogrammetry Techniques.'' \emph{Coral Reefs} 36 (1):
269--275.
doi:\href{https://doi.org/10.1007/s00338-016-1522-0}{10.1007/s00338-016-1522-0}.

\hypertarget{ref-casella_drones_2016}{}
Casella, Elisa, Alessio Rovere, Andrea Pedroncini, Colin P. Stark, Marco
Casella, Marco Ferrari, and Marco Firpo. 2016. ``Drones as Tools for
Monitoring Beach Topography Changes in the Ligurian Sea (NW
Mediterranean).'' \emph{Geo-Marine Letters} 36 (2): 151--163.
doi:\href{https://doi.org/10.1007/s00367-016-0435-9}{10.1007/s00367-016-0435-9}.

\hypertarget{ref-chabot_wildlife_2015}{}
Chabot, Dominique, and David M. Bird. 2015. ``Wildlife Research and
Management Methods in the 21st Century: Where Do Unmanned Aircraft Fit
in?'' \emph{J. Unmanned Veh. Syst.} 3 (4): 137--155.
doi:\href{https://doi.org/10.1139/juvs-2015-0021}{10.1139/juvs-2015-0021}.

\hypertarget{ref-chabot_computer-automated_2016}{}
Chabot, Dominique, and Charles M. Francis. 2016. ``Computer-Automated
Bird Detection and Counts in High-Resolution Aerial Images: A Review.''
\emph{Journal of Field Ornithology} 87 (4): 343--359.
doi:\href{https://doi.org/10.1111/jofo.12171}{10.1111/jofo.12171}.

\hypertarget{ref-christiansen_automated_2014}{}
Christiansen, Peter, Kim A rild Steen, Rasmus N yholm Jørgensen, and
Henrik Karstoft. 2014. ``Automated Detection and Recognition of Wildlife
Using Thermal Cameras.'' \emph{Sensors (Basel).} 14 (8): 13778--13793.
doi:\href{https://doi.org/10.3390/s140813778}{10.3390/s140813778}.

\hypertarget{ref-christie_unmanned_2016}{}
Christie, Katherine S., Sophie L. Gilbert, Casey L. Brown, Michael
Hatfield, and Leanne Hanson. 2016. ``Unmanned Aircraft Systems in
Wildlife Research: Current and Future Applications of a Transformative
Technology.'' \emph{Front. Ecol. Environ.} 14 (5): 241--251.
doi:\href{https://doi.org/10.1002/fee.1281}{10.1002/fee.1281}.

\hypertarget{ref-cliff_online_2015}{}
Cliff, Oliver M, Robert Fitch, Salah Sukkarieh, Debra L Saunders, and
Robert Heinsohn. 2015. ``Online Localization of Radio-Tagged Wildlife
with an Autonomous Aerial Robot System.'' \emph{Robot. Sci. Syst.}, no.
November 2016: 1--9.
doi:\href{https://doi.org/10.15607/RSS.2015.XI.042}{10.15607/RSS.2015.XI.042}.

\hypertarget{ref-colefax_potential_2017}{}
Colefax, Andrew P., Paul A. Butcher, and Brendan P. Kelaher. 2017. ``The
Potential for Unmanned Aerial Vehicles (UAVs) to Conduct Marine Fauna
Surveys in Place of Manned Aircraft.'' \emph{ICES Journal of Marine
Science}, June.
doi:\href{https://doi.org/10.1093/icesjms/fsx100}{10.1093/icesjms/fsx100}.

\hypertarget{ref-colomina_unmanned_2014}{}
Colomina, I., and P. Molina. 2014. ``Unmanned Aerial Systems for
Photogrammetry and Remote Sensing: A Review.'' \emph{ISPRS Journal of
Photogrammetry and Remote Sensing} 92 (June): 79--97.
doi:\href{https://doi.org/10.1016/j.isprsjprs.2014.02.013}{10.1016/j.isprsjprs.2014.02.013}.

\hypertarget{ref-Global2017}{}
Consulting, OZYRPAS. 2017. ``\emph{Global Drone Regulations Database}.''
\url{https://www.droneregulations.info/index.html} {[}Accessed: 19 July,
2017{]}.

\hypertarget{ref-cook_design_1979}{}
Cook, R. Dennis, and Jerald O. Jacobson. 1979. ``A Design for Estimating
Visibility Bias in Aerial Surveys.'' \emph{Biometrics} 35 (4): 735.
doi:\href{https://doi.org/10.2307/2530104}{10.2307/2530104}.

\hypertarget{ref-cornell_use_2016}{}
Cornell, Dylan, Maryann Herman, and Fernando Ontiveros. 2016. ``Use of a
UAV for Water Sampling to Assist Remote Sensing of Bacterial Flora in
Freshwater Environments.''
\url{http://fisherpub.sjfc.edu/undergraduate_ext_pub/17/}.

\hypertarget{ref-ditmer_bears_2015}{}
Ditmer, Mark A., John B. Vincent, Leland K. Werden, Jessie C. Tanner,
Timothy G. Laske, Paul A. Iaizzo, David L. Garshelis, and John R.
Fieberg. 2015. ``Bears Show a Physiological but Limited Behavioral
Response to Unmanned Aerial Vehicles.'' \emph{Curr. Biol.} 25 (17):
2278--2283.
doi:\href{https://doi.org/10.1016/j.cub.2015.07.024}{10.1016/j.cub.2015.07.024}.

\hypertarget{ref-dudley_guidelines_2008}{}
Dudley, Nigel. 2008. \emph{Guidelines for Applying Protected Area
Management Categories}. bibtex: dudley\_guidelines\_2008. IUCN.

\hypertarget{ref-duffy_waging_2014}{}
Duffy, Rosaleen. 2014. ``Waging a War to Save Biodiversity: The Rise of
Militarized Conservation.'' \emph{Int. Aff.} 90 (4): 819--834.
doi:\href{https://doi.org/10.1111/1468-2346.12142}{10.1111/1468-2346.12142}.

\hypertarget{ref-fornace_mapping_2014}{}
Fornace, Kimberly M., Chris J. Drakeley, Timothy William, Fe Espino, and
Jonathan Cox. 2014. ``Mapping Infectious Disease Landscapes: Unmanned
Aerial Vehicles and Epidemiology.'' \emph{Trends Parasitol.}, October,
1--6.
doi:\href{https://doi.org/10.1016/j.pt.2014.09.001}{10.1016/j.pt.2014.09.001}.

\hypertarget{ref-forum_toward_2008}{}
Forum, Policy. 2008. ``Toward a Global Biodiversity Observing System,''
no. April.

\hypertarget{ref-franco_five_2016}{}
Franco, Antonio Di, Pierre Thiriet, Giuseppe Di Carlo, Charalampos
Dimitriadis, Patrice Francour, Nicolas L Gutiérrez, Alain Jeudy De
Grissac, et al. 2016. ``Five Key Attributes Can Increase Marine
Protected Areas Performance for Small-Scale Fisheries Management.''
\emph{Nat. Publ. Gr.}, no. November: 1--9.
doi:\href{https://doi.org/10.1038/srep38135}{10.1038/srep38135}.

\hypertarget{ref-van_gemert_nature_2015}{}
Gemert, Jan C. van, Camiel R. Verschoor, Pascal Mettes, Kitso Epema,
Lian Pin Koh, and Serge Wich. 2015. ``Nature Conservation Drones for
Automatic Localization and Counting of Animals.'' \emph{Lect. Notes
Comput. Sci. (Including Subser. Lect. Notes Artif. Intell. Lect. Notes
Bioinformatics)} 8925: 255--270.
doi:\href{https://doi.org/10.1007/978-3-319-16178-5_17}{10.1007/978-3-319-16178-5\_17}.

\hypertarget{ref-Georg2016}{}
Gerster/Panos, Georg. 2017. ``\emph{Project Uses Drones to Monitor
Coastal Erosion in Ghana}.''
\url{http://www.scidev.net/sub-saharan-africa/environment/news/project-drones-monitor-coastal-erosion-ghana.html}
{[}Accessed: 07 May, 2017{]}.

\hypertarget{ref-gini_aerial_2012}{}
Gini, R., D. Passoni, L. Pinto, and G. Sona. 2012. ``Aerial Images from
an Uav System: 3D Modeling and Tree Species Classification in a Park
Area.'' \emph{ISPRS - Int. Arch. Photogramm. Remote Sens. Spat. Inf.
Sci.} XXXIX-B1 (September): 361--366.
doi:\href{https://doi.org/10.5194/isprsarchives-XXXIX-B1-361-2012}{10.5194/isprsarchives-XXXIX-B1-361-2012}.

\hypertarget{ref-gremillet_robots_2012}{}
Grémillet, David, William Puech, Véronique Garçon, Thierry Boulinier,
and Yvon Le Maho. 2012. ``Robots in Ecology: Welcome to the Machine.''
\emph{Open J. Ecol.} 02 (2): 49--57.
doi:\href{https://doi.org/10.4236/oje.2012.22006}{10.4236/oje.2012.22006}.

\hypertarget{ref-hansen_applying_2016}{}
Hansen, Andreas Skriver. 2016. ``Applying Visitor Monitoring Methods in
Coastal and Marine Areas -- Some Learnings and Critical Reflections from
Sweden.'' \emph{Scand. J. Hosp. Tour.} 2250 (June): 1--18.
doi:\href{https://doi.org/10.1080/15022250.2016.1155481}{10.1080/15022250.2016.1155481}.

\hypertarget{ref-hardin_small-scale_2013}{}
Hardin, Perry J, and Ryan R Jensen. 2013. ``Small-Scale Unmanned Aerial
Vehicles in Environmental Remote Sensing: Challenges and
Opportunities,'' no. October 2014: 37--41.
doi:\href{https://doi.org/10.2747/1548-1603.48.1.99}{10.2747/1548-1603.48.1.99}.

\hypertarget{ref-hodgson_unmanned_2013}{}
Hodgson, Amanda, Natalie Kelly, and David Peel. 2013. ``Unmanned Aerial
Vehicles (UAVs) for Surveying Marine Fauna: A Dugong Case Study.''
\emph{PLoS One} 8 (11): 1--15.
doi:\href{https://doi.org/10.1371/journal.pone.0079556}{10.1371/journal.pone.0079556}.

\hypertarget{ref-hodgson_unmanned_2017}{}
Hodgson, Amanda, David Peel, and Natalie Kelly. 2017. ``Unmanned Aerial
Vehicles for Surveying Marine Fauna: Assessing Detection Probability.''
\emph{Ecological Applications}.
\url{http://onlinelibrary.wiley.com/doi/10.1002/eap.1519/full}.

\hypertarget{ref-hodgson_best_2016}{}
Hodgson, Jarrod C., and Lian Pin Koh. 2016. ``Best Practice for
Minimising Unmanned Aerial Vehicle Disturbance to Wildlife in Biological
Field Research.'' \emph{Curr. Biol.} 26 (10).
doi:\href{https://doi.org/10.1016/j.cub.2016.04.001}{10.1016/j.cub.2016.04.001}.

\hypertarget{ref-ivosevic_use_2015}{}
Ivošević, Bojana, Yong Gu Han, Youngho Cho, and Ohseok Kwon. 2015. ``The
Use of Conservation Drones in Ecology and Wildlife Research.'' \emph{J.
Ecol. Environ.} 38 (1): 113--118.
doi:\href{https://doi.org/10.5141/ecoenv.2015.012}{10.5141/ecoenv.2015.012}.

\hypertarget{ref-jain_unmanned_2013}{}
Jain, Mukesh. 2013. ``Unmanned Aerial Survey of Elephants.'' \emph{PLoS
One}.
doi:\href{https://doi.org/10.1371/\%20journal.pone.0054700}{10.1371/ journal.pone.0054700}.

\hypertarget{ref-jewell_effect_2013-1}{}
Jewell, Zoe. 2013a. ``Effect of Monitoring Technique on Quality of
Conservation Science: Ethics and Science in Conservation.''
\emph{Conservation Biology} 27 (3): 501--508.
doi:\href{https://doi.org/10.1111/cobi.12066}{10.1111/cobi.12066}.

\hypertarget{ref-jewell_effect_2013}{}
Jewell, Zoe. 2013b. ``Effect of Monitoring Technique on Quality of
Conservation Science: Ethics and Science in Conservation.''
\emph{Conservation Biology} 27 (3): 501--508.
doi:\href{https://doi.org/10.1111/cobi.12066}{10.1111/cobi.12066}.

\hypertarget{ref-juffe-bignoli_protected_2014}{}
Juffe-Bignoli, Diego, Neil David Burgess, H Bingham, E M S Belle, M G De
Lima, M Deguignet, B Bertzky, et al. 2014. ``Protected Planet Report
2014.'' \emph{Cambridge, UK UNEP-WCMC}.

\hypertarget{ref-king_will_2014}{}
King, Lisa M. 2014. ``Will Drones Revolutionise Ecotourism?'' \emph{J.
Ecotourism} 13 (1): 85--92.
doi:\href{https://doi.org/10.1080/14724049.2014.948448}{10.1080/14724049.2014.948448}.

\hypertarget{ref-koh_dawn_2012}{}
Koh, Lian Pin, and Serge A. Wich. 2012. ``Dawn of Drone Ecology:
Low-Cost Autonomous Aerial Vehicles for Conservation.'' \emph{Trop.
Conserv. Sci.} 5 (2): 121--132.
doi:\href{https://doi.org/WOS:000310846600002}{WOS:000310846600002}.

\hypertarget{ref-koski_evaluation_2009}{}
Koski, William R., Travis Allen, Darren Ireland, Greg Buck, Paul R.
Smith, A. Michael Macrander, Melissa A. Halick, Chris Rushing, David J.
Sliwa, and Trent L. McDonald. 2009. ``Evaluation of an Unmanned Airborne
System for Monitoring Marine Mammals.'' \emph{Aquatic Mammals} 35 (3):
347--357.
doi:\href{https://doi.org/10.1578/AM.35.3.2009.347}{10.1578/AM.35.3.2009.347}.

\hypertarget{ref-korner_autonomous_2010}{}
Körner, Fabian, Raphael Speck, Ali Haydar, and Salah Sukkarieh. 2010.
``Autonomous Airborne Wildlife Tracking Using Radio Signal Strength,''
107--112.

\hypertarget{ref-lancia_estimating_2005}{}
Lancia, Richard A, William L Kendall, Kenneth H Pollock, and James D
Nichols. 2005. ``Estimating the Number of Animals in Wildlife
Populations.'' bibtex: lancia\_estimating\_2005.

\hypertarget{ref-leary_drones_2017}{}
Leary, David. 2017. ``Drones on Ice: An Assessment of the Legal
Implications of the Use of Unmanned Aerial Vehicles in Scientific
Research and by the Tourist Industry in Antarctica.'' \emph{Polar
Record}, May, 1--15.
doi:\href{https://doi.org/10.1017/S0032247417000262}{10.1017/S0032247417000262}.

\hypertarget{ref-leonardo_miniature_2013}{}
Leonardo, Miguel, Austin Jensen, Calvin Coopmans, Mac McKee, and
YangQuan Chen. 2013. ``A Miniature Wildlife Tracking UAV Payload System
Using Acoustic Biotelemetry.'' \emph{Proc. ASME Int. Des. Eng. Tech.
Conf. Comput. Inf. Eng. Conf.}, no. July 2015.
doi:\href{https://doi.org/10.1115/DETC2013-13267}{10.1115/DETC2013-13267}.

\hypertarget{ref-lhoest_how_2015}{}
Lhoest, S., J. Linchant, S. Quevauvillers, C. Vermeulen, and P. Lejeune.
2015. ``How Many Hippos (Homhip): Algorithm for Automatic Counts of
Animals with Infra-Red Thermal Imagery from UAV.'' \emph{Int. Arch.
Photogramm. Remote Sens. Spat. Inf. Sci. - ISPRS Arch.} 40 (3):
355--362.
doi:\href{https://doi.org/10.5194/isprsarchives-XL-3-W3-355-2015}{10.5194/isprsarchives-XL-3-W3-355-2015}.

\hypertarget{ref-linchant_are_2015}{}
Linchant, Julie, Jonathan Lisein, Jean Semeki, Philippe Lejeune, and
Cédric Vermeulen. 2015. ``Are Unmanned Aircraft Systems (UASs) the
Future of Wildlife Monitoring? A Review of Accomplishments and
Challenges.'' \emph{Mamm. Rev.} 45 (4): 239--252.
doi:\href{https://doi.org/10.1111/mam.12046}{10.1111/mam.12046}.

\hypertarget{ref-lisein_discrimination_2015}{}
Lisein, Jonathan, Adrien Michez, Hugues Claessens, and Philippe Lejeune.
2015. ``Discrimination of Deciduous Tree Species from Time Series of
Unmanned Aerial System Imagery.'' \emph{PLoS One} 10 (11).
doi:\href{https://doi.org/10.1371/journal.pone.0141006}{10.1371/journal.pone.0141006}.

\hypertarget{ref-lobermeier_mitigating_2015}{}
Lobermeier, Scott, Matthew Moldenhauer, Christopher Peter, Luke
Slominski, Richard Tedesco, Marcus Meer, James Dwyer, Richard Harness,
and Andrew Stewart. 2015. ``Mitigating Avian Collision with Power Lines:
A Proof of Concept for Installation of Line Markers via Unmanned Aerial
Vehicle.'' \emph{J. Unmanned Veh. Syst.} 3 (4): 252--258.
doi:\href{https://doi.org/10.1139/juvs-2015-0009}{10.1139/juvs-2015-0009}.

\hypertarget{ref-longmore_adapting_2017}{}
Longmore, S. N., R. P. Collins, S. Pfeifer, S. E. Fox, M.
Mulero-Pazmany, F. Bezombes, A. Goodwind, M. de Juan Ovelar, J. H.
Knapen, and S. A. Wich. 2017. ``Adapting Astronomical Source Detection
Software to Help Detect Animals in Thermal Images Obtained by Unmanned
Aerial Systems'' 00 (0): 1--16.
doi:\href{https://doi.org/10.1080/01431161.2017.1280639}{10.1080/01431161.2017.1280639}.

\hypertarget{ref-margarita_mulero-pazmany_juan_jose_negro_low_2014}{}
Margarita Mulero-Pázmány, Miguel Ferrer, Juan José Negro. 2014. ``A Low
Cost Way for Assessing Bird Risk Hazards in Power Lines: Fixed-Wing
Small Unmanned Aircraft Systems'' 2.

\hypertarget{ref-martin_estimating_2012}{}
Martin, Julien, Holly H. Edwards, Matthew A. Burgess, H. Franklin
Percival, Daniel E. Fagan, Beth E. Gardner, Joel G. Ortega-Ortiz, Peter
G. Ifju, Brandon S. Evers, and Thomas J. Rambo. 2012. ``Estimating
Distribution of Hidden Objects with Drones: From Tennis Balls to
Manatees.'' \emph{PLoS One} 7 (6): 1--8.
doi:\href{https://doi.org/10.1371/journal.pone.0038882}{10.1371/journal.pone.0038882}.

\hypertarget{ref-mateo_modelos_2011}{}
Mateo, Rubén G., Ángel M. Felicísimo, and Jesús Muñoz. 2011. ``Modelos
de Distribución de Especies: Una Revisión Sintética.'' \emph{Rev. Chil.
Hist. Nat.}, 217--240.
doi:\href{https://doi.org/10.4067/S0716-078X2011000200008}{10.4067/S0716-078X2011000200008}.

\hypertarget{ref-maza_experimental_2011}{}
Maza, Iv?n, Fernando Caballero, Jes?s Capit?n, J. R. Mart?nez-de-Dios,
and An?bal Ollero. 2011. ``Experimental Results in Multi-UAV
Coordination for Disaster Management and Civil Security Applications.''
\emph{Journal of Intelligent \& Robotic Systems} 61 (1): 563--585.
doi:\href{https://doi.org/10.1007/s10846-010-9497-5}{10.1007/s10846-010-9497-5}.

\hypertarget{ref-mccaldin_use_2015}{}
McCaldin, Guy, Michael Johnston, and Andrew Rieker. 2015. \emph{Use of
Unmanned Aircraft Systems to Assist with Decision Support for Land
Managers on Christmas Island (Indian Ocean)}. October. Australia: V-TOL
Aerospace; Department of parks; Wildlife, Western Australia.

\hypertarget{ref-mcevoy_evaluation_2016}{}
McEvoy, John F., Graham P. Hall, and Paul G. McDonald. 2016.
``Evaluation of Unmanned Aerial Vehicle Shape, Flight Path and Camera
Type for Waterfowl Surveys: Disturbance Effects and Species
Recognition.'' \emph{PeerJ} 4 (March): e1831.
doi:\href{https://doi.org/10.7717/peerj.1831}{10.7717/peerj.1831}.

\hypertarget{ref-miyamoto_use_2004}{}
Miyamoto, Michiru, Kunihiko Yoshino, Toshihide Nagano, Tomoyasu Ishida,
and Yohei Sato. 2004. ``Use of Balloon Aerial Photography for
Classification of Kushiro Wetland Vegetation, Northeastern Japan.''
\emph{Wetlands} 24 (3): 701--710.
doi:\href{https://doi.org/10.1672/0277-5212(2004)024\%5B0701:UOBAPF\%5D2.0.CO;2}{10.1672/0277-5212(2004)024{[}0701:UOBAPF{]}2.0.CO;2}.

\hypertarget{ref-mulero-pazmany_margarita_aeromab_2011}{}
Mulero-Pázmány Margarita, Negro JJ. 2011. ``AEROMAB Small UAS for
Montagu's Harrier's Circus Pygargus Nests Monitoring.'' \emph{AEROMAB
Small UAS Montagu's Harrier's Circus Pygargus Nests Monit. RED UAS
Intenational Congr. Univ. Eng. Seville, Spain. December 2011.}

\hypertarget{ref-mulero-pazmany_unmanned_2015}{}
Mulero-Pázmány, Margarita, Jose Ángel Barasona, Pelayo Acevedo, Joaquín
Vicente, and Juan José Negro. 2015. ``Unmanned Aircraft Systems
Complement Biologging in Spatial Ecology Studies.'' \emph{Ecol. Evol.} 5
(21): 4808--4818.
doi:\href{https://doi.org/10.1002/ece3.1744}{10.1002/ece3.1744}.

\hypertarget{ref-mulero-pazmany_unmanned_2017}{}
Mulero-Pázmány, Margarita, Susanne Jenni-Eiermann, Nicolas Strebel,
Thomas Sattler, Juan José Negro, and Zulima Tablado. 2017. ``Unmanned
Aircraft Systems as a New Source of Disturbance for Wildlife: A
Systematic Review.'' \emph{PloS One} 12 (6): e0178448.
\url{http://journals.plos.org/plosone/article?id=10.1371/journal.pone.0178448}.

\hypertarget{ref-mulero-pazmany_remotely_2014}{}
Mulero-Pázmány, Margarita, Roel Stolper, L. D. Van Essen, Juan J. Negro,
and Tyrell Sassen. 2014. ``Remotely Piloted Aircraft Systems as a
Rhinoceros Anti-Poaching Tool in Africa.'' \emph{PLoS One} 9 (1): 1--10.
doi:\href{https://doi.org/10.1371/journal.pone.0083873}{10.1371/journal.pone.0083873}.

\hypertarget{ref-mullerova_does_2016}{}
Müllerová, Jana, Josef Brůna, Dvořák, Peter, Tomáš Bartaloš, and
Michaela Vítková. 2016. ``DOES THE DATA RESOLUTION/ORIGIN MATTER?
SATELLITE, AIRBORNE AND UAV IMAGERY TO TACKLE PLANT INVASIONS.''
\emph{ISPRS - International Archives of the Photogrammetry, Remote
Sensing and Spatial Information Sciences} XLI-B7 (June): 903--908.
doi:\href{https://doi.org/10.5194/isprsarchives-XLI-B7-903-2016}{10.5194/isprsarchives-XLI-B7-903-2016}.

\hypertarget{ref-nugraha_urgency_2016}{}
Nugraha, Ridha Aditya, Deepika Jeyakodi, and Thitipon Mahem. 2016.
``Urgency for Legal Framework on Drones : Lessons for Indonesia , India
, and Thailand.'' \emph{Indones. Law Rev.} 6 (2): 137--157.

\hypertarget{ref-pereira_essential_2013}{}
Pereira, Henrique Miguel, Simon Ferrier, Michele Walters, Gary N Geller,
Rob H G Jongman, Robert J Scholes, Michael W Bruford, et al. 2013.
``Essential Biodiversity Variables.'' \emph{Science (80-. ).} 339
(6117): 277--278.
doi:\href{https://doi.org/10.1126/science.1229931}{10.1126/science.1229931}.

\hypertarget{ref-perroy_assessing_2017}{}
Perroy, Ryan L., Timo Sullivan, and Nathan Stephenson. 2017. ``Assessing
the Impacts of Canopy Openness and Flight Parameters on Detecting a
Sub-Canopy Tropical Invasive Plant Using a Small Unmanned Aerial
System.'' \emph{ISPRS J. Photogramm. Remote Sens.} 125: 174--183.
doi:\href{https://doi.org/10.1016/j.isprsjprs.2017.01.018}{10.1016/j.isprsjprs.2017.01.018}.

\hypertarget{ref-pimm_emerging_2015}{}
Pimm, Stuart L, Sky Alibhai, Richard Bergl, Alex Dehgan, Chandra Giri,
Zoë Jewell, Lucas Joppa, Roland Kays, and Scott Loarie. 2015. ``Emerging
Technologies to Conserve Biodiversity.'' \emph{Trends Ecol. Evol.} 30
(11): 685--696.
doi:\href{https://doi.org/10.1016/j.tree.2015.08.008}{10.1016/j.tree.2015.08.008}.

\hypertarget{ref-pomeroy_assessing_2015}{}
Pomeroy, P., L. O'Connor, and P. Davies. 2015. ``Assessing Use of and
Reaction to Unmanned Aerial Systems in Gray and Harbor Seals During
Breeding and Molt in the UK \(^{\textrm{1}}\).'' \emph{Journal of
Unmanned Vehicle Systems} 3 (3): 102--113.
doi:\href{https://doi.org/10.1139/juvs-2015-0013}{10.1139/juvs-2015-0013}.

\hypertarget{ref-puttock_aerial_2015}{}
Puttock, A.K., A.M. Cunliffe, K. Anderson, and R.E. Brazier. 2015.
``Aerial Photography Collected with a Multirotor Drone Reveals Impact of
Eurasian Beaver Reintroduction on Ecosystem Structure 1.'' \emph{J.
Unmanned Veh. Syst.} 3 (3): 123--130.
doi:\href{https://doi.org/10.1139/juvs-2015-0005}{10.1139/juvs-2015-0005}.

\hypertarget{ref-quilter_low_2000}{}
Quilter, MARK C., and Val Jo Anderson. 2000. ``Low Altitude/Large Scale
Aerial Photographs: A Tool for Range and Resource Managers.''
\emph{Rangelands Archives} 22 (2): 13--17.
\url{https://journals.uair.arizona.edu/index.php/rangelands/article/download/11454/10727}.

\hypertarget{ref-rodriguez_eye_2012}{}
Rodríguez, Airam, Juan J. Negro, Mara Mulero, Carlos Rodríguez, Jesús
Hernández-Pliego, and Javier Bustamante. 2012. ``The Eye in the Sky:
Combined Use of Unmanned Aerial Systems and GPS Data Loggers for
Ecological Research and Conservation of Small Birds.'' \emph{PLoS One} 7
(12).
doi:\href{https://doi.org/10.1371/journal.pone.0050336}{10.1371/journal.pone.0050336}.

\hypertarget{ref-sabella_preliminary_2017}{}
Sabella, Giorgio, Fabio Massimo Viglianisi, Sergio Rotondi, and
Filadelfo Brogna. 2017. ``Preliminary Observations on the Use of Drones
in the Environmental Monitoring and in the Management of Protected
Areas. the Case Study of `RNO Vendicari', Syracuse (Italy).'' bibtex:
sabella\_preliminary\_2017. \emph{Biodiversity Journal} 8: 79--86.
\url{http://www.biodiversityjournal.com/pdf/8(1)_79-86.pdf}.

\hypertarget{ref-sarda-palomera_fine-scale_2012}{}
Sardà-Palomera, Francesc, Gerard Bota, Carlos Viñolo, Oriol Pallarés,
Víctor Sazatornil, Lluís Brotons, Spartacus Gomáriz, and Francesc Sardà.
2012. ``Fine-Scale Bird Monitoring from Light Unmanned Aircraft
Systems.'' \emph{Ibis (Lond. 1859).} 154 (1): 177--183.
doi:\href{https://doi.org/10.1111/j.1474-919X.2011.01177.x}{10.1111/j.1474-919X.2011.01177.x}.

\hypertarget{ref-sasse_job-related_2003}{}
Sasse, D. Blake. 2003. ``Job-Related Mortality of Wildlife Workers in
the United States, 1937-2000.'' \emph{Wildl. Soc. Bull.} 31 (4):
1000--1003.

\hypertarget{ref-schmale_development_2008}{}
Schmale, DG, Benjamin R. Dingus, and Charles Reinholtz. 2008.
``Development and Application of an Autonomous Unmanned Aerial Vehicle
for Precise Aerobiological Sampling Above Agricultural Fields.''
\emph{J. F. Robot.} 25 (3): 133--147.
doi:\href{https://doi.org/10.1002/rob}{10.1002/rob}.

\hypertarget{ref-schwarzbach_remote_2014}{}
Schwarzbach, Marc, Maximilian Laiacker, Margarita Mulero-Pazmany, and
Konstantin Kondak. 2014. ``Remote Water Sampling Using Flying Robots.''
\emph{2014 Int. Conf. Unmanned Aircr. Syst. ICUAS 2014 - Conf. Proc.},
72--76.
doi:\href{https://doi.org/10.1109/ICUAS.2014.6842240}{10.1109/ICUAS.2014.6842240}.

\hypertarget{ref-scobie_wildlife_2016}{}
Scobie, Corey A., and Chris H. Hugenholtz. 2016. ``Wildlife Monitoring
with Unmanned Aerial Vehicles: Quantifying Distance to Auditory
Detection: UAV Sound and Wildlife Aural Detection.'' \emph{Wildlife
Society Bulletin} 40 (4): 781--785.
doi:\href{https://doi.org/10.1002/wsb.700}{10.1002/wsb.700}.

\hypertarget{ref-secretariat_of_the_convention_on_biological_diversity_global_2006}{}
Secretariat of the Convention on Biological Diversity, and UNEP World
Conservation Monitoring Centre, eds. 2006. \emph{Global Biodiversity
Outlook 2}. Montreal: Secretariat of the Convention on Biological
Diversity.

\hypertarget{ref-sepulveda_review_2010}{}
Sepúlveda, Alejandra, Mathias Schluep, Fabrice G. Renaud, Martin
Streicher, Ruediger Kuehr, Christian Hagelüken, and Andreas C. Gerecke.
2010. ``A Review of the Environmental Fate and Effects of Hazardous
Substances Released from Electrical and Electronic Equipments During
Recycling: Examples from China and India.'' \emph{Environmental Impact
Assessment Review} 30 (1): 28--41.
doi:\href{https://doi.org/10.1016/j.eiar.2009.04.001}{10.1016/j.eiar.2009.04.001}.

\hypertarget{ref-soriano_rf-based_2009}{}
Soriano, P, F Caballero, and A Ollero. 2009. ``RF-Based Particle Filter
Localization for Wildlife Tracking by Using an UAV.'' \emph{40 Th Int.
Symp. Robot.}, 239--244.
\url{http://grvc.us.es/publica/congresosint/documentos/isr_soriano.pdf}.

\hypertarget{ref-stocker_review_2017}{}
Stöcker, Claudia, Rohan Bennett, Francesco Nex, Markus Gerke, and Jaap
Zevenbergen. 2017. ``Review of the Current State of UAV Regulations.''
\emph{Remote Sensing} 9 (5): 459.
doi:\href{https://doi.org/10.3390/rs9050459}{10.3390/rs9050459}.

\hypertarget{ref-szantoi_mapping_2017}{}
Szantoi, Zoltan, Scot E. Smith, Giovanni Strona, Lian Pin Koh, and Serge
A. Wich. 2017. ``Mapping Orangutan Habitat and Agricultural Areas Using
Landsat OLI Imagery Augmented with Unmanned Aircraft System Aerial
Photography.'' \emph{Int. J. Remote Sens.} 38 (8): 1--15.
doi:\href{https://doi.org/10.1080/01431161.2017.1280638}{10.1080/01431161.2017.1280638}.

\hypertarget{ref-tomlins_remotely_1983}{}
Tomlins, G.F., and Y.J. Lee. 1983. ``Remotely Piloted Aircraft --- an
Inexpensive Option for Large-Scale Aerial Photography in Forestry
Applications.'' \emph{Can. J. Remote Sens.} 9 (2): 76--85.
doi:\href{https://doi.org/10.1080/07038992.1983.10855042}{10.1080/07038992.1983.10855042}.

\hypertarget{ref-AUVSI2017}{}
UAVSI. 2017. ``\emph{Association for Unmanned Vehicle Systems
International}.'' \url{http://www.auvsi.org} {[}Accessed: 27 July,
2017{]}.

\hypertarget{ref-USGS2014}{}
USGS. 2014. ``\emph{US Geological Survey National Unmanned Aircraft
Systems Project}.''
\href{http://\%20rmgsc.cr.usgs.gov/UAS}{http:// rmgsc.cr.usgs.gov/UAS}
{[}Accessed: 13 April, 2017{]}.

\hypertarget{ref-van_tilburg_first_2017}{}
Van Tilburg, Christopher. 2017. ``First Report of Using Portable
Unmanned Aircraft Systems (Drones) for Search and Rescue.''
\emph{Wilderness \& Environmental Medicine}.
\url{http://www.sciencedirect.com/science/article/pii/S1080603217300042}.

\hypertarget{ref-vas_approaching_2015}{}
Vas, E., A. Lescroel, O. Duriez, G. Boguszewski, and D. Gremillet. 2015.
``Approaching Birds with Drones: First Experiments and Ethical
Guidelines.'' \emph{Biol. Lett.} 11 (2): 20140754--20140754.
doi:\href{https://doi.org/10.1098/rsbl.2014.0754}{10.1098/rsbl.2014.0754}.

\hypertarget{ref-ventura_low-cost_2016}{}
Ventura, Daniele, Michele Bruno, Giovanna Jona Lasinio, Andrea
Belluscio, and Giandomenico Ardizzone. 2016. ``A Low-Cost Drone Based
Application for Identifying and Mapping of Coastal Fish Nursery
Grounds.'' \emph{Estuar. Coast. Shelf Sci.} 171.
doi:\href{https://doi.org/10.1016/j.ecss.2016.01.030}{10.1016/j.ecss.2016.01.030}.

\hypertarget{ref-weissensteiner_low-budget_2015}{}
Weissensteiner, M. H., J. W. Poelstra, and J. B. W. Wolf. 2015.
``Low-Budget Ready-to-Fly Unmanned Aerial Vehicles: An Effective Tool
for Evaluating the Nesting Status of Canopy-Breeding Bird Species.''
bibtex: weissensteiner\_low-budget\_2015. \emph{Journal of Avian
Biology} 46 (4): 425--430.
doi:\href{https://doi.org/10.1111/jav.00619}{10.1111/jav.00619}.

\hypertarget{ref-WichS2017}{}
Wich, S. 2017. ``\emph{Amazon River Dolphin Project}.''
\url{https://conservationdrones.org/2017/04/05/amazon-river-dolphin-project/}
{[}Accessed: 07 Abril, 2017{]}.

\hypertarget{ref-wich_preliminary_2016}{}
Wich, Serge, David Dellatore, Max Houghton, Rio Ardi, and Lian Pin Koh.
2016. ``A Preliminary Assessment of Using Conservation Drones for
Sumatran Orang-Utan (Pongo Abelii) Distribution and Density.'' \emph{J.
Unmanned Veh. Syst.} 4 (1): 45--52.
doi:\href{https://doi.org/10.1139/juvs-2015-0015}{10.1139/juvs-2015-0015}.

\hypertarget{ref-wilson_feasibility_2017}{}
Wilson, Andrew M, Janine Barr, and Megan Zagorski. 2017. ``The
Feasibility of Counting Songbirds Using Unmanned Aerial Vehicles.''
\emph{Auk} 134 (2): 350--362.
doi:\href{https://doi.org/10.1642/AUK-16-216.1}{10.1642/AUK-16-216.1}.

\hypertarget{ref-wilson_measuring_2006}{}
Wilson, Rory P., and Clive R. McMahon. 2006. ``Measuring Devices on Wild
Animals: What Constitutes Acceptable Practice?'' \emph{Frontiers in
Ecology and the Environment} 4 (3): 147--154.
doi:\href{https://doi.org/10.1890/1540-9295(2006)004\%5B0147:MDOWAW\%5D2.0.CO;2}{10.1890/1540-9295(2006)004{[}0147:MDOWAW{]}2.0.CO;2}.

\hypertarget{ref-wulder_high_2004}{}
Wulder, Michael A, Ronald J Hall, Nicholas C Coops, and Steven E
Franklin. 2004. ``High Spatial Resolution Remotely Sensed Data for
Ecosystem Characterization'' 54 (6): 511--521.
doi:\href{https://doi.org/10.1641/0006-3568(2004)054}{10.1641/0006-3568(2004)054}.

\hypertarget{ref-zahawi_using_2015}{}
Zahawi, Rakan A., Jonathan P. Dandois, Karen D. Holl, Dana Nadwodny, J.
Leighton Reid, and Erle C. Ellis. 2015. ``Using Lightweight Unmanned
Aerial Vehicles to Monitor Tropical Forest Recovery.'' \emph{Biol.
Conserv.} 186 (June): 287--295.
doi:\href{https://doi.org/10.1016/j.biocon.2015.03.031}{10.1016/j.biocon.2015.03.031}.

\hypertarget{ref-zaman_use_2011}{}
Zaman, Bushra, Austin M. Jensen, and Mac McKee. 2011. ``Use of
High-Resolution Multispectral Imagery Acquired with an Autonomous
Unmanned Aerial Vehicle to Quantify the Spread of an Invasive Wetlands
Species.'' \emph{Int. Geosci. Remote Sens. Symp.}, 803--806.
doi:\href{https://doi.org/10.1109/IGARSS.2011.6049252}{10.1109/IGARSS.2011.6049252}.

\hypertarget{ref-zang_investigating_2012}{}
Zang, Wenqian, Jiayuan Lin, Yangchun Wang, and Heping Tao. 2012.
``Investigating Small-Scale Water Pollution with UAV Remote Sensing
Technology.'' In \emph{World Automation Congress (WAC), 2012}, 1--4.
IEEE. \url{http://ieeexplore.ieee.org/abstract/document/6321515/}.

\newpage
\singlespacing 
\end{document}
